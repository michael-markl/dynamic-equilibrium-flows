
\documentclass[envcountsect]{beamer}
\usetheme[faculty=phil, headheight=0em, fonts=none]{fibeamer}


\usepackage[utf8]{inputenc} % for input encoding
\usepackage[german]{babel} % for german localization
\usepackage{colonequals} % for :=
\usepackage{graphicx} % for \includegraphics
\usepackage{wrapfig} % for floating images to the right
\usepackage{tikz}
\usepackage{esvect}
\usepackage{tabto}
\usepackage{enumitem}

\usetikzlibrary{arrows, shapes}


\usepackage{mathabx}
%-------------------------------------------------------------------------------
% Hilfreiche Befehle
%-------------------------------------------------------------------------------
\newcommand{\betrag}[1]{\lvert #1 \rvert}	        % Betrag
\providecommand*{\Lfloor}{\left\lfloor}                 % gro\ss{}es Abrunden
\providecommand*{\Rfloor}{\right\rfloor}                % gro\ss{}es Abrunden
\providecommand*{\Floor}[1]{\Lfloor #1 \Rfloor}         % gro\ss{}es ganzes Abrunden
\providecommand*{\Ceil}[1]{\left\lceil #1 \right\rceil} % gro\ss{}es ganzes Aufrunden

\newcommand{\Z}{\mathbb{Z}}
\newcommand{\N}{\mathbb{N}}
\newcommand{\R}{\mathbb{R}}
\newcommand{\Q}{\mathbb{Q}}
\newcommand{\firstNumbers}[1]{[#1]}
\newcommand{\transpose}{^\intercal}
\newcommand{\subjectTo}{\textbf{s.t.}}
\newcommand{\MIPR}{MIP\textsuperscript{*}}
\newcommand{\MIPI}{MIP}
\newcommand{\oBdA}{oBdA.}
\newcommand{\rang}{\operatorname{rang}}
\newcommand{\norm}[1]{\left\lVert#1\right\rVert_\infty}
\newcommand{\zero}{0}
\newcommand{\todo}[1]{{\color{red}{\emph{TODO: }}#1}}
\newcommand{\one}{\mathbbm{1}}
\newcommand{\eq}[1]{{\operatorname{eq}(#1)}}
\newcommand{\co}[1]{\operatorname{co}(#1)}

\setbeamertemplate{theorems}[numbered]
\newtheorem{conjecture}[theorem]{Vermutung}
\newtheorem{korollar}[theorem]{Korollar}
\newtheorem{beispiel}[theorem]{Beispiel}
\newtheorem{proposition}[theorem]{Proposition}

\definecolor{darkblue}{HTML}{00446B}


\newcommand{\coloniff}{\vcentcolon\Longle ftrightarrow}



\makeatletter
\setlength\fibeamer@lengths@logowidth{0em}
\setlength\fibeamer@lengths@logoheight{0em}
\makeatother
\useoutertheme{infolines}
\newenvironment{noheadline}{
	\setbeamertemplate{headline}{}
}{}
\newenvironment{nofootline}{
\setbeamertemplate{footline}{}
}{}

\setbeamersize{text margin left=2.7em, text margin right=2.7em}
\setbeamertemplate{frametitle}{\insertframetitle}
\setbeamercolor{block body}{fg=black!90}


\usefonttheme{professionalfonts}
\setbeamerfont{title}{size=\huge}
\setbeamertemplate{institute}{\insertinstitute}

\AtBeginSection{
	\begin{noheadline}
		\begin{frame}
		\vfill
		\centering
		\begin{beamercolorbox}[sep=8pt,center]{title}
			\usebeamerfont{title}\insertsectionhead\par%
		\end{beamercolorbox}
		\vfill
	\end{frame}
	\end{noheadline}

}

\makeatletter
\setbeamertemplate{footline}{%
	\color{darkblue}% to color the progressbar
	\small
	\hfill{\insertframenumber\hspace{1.5em}}
	\vspace{1em}


	\hspace*{-\beamer@leftmargin}%
	\rule{\beamer@leftmargin}{1pt}%
	\rlap{\rule{\dimexpr\numexpr0\insertframenumber\dimexpr
			\textwidth\relax/\numexpr0\inserttotalframenumber}{1pt}}
	% next 'empty' line is mandatory!

	\vspace{0\baselineskip}
	{}
}


\setbeamertemplate{bibliography item}{\insertbiblabel}

\title{Nash Gleichgewichte in Dynamischen Flüssen}
\subtitle{Seminar zur Optimierung und Spieltheorie}
\author{~\\Michael Markl \\ 27. Juni 2019}
\date{08.11.2018}
\institute{Insitut für Mathematik der Universität Augsburg\\Diskrete Mathematik, Optimierung und Operations Research}

\renewcommand{\[}{
	\setlength\abovedisplayskip{0.5ex}
	\setlength{\belowdisplayskip}{0.5ex}
	\setlength{\abovedisplayshortskip}{0.5ex}
	\setlength{\belowdisplayshortskip}{0.5ex}\begin{equation*}}

\renewcommand{\]}{\end{equation*}}

\newcommand*\diff{\mathop{}\!\mathrm{d}}
%\setlist[enumerate]{topsep=0.5ex,itemsep=0ex,partopsep=0ex,parsep=0.8ex}


\beamertemplatenavigationsymbolsempty

\begin{document}

	\setcounter{framenumber}{-1}

	\begin{nofootline}
		\frame{\titlepage}
	\end{nofootline}

	\begin{frame}{Gliederung}
		\tableofcontents
	\end{frame}

	\chapter{Dynamische Flüsse}\label{chapter-dynamic-flows}

\section{Grundlegende Definitionen}

Zunächst werden einige grundlegende Begriffe eingeführt.
In der gesamten Arbeit werden grundsätzlich nur gerichtete Graphen mit endlicher Knoten- und Kantenmenge betrachtet.
Dabei bezeichne $\tail(e)$ den Start- und $\head(e)$ den Zielknoten einer Kante $e$.
Außerdem sind parallele Kanten zwischen Knoten stets erlaubt, obwohl häufig die vereinfachte Schreibweise $vw\coloneq (v,w)\coloneq e$ mit $v=\tail(e)$ und $w=\head(e)$ benutzt wird.

Für eine Menge $X$ an Knoten bezeichne $\delta^+(X)\coloneq \{ e\in E \mid \tail(e) \in X \notni \head(e) \}$ die Menge der ausgehenden Kanten von~$X$ und analog bezeichne $\delta^-(X)$ die Menge der eingehenden Kanten in~$X$.
Ist der zugehörige Graph unklar, so schreibt man $\delta^+_M(X)$ bzw. $\delta^-_M(X)$ für ein Netzwerk, einen Graphen oder eine Kantenmenge $M$.
Für die ausgehenden oder eingehenden Kanten eines einzelnen Knotens schreibt man verkürzend $\delta^+(v)$ bzw. $\delta^-(v)$.

Ist weiter $f: E \rightarrow \R$ eine Kantenbewertung und $b: V \rightarrow \R$ eine Knotenbewertung, so schreibt man meist $f_e$ statt $f(e)$ und $b_v$ statt $b(v)$ und für Teilmengen $E'\subseteq E$ und $X\subseteq V$ abkürzend
\[ 
	f(E')\coloneq \sum_{e \in E'} f_e \text{~~~ und ~~~} b(X)\coloneq\sum_{v\in X} b_v.
\]

Ein \emph{Pfad} $P=(e_1, \dots, e_k)$ ist eine Aneinanderreihung von Kanten, das heißt, es gilt $\head(e_i) = \tail(e_{i+1})$ für alle $i\in[k-1]$.
Dabei bezeichnet $[n]$ die Menge der ersten $n$ natürlichen Zahlen, also $\{ 1,\dots, n \}$.
Die Kanten eines Pfades werden in der Menge $E(P)$, die Knoten in der Menge $V(P)$ gesammelt.
Ein Pfad heißt \emph{Weg}, falls kein Knoten mehrmals besucht wird. 
Ein Pfad heißt \emph{Kreis}, falls $\tail(e_1) = \head(e_k)$ gilt, und \emph{Zyklus} oder \emph{elementarer Kreis}, falls zusätzlich $\tail(e_1)$ genau zweimal und sonst kein Knoten mehrmals besucht werden.
Ein Weg oder Zyklus $P$ wird oft als Vektor in $\R^E$ aufgefasst, der als Einträge $1$-en auf Kanten in $P$ und sonst $0$-en enthält.

\begin{definition}
	Ein \emph{Netzwerk $(V, E, u)$} ist ein gerichteter, endlicher Graph $(V, E, u)$ mit \emph{Kapazitäten} $u: E \to \R_{>0}$.
	Im Falle statischer Flüsse ist ein Netzwerk häufig mit Balancen $b:V\to\R$ ausgestattet.
	
	Ein \emph{dynamisches Netzwerk} $(V, E, u, s, t, \tau)$ ist ein Netzwerk $(V, E, u)$, in dem $s$ alle Knoten in $V$ erreicht.
	Dabei heißen $s\in V$ die \emph{Quelle} und $t\in V$ die \emph{Senke} des Netzwerks.
	Jeder Kante $e\in E$ wird zusätzlich eine \emph{Verzögerungszeit} $\tau_e\geq 0$ zugeordnet, wobei alle Zyklen $C$ eine positive Gesamtverzögerung $\tau(E(C))$ haben.
\end{definition}
\begin{definition}[Statischer Fluss]
	Eine Kantenbewertung $f\in\R_{\geq 0}^E$ in einem gerichteten Graphen $(V, E)$ heißt \emph{statischer Fluss}.
	Sein \emph{Balancevektor $b\in \R^V$} ist gegeben durch
	\[ b_v \coloneq f(\delta^+(v)) - f(\delta^-(v)) \text{~~~ für $v\in V$}. \]
	Man nennt $f$ auch einen \emph{$b$-Fluss} oder sagt, \emph{$f$ gewähre Flusserhaltung bzgl. $b$}.
	Gibt es zwei Knoten $s$ und $t$, sodass $b_v$
	für alle Knoten $v\in V\setminus\{ s, t \}$ verschwindet, für $s$ nicht-negativ und für $t$ nicht-positiv ist, so ist $f$ ein \emph{statischer $s$-$t$-Fluss mit Wert $b_s$}.
	Ist $b$ der Nullvektor, so nennt man $f$ auch eine \emph{Strömung}.
\end{definition}

Man bemerke, dass ein statischer Fluss keine Kapazitätsbedingung erfüllen muss.
Außerdem sind für eine Strömung nur nichtnegative Werte zugelassen.
Eine wichtige Aussage bei der Anaylse von statischen Flüssen liefert der Dekompositionssatz (siehe \cite[Satz 8.8]{Korte2012}):
\begin{theorem}[Dekompositionssatz]\label{thm-decomposition}
	Ein statischer $s$-$t$-Fluss $f$ besitzt eine Dekomposition in elementare $s$-$t$-Wege und Zyklen, das heißt es existieren $s$-$t$-Wege $P_1,\dots,P_k$ und Zyklen $C_1, \dots, C_l$ sowie $\lambda_1,\dots,\lambda_k,\mu_1,\dots,\mu_l > 0$ mit \[
		f = \sum_{i\in[k]}\lambda_i P_i + \sum_{i\in[l]} \mu_i C_i.
	\]
	Ist $f$ eine Strömung, so gibt es eine Dekomposition in Zyklen.
\end{theorem}

Die Einführung dynamischer Flüsse und kürzester Wege folgt im Wesentlichen der Darstellung von Cominetti, Correa und Larré aus~\cite{Cominetti2015} mit Ergänzungen von Ronald Koch und Martin Skutella aus~\cite{Koch2011}.
So haben Cominetti u. a. dynamische Flüsse beispielsweise statt mit Lebesgue-integrierbarer Funktionen mit lokal Lebesgue-integrierbaren Funktionen ausgestattet, die den Vorteil bieten, über den gesamten Zeitraum $[0, \infty)$ unendlich viel Fluss schicken zu können.
Dazu wird der folgende Funktionenraum eingeführt:

\begin{definition}
	Der Raum $\mathfrak{F}_0$ sei die Menge der Funktionen $g: \R \to \R_{\geq 0}$, die lokal integrierbar bzgl. des Lebesgue-Maßes sind und auf der negativen Achse verschwinden, die also $\int_a^b |g(t)| \diff t< \infty$ für beliebige beschränkte Intervalle $(a,b)$ und $g(t)=0$ für $t<0$ erfüllen.
\end{definition}

\begin{definition}[Dynamischer Fluss]
	Ein \emph{dynamischer Fluss $f=(f^+, f^-)$} ist ein Paar zweier über die Kanten $E$ eines dynamischen Netzwerks indizierter Familien mit $f^+_e,f^-_e\in\mathfrak F_0$ für alle $e\in E$.
	Dabei bezeichnen $f_e^+(\theta)$ und $f_e^-(\theta)$ die \emph{Zu- bzw. Abflussrate an Kante $e\in E$ zum Zeitpunkt $\theta\in\R$}.
	
	Der (kumulative) \emph{Zu- bzw. Abfluss an einer Kante $e$ bis zum Zeitpunkt $\theta$} sei definiert durch $F^+_e(\theta)\coloneq\int_0^\theta f^+_e(t) \diff t<\infty$ bzw. $F^-_e(\theta)\coloneq\int_0^\theta f^-_e(t) \diff t<\infty$.
	
	Die \emph{(Länge der) Warteschlange $z_e(\theta)$ und die Wartezeit $q_e(\theta)$ an einer Kante $e\in E$ zum Zeitpunkt $\theta\in\R$} seien gegeben durch $z_e(\theta)\coloneq F_e^+(\theta) - F_e^-(\theta + \tau_e)$ und $q_e(\theta) \coloneq z_e(\theta) / u_e$.
	
	Man bezeichne die \emph{Austrittszeit $T_e(\theta)$ aus einer Kante $e\in E$ bei Eintrittszeit $\theta$}, zu der ein Partikel eine Kante verlässt, die es zum Zeitpunkt $\theta$ betreten hat, als $T_e(\theta)\coloneq\theta + q_e(\theta) + \tau_e$.
\end{definition}

Der Definition der Austrittszeit kann man bereits entnehmen, wie sich Partikel, die an einer Kante $e$ ankommen, verhalten sollen:
Nachdem sie zur Zeit $\theta$ die Kante betreten haben, müssen sie sich zunächst in eine Warteschlange einreihen, welche mit der Kapazität $u_e$ nach dem FIFO-Prinzip (First-In-First-Out-Prinzip) abgebaut wird.
Nachdem diese Wartezeit $q_e(\theta)$ vorüber ist, vergeht eine weitere konstante Verzögerungszeit $\tau_e$, bevor sie wieder aus der Kante austreten.
Des Weiteren müssen sich die Partikel bereits sofort bei der Ankunft an einem Knoten entscheiden, in welche Kante sie eintreten wollen, und können nicht an einem Knoten verweilen.
Um das beschriebene Verhalten zu gewährleisten, führt man im Folgenden die Zulässigkeit dynamischer Flüsse ein:

\begin{definition}[Zulässiger dynamischer Fluss]
	Ein dynamischer Fluss $f$ heißt \emph{zulässig}, falls er die folgenden Eigenschaften erfüllt:
	\begin{enumerate}[label=(F\arabic*)]
		\item\label{def-feasible-flow-capacity} Keine Abflussrate übersteigt die Kapazität, d.h. $\forall e\in E, \theta\in\R: f_e^-(\theta)\leq u_e$.
		\item\label{def-feasible-flow-no-negative-flow} Fluss verlässt eine Kante nur, falls er sie zuvor betreten hat,\\ d.h. $\forall e\in E, \theta\in\R: F_e^+(\theta) \geq F_e^-(\theta + \tau_e).$
		\item\label{def-feasible-flow-no-flow-at-node} Bis auf Quelle und Senke erfüllt jeder Knoten $v$ Flusserhaltung,\\
		d.h. $\forall\theta\in\R: \sum_{e\in\delta^+(v)}f^+_e(\theta) - \sum_{e\in\delta^-(v)} f_e^-(\theta) = 0$.\\
		Für die Senke $t$ muss dieser Wert nicht-positiv und für die Quelle $s$ nicht-negativ sein. 
		Für $s$ bezeichnet er den \emph{Zufluss $d(\theta)$ in das Netzwerk}.
		\item\label{def-feasible-flow-queue-with-capacity} Warteschlangen werden mit der Kapazität der Kante abgebaut,\\ d.h. $\forall e\in E, \theta\in\R: z_e(\theta) > 0 \implies f_e^-(\theta + \tau_e) = u_e$.
	\end{enumerate}
\end{definition}

Die folgende Proposition beschreibt wichtige Folgerungen über zulässige dynamische Flüsse:

\begin{proposition}\label{prop-feasible-flow}
	Für eine Kante $e\in E$ und einen zulässigen dynamischen Fluss $f$ gilt:
	\begin{enumerate}[label=(\roman*)]
		\item\label{prop-feasible-flow-T-mon-inc-cont} Die Funktion $\theta \mapsto \theta + q_e(\theta)$ ist monoton wachsend und stetig.
		\item\label{prop-feasible-flow-positive-queue} Für alle $\theta\in\R$ ist die Warteschlange $z_e$ auf dem Intervall $(\theta, \theta + q_e(\theta))$ positiv.
		\item\label{prop-feasible-flow-det-outflow} Zu jeder Zeit $\theta\in\R$ gilt $F_e^+(\theta) = F_e^-(T_e(\theta))$.
		\item\label{prop-feasible-flow-queue-delay} Für alle $\theta_1 \leq \theta_2$ mit $\int_{\theta_1}^{\theta_2} f^+_e(t) \diff t = 0$ und $q_e(\theta_2)>0$ gilt $\theta_1 + q_e(\theta_1) = \theta_2 + q_e(\theta_2)$.
	\end{enumerate}
\end{proposition}
\begin{proof}
	In~\ref{prop-feasible-flow-T-mon-inc-cont} folgt die Stetigkeit bereits aus der Stetigkeit von $F_e^+$ und $F_e^-$.
	Um zu zeigen, dass die Funktion monoton wachsend ist, seien $\theta_1 \leq \theta_2$ gegeben.
	Mit $F_e^-(\theta_2 + \tau_e) = F_e^-(\theta_1+\tau_e) + \int_{\theta_1+\tau_e}^{\theta_2+\tau_e} f_e^-(t)\diff t\leq F_e^-(\theta_1 + \tau_e) + (\theta_2 - \theta_1)u_e$ und mit der Monotonie von $F_e^+$ folgt: 
	\[
		\theta_1 + q_e(\theta_1)
		= \theta_1 + \frac{F_e^+(\theta_1) - F_e^-(\theta_1 + \tau_e)}{u_e}
		\leq \theta_1 + \frac{F_e^+(\theta_2) - F_e^-(\theta_1+\tau_e)}{u_e} \leq \theta_2 + q_e(\theta_2).
	\]
	
	Für $\theta'\in (\theta, \theta+q_e(\theta))$ gilt also $\theta' + q_e(\theta') \geq \theta + q_e(\theta)$, womit $q_e(\theta') \geq \theta + q_e(\theta) - \theta' > 0$ gerade Aussage (ii) beweist.
	
	Aussage (iii) folgt dann mit~\ref{def-feasible-flow-queue-with-capacity} und Aussage~(ii), weil daraus
	\[ 
	\int_{\theta}^{\theta + q_e(\theta)}f_e^-(t + \tau_e) \diff t = q_e(\theta)  u_e = z_e(\theta)
	\]
	folgt, weshalb $F_e^-(T_e(\theta)) = F_e^-(\theta+\tau_e) + \int_{\theta+\tau_e}^{\theta+\tau_e+q_e(\theta)}f_e^-(t)\diff t = F_e^+(\theta)$ gilt.
	
	Zu Aussage (iv):
	Für alle $\theta'\in [\theta_1, \theta_2]$ gilt $F_e^+(\theta') = F_e^+(\theta_2)$.
	Also ist die Warteschlange $z_e(\theta') = F_e^+(\theta_2) - F_e^-(\theta' + \tau_e) \geq z_e(\theta_2)$ positiv und nach~\ref{def-feasible-flow-queue-with-capacity} gilt $f_e^-(\theta' + \tau_e)=u_e$.
	Die Differenz der Warteschlangen erfüllt
	\[
	z_e(\theta_1)-z_e(\theta_2)=-F^-_e(\theta_1 + \tau_e) + F^-_e(\theta_2 + \tau_e) = (\theta_2 - \theta_1)u_e,
	\]
	was $q_e(\theta_1) - q_e(\theta_2) = \theta_2 - \theta_1$ impliziert.
\end{proof}

\section{Kürzeste Wege}\label{sec-travel-times}

In diesem Abschnitt wird der Begriff der frühesten Ankunftszeit an einem Knoten eingeführt und erörtert, wann eine Kante $vw$ in einem kürzesten $s$-$w$-Pfad liegt.

\begin{definition}
	Für einen dynamischen Fluss $f$ und einen Pfad $P=(e_1,\dots,e_k)$ definiere $l^P(\theta)\coloneq T_{e_k}\circ\dots\circ T_{e_1}(\theta)$ den Zeitpunkt, an dem ein Partikel den Endknoten des Pfads erreicht, falls es den Pfad zum Zeitpunkt $\theta$ betritt.
	
	Für einen Knoten $w\in V$ beschreibe $\mathcal{P}_w$ die Menge aller $s$-$w$-Pfade.
	Dann ist die früheste Ankunft eines Partikels, das zur Zeit $\theta$ bei $s$ startet, gegeben durch $l_w(\theta)\coloneq \min_{P\in\mathcal{P}_w}l^P(\theta)$.
	Ein Pfad $P\in \mathcal{P}_w$ heißt \emph{kürzester $s$-$w$-Pfad zur Zeit $\theta$}, falls er $l_w(\theta)=l^P(\theta)$ erfüllt.
\end{definition}

\begin{proposition}\label{prop-abs-cont-sur}
	Für einen zulässigen Fluss $f$ sind die Funktionen $F_e^+$ und $F_e^-$ für alle $e\in E$ lokal absolut stetig.
	Die Funktionen $T_e$, $l^P$ sowie $l_v$ sind dabei für alle Kanten $e\in E$, Pfade $P$ in G und Knoten $v\in V$ monoton wachsend, lokal absolut stetig und surjektiv.
\end{proposition}
\begin{proof}
	Nach dem Hauptsatz der Differential- und Integralrechnung für das Lebes\-gue-Inte\-gral (siehe \cite[Kap. VII, Satz 4.14]{Elstrodt2011}) ist $G: [a,b] \to \R, x\mapsto \int_a^x g(t) \diff t$ für eine Lebesgue-integrierbare Funktion $g: [a,b] \to \R$ absolut stetig.
	Insbesondere sind also $F_e^+$ sowie $F_e^-$ und damit auch $q_e$ und $T_e$ lokal absolut stetig.
	Als Komposition bzw. punktweises Minimum endlich vieler lokal absolut stetiger Funktionen sind auch $l^P$ und $l_v$ für alle Pfade $P$ und Knoten $v$ lokal absolut stetig.
	Nach Proposition~\ref{prop-feasible-flow}~\ref{prop-feasible-flow-T-mon-inc-cont} ist die Monotonie von $T_e$ bereits gegeben, welche auch die Monotonie von $l^P$ und $l_v$ impliziert.
	Wegen $f_e^+, f_e^-\in\mathfrak{F_0}$ gilt $q_e(\theta)=0$ für $\theta\leq 0$, wodurch auch $\lim_{\theta\to-\infty} T_e(\theta) = - \infty$ folgt.
	Mit $T_e(\theta)\geq \theta$ ergibt sich die Surjektivität von $T_e$.
	Daher sind auch $l^P$ und $l_v$ surjektiv.
\end{proof}

Die Monotonie lässt sich spieltheoretisch interpretieren:
Partikel, die zur Zeit $\theta$ in $s$ starten und sich auf einem kürzesten Pfad zu $t$ begeben, kommen zur Zeit $l_t(\theta)$ in $t$ an.
Partikel, die später starten, können also nicht früher in $t$ ankommen.
Wie im statischen Szenario von kürzesten Pfaden, gilt auch hier die Dreiecksungleichung: 

\begin{lemma}\label{lemma-dreicksungl}
	Für einen zulässigen dynamischen Fluss gilt 
	$T_{vw}(l_v(\theta)) \geq l_w(\theta)$ für alle Kanten $vw\in E$.
\end{lemma}
\begin{proof}
	Sei ein kürzester $s$-$v$-Pfad $P$ zum Zeitpunkt $\theta$ gegeben.
	Hängt man an $P$ die Kante $vw$ an, erhält man einen $s$-$w$-Pfad, der zur Eintrittszeit $\theta$ die Ankunftszeit $T_{vw}(l_v(\theta))$ liefert.
	Da $l_w(\theta)$ das Minimum über die Ankunftszeit aller $s$-$w$-Pfade ist, gilt die Behauptung.
\end{proof}

\begin{definition}
	Man bezeichne eine Kante $vw\in E$ als \emph{aktiv zum Zeitpunkt $\theta$}, falls sie auf einem zur Zeit $\theta$ kürzesten $s$-$w$-Pfad liegt; sonst nennt man sie \emph{inaktiv zum Zeitpunkt $\theta$}.
	Es bezeichne $\Theta_e$ die Menge aller Zeitpunkte, zu denen die Kante $e$ aktiv ist, und $G_\theta \coloneq (V, E_\theta)$ den durch die zur Zeit $\theta$ aktiven Kanten induzierten Teilgraphen.
\end{definition}

\begin{proposition}
	Für einen zulässigen Fluss ist eine Kante $vw$ genau dann aktiv zum Zeitpunkt $\theta$, falls $T_{vw}(l_v(\theta)) = l_w(\theta)$ gilt.
	Außerdem ist die Menge $\Theta_{vw}$ abgeschlossen.
\end{proposition}
\begin{proof}
	Ist $vw$ aktiv zum Zeitpunkt $\theta$, existiert ein zur Zeit $\theta$ kürzester $s$-$w$-Pfad $P$, der die Kante $vw$ benutzt.
	Da Zyklen nach Voraussetzung eine positive Gesamtverzögerung haben, ist $P$ ein $s$-$w$-Weg, dessen letzte Kante gerade $vw$ ist.
	Sei also $Q$ das Anfangsstück von $P$ bis zum Knoten $v$.
	Dann gilt aufgrund der Monotonie $
	T_{vw}(l_v(\theta)) \leq T_{vw}( l^Q(\theta) ) = l_w(\theta),
	$
	sodass mit Lemma~\ref{lemma-dreicksungl} sogar Gleichheit gilt.
	
	Gilt umgekehrt $T_{vw}(l_v(\theta)) = l_w(\theta)$ und sei $Q$ ein kürzester $s$-$v$-Pfad, so ist der Pfad $P$, der an $Q$ noch die Kante $vw$ anhängt, ein kürzester $s$-$w$-Pfad zur Zeit $\theta$, der die Kante $vw$ benutzt.
	
	Aufgrund der Stetigkeit von $T_{vw}$ und $l_v$ ist die Menge $\Theta_{vw}$ abgeschlossen.
\end{proof}


Man beachte, dass Teilpfade kürzester Pfade im statischen Sinne wieder kürzeste Pfade sind; im dynamischen Sinne gilt dies nicht unbedingt, jedoch aber in folgendem Teilgraph:

\begin{lemma}\label{lemma-shortest-path-using-active-edges}
	Für einen zulässigen Fluss ist $G_\theta$ zu jeder Zeit $\theta$ ein azyklischer Graph, in dem $s$ jeden Knoten $v\in V$ erreichen kann.
\end{lemma}
\begin{proof}
	Angenommen, es existiere ein Zyklus $C=(e_1, \dots, e_n)$ mit ausschließlich aktiven Kanten.
	Es ist $l^C(\theta) > \theta$, da für Zyklen eine positive Gesamtverzögerung vorausgesetzt ist.
	Setzt man $v:= \tail(e_1)$, so erzeugt $l_{v}(\theta) = l^C(l_{v}(\theta)) > l_{v}(\theta)$ einen Widerspruch aufgrund der Aktivität aller Kanten.
	
	Für jeden Knoten $w\neq s$ existiert mindestens eine eingehende aktive Kante -- zum Beispiel die letzte Kante eines kürzesten $s$-$w$-Pfades, welcher wiederum existiert, weil $w$ von $s$ aus in $G$ erreichbar ist.
	Daher ist $w$ von $s$ aus auch in $G_\theta$ erreichbar.
\end{proof}

\begin{proposition}\label{prop-arrival-times-vector}
	Für einen zulässigen dynamischen Fluss $f$ ist $(l_v(\theta))_{v\in V}$ die eindeutige Lösung des Gleichungssystems
	\[ \tilde{l}_w = \begin{cases}
	\theta, & \text{falls } w=s, \\
	\min\limits_{vw\in \delta^-(w)} T_{vw}(\tilde{l}_v), & \text{sonst}.
	\end{cases} \]
\end{proposition}
\begin{proof}
	Offenbar löst $(l_v(\theta))_{v\in V}$ dieses System, da jeder Knoten $w\neq s$ eine eingehende Kante hat, welche $T_{vw}(l_v) = l_w$ erfüllt.
	Für eine Lösung $(\tilde{l}_v)_{v\in V}$ des Gleichungssystems zeige man $l_w(\theta) = \tilde{l}_w$ für jeden Knoten $w\in V$.
	Dabei ist der Teilgraph $G'=(V, E')$ mit
	\[ E' \coloneq  \{ vw \in E \mid T_{vw}(\tilde{l}_v ) = \tilde{l}_w \} \]
	ein azyklischer Graph, in dem $s$ jeden Knoten $w\in V$ erreichen kann:
	Zyklen können wegen der positiven Gesamtverzögerung nicht entstehen und jeder Knoten $w\neq s$ hat mindestens eine eingehende Kante $vw$ mit $T_{vw}(\tilde{l}_v) = \tilde{l}_w$.
	Daher ist $\tilde{l}_w$ bereits durch einen $s$-$w$-Pfad $P$ in $G'$ festgelegt auf $l^P(\theta)\geq l_w(\theta)$.
	Für einen zur Zeit $\theta$ kürzesten $s$-$w$-Pfad $Q$ gilt außerdem $\tilde{l}_w \leq T^Q(\tilde{l}_s) = T^Q(\theta) = l_w(\theta)$.
\end{proof}

Um den Vektor $(l_v(\theta))_{v\in V}$ für alle $\theta\in\R$ gleichzeitig zu berechnen, kann der Bellman-Ford-Algorithmus auf den Distanzvektor-Funktionen $(l_v)_{v\in V}$ genutzt werden:
Dazu wird in jeder der $n-1$ Iterationen für jede Kante das punktweise Minimum $l_w \coloneq  \min\{ l_w, T_{vw}\circ l_v \}$ gebildet.
Sind Operationen auf Funktionen nicht möglich oder zu teuer, so kann $(l_v(\theta))_{v\in V}$ für ein spezielles $\theta\in\R$ mit dem Dijkstra-Algorithmus ermittelt werden, wobei man die Kosten einer Kante $vw$ erst bei Scanning von $v$ in der Form $q_{vw}(l_v(\theta)) + \tau_{vw}$ berechnet.

	\subsection{Kürzeste Wege}

\begin{frame}{Kürzeste Wege}
	\begin{definition}[Kürzeste Wege]
		Für einen Fluss $f$ bezeichne:
		\begin{itemize}[label=\color{darkblue}$\bullet$]
			\item\pause $l^P(\theta)\coloneq T_{e_k}\circ\dots\circ T_{e_1}(\theta)$ die Ankunftszeit am Endknoten eines Pfades $P=(e_1,\dots,e_k)$ zur Startzeit $\theta$ am Startknoten,
			\item\pause $\mathcal{P}_w$ die Menge aller $s$-$w$-Pfade,
			\item\pause $l_w(\theta) \coloneq \min_{P\in\mathcal{P}_w} l^P(\theta)$ die früheste Ankunftszeit bei $w$ zur Startzeit $\theta$.
		\end{itemize}
	\pause Ein Pfad $P\in \mathcal{P}_w$ heißt \emph{kürzester $s$-$w$-Pfad zur Zeit $\theta$}, falls $l^P(\theta)=l_w(\theta)$ gilt.
	\end{definition}
\end{frame}

\begin{frame}{Kürzeste Wege}
	\begin{lemma}[Dreiecksungleichung]
		Für alle Kanten $vw\in E$ gilt in einem zulässigen Fluss $T_{vw}(l_v(\theta))\geq l_w(\theta)$.
	\end{lemma}
	
	\pause\begin{definition}[Aktivität einer Kante]
		Eine Kante $vw\in E$ ist \emph{aktiv zum Zeitpunkt $\theta$}, falls $T_{vw}(l_v(\theta)) = l_w(\theta)$ gilt; sonst ist sie \emph{inaktiv zum Zeitpunkt $\theta$}.
		
		\pause
		Es sei $\Theta_e$ die Menge der Zeitpunkte, zu denen $e$ aktiv ist.
		Für $\theta\in\R$ sei $G_\theta\coloneq (V, E_\theta)$ der Teilgraph der zur Zeit $\theta$ aktiven Kanten.
	\end{definition}
\end{frame}
	
	\chapter{Dynamische Nash-Flüsse}\label{chapter-nash-flows}

Dieser Abschnitt dient dazu, Nash Gleichgewichte im Kontext dynamischer Flüsse einzuführen.
Dabei hilft die Anschauung, dass Partikel, die zur Zeit $\theta$ an der Quelle erscheinen, in einem Nash Gleichgewicht möglichst früh, also zum Zeitpunkt $l_t(\theta)$, an der Senke ankommen.

\section{Charakterisierung dynamischer Nash-Flüsse}


Für die formale Einführung dynamischer Nash-Flüsse benötigt man weitere Definitionen: 

\begin{definition}
	Für eine Kante $vw\in E$ bezeichne $x_{vw}^+(\theta):= F_{vw}^+(l_v(\theta))$ den Zufluss bis zur frühestmöglichen Ankunftszeit von Partikeln in $v$, die zur Zeit $\theta$ in $s$ starten.\\
	Dagegen bezeichne $x_{vw}^-(\theta):= F^-_{vw}(l_w(\theta))$ den Abfluss bis zur frühestmöglichen Ankunftszeit von Partikeln in $w$, die zur Zeit $\theta$ in $s$ starten.
	
	Für einen Knoten $v\in V$ sei $b_v(\theta):=\sum_{e\in\delta^+(v)} x_e^+(\theta) - \sum_{e\in\delta^-(v)} x_e^-(\theta)$ die Balance des Knoten $v$ zum Zeitpunkt $\theta$.
\end{definition}

\begin{remark}\label{remark-x^-leqx^+}
	In einem zulässigen Fluss gilt nach Proposition~\ref{prop-feasible-flow}~\ref{prop-feasible-flow-det-outflow} und mit der Monotonie von $F_{vw}^-$ bereits 
	\[
	x_{vw}^-(\theta) = F_{vw}^-(l_w(\theta)) \leq F_{vw}^-(T_{vw}(l_v(\theta)))=F_{vw}^+(l_v(\theta)) = x_{vw}^+(\theta).
	\]
\end{remark}

\begin{lemma}\label{lemma-balance-0}
	Für einen zulässigen dynamischen Fluss $f$ gilt $b_v(\theta)=0$ für alle Knoten $v\in V\setminus\{ s,t \}$ und alle $\theta\in\R$.
\end{lemma}
\begin{proof}
	Unter Benutzung der Voraussetzung~\ref{def-feasible-flow-no-flow-at-node} folgere man für $v\in V\setminus \{ s, t\}, \theta\in\R$:
	\[ \sum_{e\in\delta^-(v)} x_e^-(\theta) = \int_{0}^{l_v(\theta)} \sum_{e\in\delta^-(v)} f_e^-(t) \diff t = \int_{0}^{l_v(\theta)} \sum_{e\in\delta^+(v)} f_e^+(t) \diff t = \sum_{e\in\delta^+(v)}x_e^+(\theta). \]
\end{proof}

\begin{notation}
	 $M^c:= \R\setminus M$ bezeichne das Komplement von $M\subseteq\R$, $\overline{M}$ den Abschluss.
\end{notation}

\begin{definition}\label{def-flow-along-active-edges}
	Man sage, der Fluss $f$ \emph{fließe nur entlang aktiver Kanten}, falls $f_{vw}^+$ fast überall auf $l_v(\Theta_{vw}^c)$ verschwindet für alle Kanten $vw\in E$.
\end{definition}

\begin{remark}
	Diese Definition weicht von der Definition von Koch und Skutella ab und entspricht derjenigen aus~\cite[Definition 1]{Cominetti2015}:
	Nach \cite[Definition 2]{Koch2011} sagt man, $f$ \emph{sende Fluss nur entlang aktuell kürzester Pfade}, falls $f_{vw}^+\circ l_v$ fast überall auf $\Theta_{vw}^c$ verschwindet für alle Kanten $vw$.

	Entspricht $f$ dieser Definition, so auch Definition~\ref{def-flow-along-active-edges}: 
	Da $l_v$ nach Proposition~\ref{prop-abs-cont-sur} absolut stetig ist, bildet es nach~\cite[Aufgabe 4.9]{Elstrodt2011Abs} Nullmengen wieder auf Nullmengen ab, weshalb folgende Menge eine Nullmenge ist: \[ l_v(\{ \theta \in \Theta_{vw}^c \mid f_{vw}^+ (l_v(\theta)) > 0 \}) = \{ \xi \in l_v(\Theta_{vw}^c) \mid f_{vw}^+ (\xi) > 0 \}. \]
	 
	Koch und Skutella zeigen im Beweis von~\cite[Lemma 1]{Koch2011} die entsprechende Äquivalenz von Lemma~\ref{lemma-only-active-edges} (i) und (iii) -- jedoch in (i) unter Verwendung ihrer Definition -- und
	verwenden bei der Implikation (iii)$\Rightarrow$(i) das Argument, dass für jede Kante $vw\in E$ und alle $\theta\in \Theta_{vw}^c$ eine Umgebung $U$ von $\theta$ existiert, sodass $f_{vw}^+$ fast überall in $l_v(U)$ verschwindet.
	Dies reicht aber nicht aus, um zu zeigen, dass $f_{vw}^+(l_v(\theta))=0$ für fast alle $\theta\in\Theta_{vw}^c$ gilt:
	So kann $f_{vw}^+(l_v(\theta))$ für ein $\theta\in\Theta_{vw}^c$ positiv sein und $l_v$ konstant in einer Umgebung um $\theta$.
	Dann ist $f_{vw}^+ \circ l_v$ in einer Umgebung um $\theta$ positiv, was im Widerspruch zur Forderung ist.
	
	Dies wurde in~\cite[Example 2]{Cominetti2015} ausgenutzt, um einen Beispielfluss anzugeben, der beweist, dass die Forderung von Koch und Skutella sogar echt stärker ist.
\end{remark}

Für eine äquivalente Umschreibung dieser Definition, benötigen wir folgendes Lemma der Maßtheorie:

\begin{lemma}\label{lemma-vanishes-intervals}
	Seien $g: \R \to \R_{\geq 0}$ eine lokal Lebesgue-integrierbare Funktion und $((a_i, b_i))_{i\in I}$ eine Familie offener Intervalle.
	Dann verschwindet $g$ fast überall auf $\Theta:=\bigcup_{i\in I} (a_i, b_i)$ genau dann, wenn es für alle $i\in I$ fast überall auf $(a_i, b_i)$ verschwindet.
\end{lemma}
\begin{proof}
	Verschwindet $g$ fast überall auf $\Theta$, so erst recht auf jedem Intervall $(a_i, b_i)$.
	Für die andere Richtung definiert die Funktion $\mu(A):= \int_A g \diff \lambda$ ein Maß auf den Borelmengen~$\mathfrak{B}$.
	Da jede offene Menge $O\subseteq\R$ $\sigma$-kompakt ist, also eine Darstellung als abzählbare Vereinigung kompakter Mengen -- hier $O=\bigcup_{n\in\N}(\{ x \in\R \mid d(x, O^c) \geq 1/n \} \cap [-n, n] )$ -- besitzt, ist jede offene Menge nach~\cite[1.2 Folgerungen (e)]{Elstrodt2011Top} innen regulär.
	Das heißt, es gilt
	\[ \mu(O)=\sup\{ \mu(K) \mid K\subseteq O \text{ kompakt} \} \]
	für offene Mengen $O\subseteq\R$.
	Für ein kompaktes $K\subseteq \Theta$ existiert eine endliche Teil\-über\-deckung $\bigcup_{i=1}^n (a_i, b_i) \supseteq K$, für die $\mu(K) \leq \sum_{i=1}^{n} \mu((a_i, b_i)) = \sum_{k=1}^{n} \int_{a_i}^{b_i} g(t) \diff t = 0$ gilt.
	Also ist auch $\mu(\Theta)=0$.
\end{proof}

\begin{lemma}\label{lemma-only-active-edges}
	Für einen zulässigen Fluss $f$ sind folgende Aussagen äquivalent:
	\begin{enumerate}[label=(\roman*)]
		\item Der Fluss $f$ fließt nur entlang aktiver Kanten.
		\item Für jede Kante $vw\in E$ und für fast alle $\xi\in\R$ mit	$f_{vw}^+(\xi)>0$ gilt $\xi \in l_v(\Theta_{vw})$.
		\item Für jede Kante $e\in E$ und für alle $\theta\in\R$ gilt $x_e^+(\theta) = x_e^-(\theta)$.
	\end{enumerate}
\end{lemma}
\begin{proof}
	$(i) \Leftrightarrow (ii)$: Bedingung~(ii) gilt genau dann, wenn $f_{vw}^+$ fast überall auf $l_v(\Theta_{vw})^c$ verschwindet.
	Daher genügt es, zu zeigen, dass sich $l_v(\Theta_{vw})^c$ und $l_v(\Theta_{vw}^c)$ nur um eine Nullmenge voneinander unterscheiden.
	Mit der Surjektivität von $l_v$ gilt $l_v(\Theta_{vw})^c\subseteq l_v(\Theta_{vw}^c)$.
	
	Des Weiteren ist $S:=l_v(\Theta_{vw}^c)\setminus l_v(\Theta_{vw})^c = l_v(\Theta_{vw}^c)\cap l_v(\Theta_{vw})$ eine Teilmenge von $l_v(\Q)$:
	Für ein $\xi\in S$ gibt es $\theta\in\Theta_{vw}^c$ und $\theta'\in\Theta_{vw}$ mit $l_v(\theta)=\xi=l_v(\theta')$.
	Da $\theta\neq\theta'$ ist, existiert ein $\theta_q\in\Q\cap(\theta,\theta')$.
	Wegen der Monotonie von $l_v$ gilt $l_v(\theta_q)=\xi$, womit $\xi\in l_v(\Q)$ folgt.
	Also unterscheiden sich die beiden Mengen nur um eine abzählbare Menge.
	
	$(i)\Leftrightarrow (iii)$: Sei eine Kante $vw\in E$ gegeben.
	Für ein $\theta\in\R$ bezeichne $\omega_\theta\leq \theta$ den spätesten Startzeitpunkt, sodass man unter Benutzung von $vw$ zum Zeitpunkt $l_w(\theta)$ zu $w$ gelangt:
	\[ \omega_\theta:=\max\{ \omega\leq\theta \mid l_w(\theta) = T_{vw}(l_v(\omega)) \}. \]
	
	Es gilt $\Theta_{vw}^c = \bigcup_{\theta\in\R} (\omega_\theta, \theta)$:
	Für $\theta\in\Theta_{vw}^c$ gilt $T_{vw}(l_v(\theta)) > l_w(\theta)$.
	Aufgrund der Stetigkeit von $T_{vw}\circ l_v$ und von $l_w$ existiert ein $\varepsilon>0$, sodass $T_{vw}(l_v(\theta')) > l_w(\theta+\varepsilon)$ für $\theta'\in[\theta,\theta+\varepsilon]$ gilt.
	Also ist $\theta\in(\omega_{\theta+\varepsilon}, \theta+\varepsilon)$.
	Ist umgekehrt $\theta'\in (\omega_\theta,\theta)$, so ist aufgrund der Monotonie $T_{vw}(l_v(\theta'))\geq T_{vw}(l_v(\omega_\theta)) = l_w(\theta)\geq l_w(\theta')$.
	Die erste Ungleichung kann nicht mit Gleichheit erfüllt sein, da $\omega_\theta$ maximal mit der Eigenschaft $T_{vw}(l_v(\omega)) = l_w(\theta)$ ist, wodurch $\theta'\in\Theta_{vw}^c$ folgt.
	
	Mit $l_v(\Theta_{vw}^c) = \bigcup_{\theta\in\R}(l_v(\omega_\theta),l_v(\theta))$ verschwindet $f_{vw}^+$ nach Lemma~\ref{lemma-vanishes-intervals} genau dann fast überall auf $l_v(\Theta_{vw}^c)$, wenn es für alle $\theta\in\R$ fast überall auf $(l_v(\omega_\theta),l_v(\theta))$ verschwindet.
	Dies ist nach Proposition~\ref{prop-feasible-flow}~\ref{prop-feasible-flow-det-outflow} wiederum äquivalent zu
	$F_{vw}^+(l_v(\theta))-F_{vw}^-(l_w(\theta))=0$ für alle $\theta\in\R$.
\end{proof}

\begin{definition}
	Man sage, ein zulässiger dynamischer Fluss $f$ \emph{fließe ohne Über\-holungs\-möglichkeiten}, falls $b_s(\theta) = -b_t(\theta)$ für alle $\theta\in\R$.
\end{definition}

Dabei betrachte man folgende Intuition:
 Partikel, die zur Zeit $\theta\in\R$ bei $s$ starten und sich auf einem kürzesten Weg zu $t$ bewegen  -- also zur Zeit $l_t(\theta)$ in $t$ ankommen --, überholen andere Partikel, falls $b_s(\theta) > -b_t(\theta)$.
Falls jedoch $b_s(\theta) < - b_t(\theta)$ gilt, wurde das Partikel bereits von anderen überholt.
Ein Nash-Gleichgewicht sollte diese Eigenschaft daher erfüllen.

\begin{definition}
	Seien ein statischer Fluss $f \in \R^E$ in einem Graphen $G=(V,E)$ mit Kapazitäten $u\in \R_+^E$ und ein Balancevektor $b\in\R^V$ mit $\sum_{v\in V} b_v = 0$ gegeben.
	Der Fluss $f$ heißt \emph{$b$-Fluss}, falls er Flusserhaltung bzgl. $b$ gewährt, d.h. falls alle $v\in V$ die Bedingung $\sum_{e\in\delta^+(v)}f_e - \sum_{e\in\delta^-(v)}f_e = b_v$ erfüllen.
\end{definition}

\newcommand{\newv}{\mathbf{v}}
\begin{lemma}\label{lemma-b-graph}
	Seien ein dynamischer Fluss $f$ in einem Graphen $G=(V,E)$ und ein Zeitpunkt $\theta\in\R$ gegeben.
	Der Graph $H$ entstehe aus $G$, indem man jede Kante $vw\in E$ aus $G$ durch einen neuen Knoten $\newv_{vw}$ und zwei Kanten $v\newv_{vw}$ und $\newv_{vw}w$ ersetze.
	Der statische Fluss $g$ auf $H$ sei definiert durch
	\[ g_{v\newv_{vw}} := x_{vw}^+(\theta) \text{ und } g_{\newv_{vw}w} := x_{vw}^-(\theta) \text{ für alle $vw\in E$} \]
	und die Balance $b$ auf $H$ sei gegeben durch $b_v:= b_v(\theta)$ für $v\in V$ und $b_{\newv_e}:= x_e^-(\theta) - x_e^+(\theta)$ für $e\in E$.
	Dann gelten die folgenden Aussagen:
	
	\begin{enumerate}[label=(\roman*)]
		\item Der Fluss $g$ ist ein statischer $b$-Fluss.
		\item\label{lemma-b-graph-imp} Ist $f$ zulässig, so gilt $\forall e\in E : x_e^+(\theta) = x_e^-(\theta)\iff b_s(\theta) + b_t(\theta) = 0$.
	\end{enumerate}
\end{lemma} 
\begin{proof}
	$(i)$: Um zu zeigen, dass die Summe über die Balanceeinträge verschwindet, erkenne man, dass der Anteil einer Kante $e\in E$ in $\sum_{v\in V} b_v$ gerade $x_e^+(\theta) - x_e^-(\theta)$ ist.
	Damit gilt:
		\[ \sum_{v\in V}b_v + \sum_{e\in E} b_{\newv_e} = \sum_{e\in E}  (x_e^+(\theta) - x_e^-(\theta) + x_e^-(\theta) - x_e^+(\theta)) = 0. \]
		Es bleibt zu zeigen, dass $g$ bezüglich $b$ Flusserhaltung gewährt.
		Für die Knoten der Form $\newv_{vw}$ gilt dies, da $g_{\newv_{vw}w} - g_{v\newv_{vw}} = x_{vw}^-(\theta) - x_{vw}^+(\theta) = b_{\newv_{vw}}$.
		Für $v\in V$ gilt nach Konstruktion
		\[ b_v =
		\sum_{e\in\delta^+_G(v)} x_{e}^+(\theta) - \sum_{e\in\delta^-_G(v)} x_{e}^-(\theta) =
	\sum_{e\in\delta_H^+(v)} g_e - \sum_{e\in\delta^-_H(v)}g_e
		. \]
	
	$(ii)$: Tatsächlich benötigt man aus $(i)$ nur die Eigenschaft, dass die Summe über die Einträge des Balancevektors verschwindet.
	Mit Lemma~\ref{lemma-balance-0} gilt wegen der Zulässigkeit von $f$ sogar $b_s(\theta)+b_t(\theta) + \sum_{e\in E} b_{\newv_e} = 0$.
	
	Angenommen, es gelte $x_e^+(\theta) = x_e^-(\theta)$ für alle $e\in E$.
	Dann sind auch alle $b_{\newv_e} = 0$ und es gilt $b_s(\theta) + b_t(\theta) = 0$.
	Setzt man $b_s(\theta) + b_t(\theta) = 0$ voraus, so ist $\sum_{e\in E} b_{\newv_e} = 0$ und, da $f$ zulässig ist, gilt $x_e^-(\theta)\leq x_e^+(\theta)$ nach Bemerkung~\ref{remark-x^-leqx^+}.
	Daher gilt $b_{\newv_e}\leq 0$ für alle $e\in E$, weshalb bereits alle $b_{\newv_e} = 0$ sein müssen.
\end{proof}

Die Ergebnisse aus Lemma~\ref{lemma-only-active-edges} und Lemma~\ref{lemma-b-graph} werden im folgenden Theorem gesammelt, welches Nash-Gleichgewichte in dynamischen Flüssen charakterisiert:

\begin{theorem}[Charakterisierung dynamischer Nash-Flüsse]\label{thm-equivalencies-nash-flow}
	Ist $f$ ein zu\-läs\-siger dynamischer Fluss, so sind die folgenden Aussagen äquivalent:
	\begin{enumerate}[label=(\roman*)]
		\item Der Fluss $f$ fließt nur entlang aktiver Kanten.
		\item Für alle Kanten $e\in E$ und zu jeder Zeit $\theta\in\R$ gilt $x_e^+(\theta) = x_e^-(\theta)$.
		\item Der Fluss $f$ fließt ohne Überholungsmöglichkeiten.
	\end{enumerate}
	Gilt eine dieser Aussagen, so nennt man $f$ einen \emph{dynamischen Nash-Fluss}.
\end{theorem}

\section{Eigenschaften dynamischer Nash-Flüsse}

In diesem Abschnitt werden einige Ergebnisse über Nash-Flüsse gesammelt, die in Abschnitt~\ref{sec-nash-flow-extension} benötigt werden.

\begin{remark}\label{remark-s-t-flow}
	In einem Nash-Fluss ist der statische Fluss $x(\theta)$ mit $x_e(\theta):=x_e^+(\theta)=x_e^-(\theta)$ nach Lemma~\ref{lemma-balance-0} für alle $\theta\in\R$  ein statischer $s$-$t$-Fluss.
	Wegen der Monotonie von $x_e$ ist auch $x(\theta_2) - x(\theta_1)$ für $\theta_1 \leq \theta_2$ ein statischer $s$-$t$-Fluss, genauso wie $x'(\theta)$, falls $x_e$ für alle $e\in E$ differenzierbar in $\theta$ ist, da Differenzieren die Flusserhaltung erhält und $x_e$ monoton wachsend ist für alle $e\in E$.
\end{remark}

\begin{lemma}\label{lemma-x-locally-constant}
In einem dynamischen Nash-Fluss ist $x_e$ eingeschränkt auf $\overline{\Theta_e^c}$, also dem Abschluss der Menge der inaktiven Zeitpunkte von $e$, für jede Kante $e\in E$ lokal konstant.
\end{lemma}
\begin{proof}
Da $\Theta_{vw}^c$ eine in $\R$ offene Menge ist, hat sie eine Darstellung als abzählbare Vereinigung paarweise disjunkter offener Intervalle.
Innerhalb eines solchen Intervalls $(\theta_1, \theta_2)$ gilt $x_{vw}(\theta_2) - x_{vw}(\theta_1) = \int_{l_v(\theta_1)}^{l_v(\theta_2)} f_{vw}^+(t) \diff t = 0$, da $f$ nur entlang aktiver Kanten fließt.
Der Rest folgt mit der Monotonie und Stetigkeit von $x_{vw}$.
\end{proof}

\begin{lemma}\label{lemma-nash-flow-waiting-queue-implies-active-edge}
	Seien ein dynamischer Nash-Fluss $f$, eine Kante $vw\in E$ und ein Zeitpunkt $\theta\in\R$ gegeben.
	Gilt eine der folgenden Aussagen, so ist $vw$ zum Zeitpunkt $\theta$ aktiv:
	\begin{enumerate}[label=(\roman*)]
		\item Die Ableitung $x_{vw}'(\theta)$ existiert und es gilt $x_{vw}'(\theta)> 0$.
		\item Die Wartezeit $q_{vw}$ an der Kante $vw$ ist zur Zeit $l_v(\theta)$ positiv.
	\end{enumerate}
	Insbesondere verschwindet die Wartezeit $q_{vw}(l_v(\theta))$ für alle $\theta\in\overline{\Theta_{vw}^c}$.
\end{lemma}
\begin{proof}
	Zu Aussage (i): Angenommen, $vw$ wäre zum Zeitpunkt $\theta$ nicht aktiv, so würde wegen der Offenheit von $\Theta_{vw}^c$ und Lemma~\ref{lemma-x-locally-constant} die Ableitung $x_{vw}'(\theta)$ verschwinden.
	
	Für Aussage (ii) zeige man $T_{vw}(l_v(\theta)) \leq l_w(\theta)$.
	Sei $\theta_1$ der früheste Zeitpunkt mit $x_{vw}^+(\theta_1)= x_{vw}^+(\theta)$.
	Dieser existiert, da $l_v$ nach Proposition~\ref{prop-abs-cont-sur} surjektiv ist.
	Dann ist $\theta_1\in \Theta_{vw}$ nach Lemma~\ref{lemma-x-locally-constant}.
	Außerdem ist $\theta_1 \leq \theta$ wegen der Monotonie von $F_{vw}^+ \circ l_v$.
	Nach Aussage (i) gilt nun $T_{vw}(l_v(\theta_1)) = l_w(\theta_1)$.
	Nach Proposition~\ref{prop-feasible-flow}~\ref{prop-feasible-flow-queue-delay} ist $T_{vw}(l_v(\theta_1)) = T_{vw}(l_v(\theta))$ und mit der Monotonie von $l_w$ folgt $T_{vw}(l_v(\theta))\leq l_w(\theta)$.
\end{proof}

\begin{proposition}\label{prop-nash-flow-s-t-path-decomposable}
	Für einen dynamischen Nash-Fluss $f$ und zwei Zeitpunkte $\theta_1 \leq \theta_2$ ist der statische $s$-$t$-Fluss $x(\theta_2) - x(\theta_1)$ eine Komposition von $s$-$t$-Wegen.
\end{proposition}
\begin{proof}
	Sei $\theta$ das Infimum aller Zeitpunkte $\xi\geq\theta_1$, zu denen $x(\xi) - x(\theta_1)$ nicht in $s$-$t$-Wege zerlegbar ist.
	Man nehme $\theta \leq \theta_2$ an.
	Da inaktive Kanten zum Zeitpunkt $\theta$ bereits kurz vor $\theta$ und noch kurz nach $\theta$ inaktiv sind, existiert ein Intervall $[\theta - \varepsilon, \theta + \varepsilon]$, in der keine inaktive Kante aktiv wird.
	Außerdem existiert für $\xi_0 := \max \{ \theta_1, \theta - \varepsilon \}$ eine $s$-$t$-Wegezerlegung von $x(\xi_0) - x(\theta_1)$.
	
	Für einen Pfad $P$ und einen statischen Fluss $g$ sei $g^P := \min_{e\in P} g_e$ der Fluss, der auf dem Pfad $P$ fließt.
	Für einen Zyklus $C$ ist $(x(\xi) - x(\xi_0))^C = 0$ für $\xi\in [\xi_0, \theta+\varepsilon]$, da aufgrund der Azyklizität von $G_{\xi_0}$ eine Kante $e\in C$ des Zyklus existiert, die zur Zeit $\xi_0$ und damit in ganz $[\xi_0, \theta+\varepsilon]$ inaktiv ist, wodurch $x_e(\xi_0) = x_e(\xi)$ nach Lemma~\ref{lemma-x-locally-constant} folgt.
	
	Also hat der $s$-$t$-Fluss $x(\xi) - x(\xi_0)$ für $\xi\in [\xi_0, \theta + \varepsilon]$ keinen Zyklus mit positivem Fluss und besitzt daher eine $s$-$t$-Wegezerlegung.
	Addiert man diese zur $s$-$t$-Wegezerlegung von $x(\xi_0) - x(\theta_1)$, so erhält man eine $s$-$t$-Wegezerlegung von $x(\xi) - x(\theta_1)$, was für $\xi > \theta$ einen Widerspruch zur Definition von $\theta$ darstellt.
\end{proof}

\begin{corollary}
	Für einen dynamischen Nash-Fluss $f$ ist der statische $s$-$t$-Fluss $x(\theta)$ zu jeder Zeit $\theta$ eine Komposition von $s$-$t$-Wegen.
\end{corollary}
\begin{proof}
	Nach Proposition~\ref{prop-abs-cont-sur} existiert ein Zeitpunkt $\xi_0$ mit $l_v(\xi_0) \leq 0$ für alle Knoten $v\in V$.
	Für $\theta \leq \xi_0$ ist $x(\theta)$ der Nullfluss und offenbar in $s$-$t$-Wege zerlegbar, da die Funktionen $f_e^+$ und $f_e^-$ links der $y$-Achse verschwinden.
	Sonst ist $x(\theta)=  x(\theta) - x(\xi_0)$ nach Proposition~\ref{prop-nash-flow-s-t-path-decomposable} in $s$-$t$-Wege zerlegbar.
\end{proof}
	
	\section{Erweiterung dynamischer Nash-Flüsse}

\begin{frame}\begin{definition}[Schmaler Fluss mit Zurücksetzen]\label{def-thin-flow}
		Seien ein statischer $s$-$t$-Fluss  $x'$ von Wert $F$ in einem Netzwerk mit Versorgungsrate $d$ sowie $E_1\subseteq E$ gegeben. \\
		Der Fluss $x'$ ist ein \emph{schmaler Fluss mit Zurücksetzen auf $E_1$}, falls $l'\in\R^V$ existiert mit:
		\begin{enumerate}[label=(T\arabic*)]
			\item\label{def-thin-flow-source} $l_s' = F/d$,
			\item\label{def-thin-flow-x-zero} $l_w' \leq l_v'$, \tabto{5cm} für $vw\in E \setminus E_1$ mit $x'_{vw}=0$,
			\item\label{def-thin-flow-x-positive} $l_w' = \max(l_v', x'_{vw} / u_{vw} )$,  \tabto{5cm} für $vw\in E\setminus E_1$ mit $x'_{vw} > 0$,
			\item\label{def-thin-flow-resetting-edge} $l_w' = x'_{vw} / u_{vw}$,  \tabto{5cm} für $vw\in E_1$,
			\item\label{def-thin-flow-no-resetting-edge} $l_w' \geq \min_{vw\in \delta^-(w)} l_v'$, \tabto{5cm} falls $\delta^-(w)\cap E_1 = \emptyset$.
		\end{enumerate}
	\end{definition}
\end{frame}

\begin{frame}
	\begin{definition}[Dynamischer Fluss mit Zeithorizont]
		Ein \emph{dynamischer Fluss $f$ mit Zeithorizont $T\geq0$} ist ein Fluss, für dessen Zufluss $d(\theta)= 0$ für $\theta\geq T$ gilt.
	\end{definition}

	\begin{definition}[$\alpha$-Erweiterung]
		Seien ein dynamischer Nash-Fluss $f$ mit Horizont $T$ und ein schmaler Fluss mit Zurücksetzen auf $E_1 := \{ vw\in E \mid q_{vw}(l_v(\theta)) > 0 \}$ im Graphen $G_T$ und ein $\alpha > 0$ gegeben.
		
		Ergänzt man $f$, sodass für zur Zeit $T$ aktive Kanten $vw\in E_T$
		$\tilde{f}_{vw}^+(\theta):= x_{vw}'/l_v'$ für $\theta\in [l_v(T), l_v(T)+\alpha l_v')$ und \\
		$\tilde{f}_{vw}^-(\theta):=x_{vw}'/l_w'$ für $\theta\in [l_w(T), l_w(T)+\alpha l_w')$
		gelten, erhält man eine \emph{$\alpha$-Erweiterung $\tilde{f}$ von $f$}.
	\end{definition}
\end{frame}

\begin{frame}
	\begin{theorem}[Erweiterung eines Nash-Flusses]
		Jede $\alpha$-Erweiterung $\tilde{f}$ eines dynamischen Nash-Flusses $f$ mit Zeithorizont~$T$ und
		\begin{align*}
			l_w(T) - l_v(T) + \alpha(l_w' - l_v') \geq \tau_{vw} &\text{ ~~~~falls $q_{vw}(l_v(T)) > 0$,}\\
			l_w(T) - l_v(T) + \alpha(l_w' - l_v') \leq \tau_{vw} &\text{ ~~~~falls $T\in\Theta_{vw}^c $}
		\end{align*}
		ist ein dynamischer Nash-Fluss.
	\end{theorem}
\end{frame}

	

	\begin{noheadline}
		\begin{frame}<presentation:0>[noframenumbering]
			\cite{Koch2011}
			\cite{Cominetti2011}
			\cite{Cominetti2015}
			\cite{Elstrodt2011Abs}
			\cite{Elstrodt2011Top}
		\end{frame}
	
		\begin{frame}[allowframebreaks]{Literatur}
			\scriptsize
			\bibliographystyle{alphadin}
			\bibliography{literature}
		\end{frame}
	\end{noheadline}

\end{document}
