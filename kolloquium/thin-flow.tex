\subsection{Berechnung schmaler Flüsse ohne Zurücksetzen}

\begin{frame}<presentation:0>[noframenumbering]
	\begin{definition}[Auslastungsminimaler $b$-Fluss]
		Für einen $b$-Fluss $x'$ ist $x'_e/u_e$ die \emph{Auslastung von $e\in E$}.
		Die \emph{Auslastung von $x'$} ist gegeben durch $c(x')\coloneq \max_{e\in E} x'_e / u_e$.	
	\end{definition}
	\pause\begin{definition}[Dünnster Schnitt]
	Für einen Schnitt $X\subseteq V$ mit $\delta^+(X)\neq \emptyset$ bezeichnet $b(X) / u(\delta^+(X))$ die Auslastung von $X$.
	Ein Schnitt maximaler Auslastung heißt \emph{dünnster Schnitt}.
	\end{definition}
	\pause\begin{theorem}[Starke Dualität]
		Die minimale Auslastung eines $b$-Flusses ist die maximale eines dünnsten Schnittes:
		\[
		\min_{\text{$x'$ $b$-Fluss}}~ \max_{e\in E} \frac{x'_e}{u_e} = \max_{\substack{X\subseteq V\\ \delta^+(X) \neq \emptyset}} \frac{b(X)}{u(\delta^+(X))}
		\]
	\end{theorem}
\end{frame}


\newcommand{\newnode}[4]{\node[lul] (#1) at #2 {#3 \\[0.5em] {\color{blue} #4}};}
\newcommand{\transnode}[4]{\node[lul, opacity=0.3] (#1) at #2 {#3 \\[0.5em] {\color{blue} #4}};}
\newcommand{\edgeu}[1]{{\color{red} #1}}
\newcommand{\edgefu}[2]{{\color{darkgreen} #1}/{\color{red} #2}}
\begin{frame}[t]{Berechnung schmaler Flüsse ohne Zurücksetzen (1)}
\vspace{0.35em}
\begin{figure}
	\small
	\centering
	\makebox[0pt]{
		\begin{tikzpicture}[lul/.style={draw,
			ellipse,
			align=center,
			inner sep=0pt,
			outer sep=4pt,
			text width=7mm,
			minimum height=1.5cm
		},
		scale=0.8]
		
		\newnode{0}{(0,2)}{$a$}{$32$}
		\newnode{1}{(3,2)}{$b$}{$0$}
		\newnode{2}{(5,0)}{$c$}{$-2$}
		\newnode{3}{(7,2)}{$d$}{$-12$}
		\newnode{4}{(10,3)}{$e$}{$-2$}
		\newnode{5}{(10,1)}{$f$}{$0$}
		\newnode{6}{(13,2)}{$g$}{$-16$}
		
		\begin{scope}[-Latex]
		\path [-Latex] (0) edge node[above] {\edgeu{$4$}} (1);
		\path [-Latex] (0) edge[bend right] node[above right] {\edgeu{$4$}} (2);
		\path [-Latex] (0) edge[bend left] node[above] {\edgeu{$4$}} (4);
		\path [-Latex] (1) edge node[above right] {\edgeu{$2$}} (2);
		\path [-Latex] (1) edge node[above] {\edgeu{$2$}} (3);
		\path [-Latex] (2) edge node[below right] {\edgeu{$2$}} (3);
		\path [-Latex] (3) edge node[above] {\edgeu{$1$}} (4);
		\path [-Latex] (3) edge node[above] {\edgeu{$2$}} (5);
		\path [-Latex] (4) edge node[above] {\edgeu{$1$}} (6);
		\path [-Latex] (5) edge node[above] {\edgeu{$1$}} (6);
		\end{scope}
		\end{tikzpicture}
	}
\end{figure}
\vfill
\vspace{0.13em}
{\color{red}Kapazität $u$}\\
{\color{blue}Balance $b$}
\end{frame}

\begin{frame}[t]{Berechnung schmaler Flüsse ohne Zurücksetzen (2)}
\begin{figure}
\small
\centering
\makebox[0pt]{
	
\begin{tikzpicture}[lul/.style={draw,
	ellipse,
	align=center,
	inner sep=0pt,
	outer sep=4pt,
	text width=7mm,
	minimum height=1.5cm
},
scale=0.8]

	\newnode{0}{(0,2)}{$a$}{$32$}
	\newnode{1}{(3,2)}{$b$}{$0$}
	\newnode{2}{(5,0)}{$c$}{$-2$}
	\newnode{3}{(7,2)}{$d$}{$-12$}
	\newnode{4}{(10,3)}{$e$}{$-2$}
	\newnode{5}{(10,1)}{$f$}{$0$}
	\newnode{6}{(13,2)}{$g$}{$-16$}
	
	\begin{scope}[-Latex]
	\path [-Latex] (0) edge node[above] {\edgefu{$4$}{$4$}} (1);
	\path [-Latex] (0) edge[bend right] node[above right] {\edgefu{$18$}{$4$}} (2);
	\path [-Latex] (0) edge[bend left] node[above] {\edgefu{$10$}{$4$}} (4);
	\path [-Latex] (1) edge node[above right] {\edgefu{$0$}{$2$}} (2);
	\path [-Latex] (1) edge[segment amplitude=.4mm,segment length=2mm,line after snake=2mm] node[above] {\edgefu{$4$}{$2$}} (3);
	\path [-Latex] (2) edge[snake=snake,segment amplitude=.4mm,segment length=2mm,line after snake=2mm] node[below right] {\edgefu{$16$}{$2$}} (3);
	\path [-Latex] (3) edge node[above] {\edgefu{$0$}{$1$}} (4);
	\path [-Latex] (3) edge node[above] {\edgefu{$8$}{$2$}} (5);
	\path [-Latex] (4) edge[snake=snake,segment amplitude=.4mm,segment length=2mm,line after snake=2mm] node[above] {\edgefu{$8$}{$1$}} (6);
	\path [-Latex] (5) edge[snake=snake,segment amplitude=.4mm,segment length=2mm,line after snake=2mm] node[above] {\edgefu{$8$}{$1$}} (6);
	\end{scope}
	
	\begin{scope}
	\draw[dashed, color=blue] (11.5,-1) -- (11.5, 4.5);
	\node[color=blue] at (11,-0.5) {$X^1$};
	\end{scope}
	\end{tikzpicture}
}
\end{figure}
\vfill
Berechne auslastungsminimalen Fluss $x'$ und dünnsten Schnitt.\\
{\color{red}Kapazität $u$}\\
{\color{blue}Balance $b$}\\
{\color{darkgreen}Fluss $x'$}
\end{frame}


\begin{frame}[t]{Berechnung schmaler Flüsse ohne Zurücksetzen (3)}
\begin{figure}
	\small
	\centering
	\makebox[0pt]{
		
		\begin{tikzpicture}[lul/.style={draw,
			ellipse,
			align=center,
			inner sep=0pt,
			outer sep=4pt,
			text width=7mm,
			minimum height=1.5cm
		},
		scale=0.8]
		
		\newnode{0}{(0,2)}{$a$}{$32$}
		\newnode{1}{(3,2)}{$b$}{$0$}
		\newnode{2}{(5,0)}{$c$}{$-2$}
		\newnode{3}{(7,2)}{$d$}{$-12$}
		\newnode{4}{(10,3)}{$e$}{$-10$}
		\newnode{5}{(10,1)}{$f$}{$-8$}
		\transnode{6}{(13,2)}{$g$}{$-16$}
		
		\begin{scope}[-Latex]
		\path [-Latex] (0) edge node[above] {\edgefu{$4$}{$4$}} (1);
		\path [-Latex] (0) edge[bend right] node[above right] {\edgefu{$18$}{$4$}} (2);
		\path [-Latex] (0) edge[bend left] node[above] {\edgefu{$10$}{$4$}} (4);
		\path [-Latex] (1) edge node[above right] {\edgefu{$0$}{$2$}} (2);
		\path [-Latex] (1) edge[segment amplitude=.4mm,segment length=2mm,line after snake=2mm] node[above] {\edgefu{$4$}{$2$}} (3);
		\path [-Latex] (2) edge[snake=snake,segment amplitude=.4mm,segment length=2mm,line after snake=2mm] node[below right] {\edgefu{$16$}{$2$}} (3);
		\path [-Latex] (3) edge node[above] {\edgefu{$0$}{$1$}} (4);
		\path [-Latex] (3) edge node[above] {\edgefu{$8$}{$2$}} (5);
		\path [-Latex, opacity=0.3] (4) edge[snake=snake,segment amplitude=.4mm,segment length=2mm,line after snake=2mm] node[above] {\edgefu{$8$}{$1$}} (6);
		\path [-Latex, opacity=0.3] (5) edge[snake=snake,segment amplitude=.4mm,segment length=2mm,line after snake=2mm] node[above] {\edgefu{$8$}{$1$}} (6);
		\end{scope}
		
		\begin{scope}
		\draw[dashed, color=blue, opacity=0.3] (11.5,-1) -- (11.5, 4.5);
		\node[color=blue, opacity=0.3] at (11,-0.5) {$X^1$};
		\end{scope}
		\end{tikzpicture}
	}
\end{figure}
\vfill
Berechne induzierte kleinere Probleminstanz.\\
{\color{red}Kapazität $u$}\\
{\color{blue}Balance $b$}\\
{\color{darkgreen}Fluss $x'$}
\end{frame}



\begin{frame}[t]{Berechnung schmaler Flüsse ohne Zurücksetzen (4)}
\begin{figure}
	\small
	\centering
	\makebox[0pt]{
		
		\begin{tikzpicture}[lul/.style={draw,
			ellipse,
			align=center,
			inner sep=0pt,
			outer sep=4pt,
			text width=7mm,
			minimum height=1.5cm
		},
		scale=0.8]
		
		\newnode{0}{(0,2)}{$a$}{$32$}
		\newnode{1}{(3,2)}{$b$}{$0$}
		\newnode{2}{(5,0)}{$c$}{$-2$}
		\newnode{3}{(7,2)}{$d$}{$-12$}
		\newnode{4}{(10,3)}{$e$}{$-10$}
		\newnode{5}{(10,1)}{$f$}{$-8$}
		\transnode{6}{(13,2)}{$g$}{$-16$}
		
		\begin{scope}[-Latex]
		\path [-Latex] (0) edge node[above] {\edgefu{$10$}{$4$}} (1);
		\path [-Latex] (0) edge[bend right] node[above right] {\edgefu{$12$}{$4$}} (2);
		\path [-Latex] (0) edge[bend left] node[above] {\edgefu{$10$}{$4$}} (4);
		\path [-Latex] (1) edge node[above right] {\edgefu{$0$}{$2$}} (2);
		\path [-Latex] (1) edge[snake=snake,segment amplitude=.4mm,segment length=2mm,line after snake=2mm] node[above] {\edgefu{$10$}{$2$}} (3);
		\path [-Latex] (2) edge[snake=snake,segment amplitude=.4mm,segment length=2mm,line after snake=2mm] node[below right] {\edgefu{$10$}{$2$}} (3);
		\path [-Latex] (3) edge node[above] {\edgefu{$0$}{$1$}} (4);
		\path [-Latex] (3) edge node[above] {\edgefu{$8$}{$2$}} (5);
		\path [-Latex, opacity=0.3] (4) edge[snake=snake,segment amplitude=.4mm,segment length=2mm,line after snake=2mm] node[above] {\edgefu{$8$}{$1$}} (6);
		\path [-Latex, opacity=0.3] (5) edge[snake=snake,segment amplitude=.4mm,segment length=2mm,line after snake=2mm] node[above] {\edgefu{$8$}{$1$}} (6);
		\end{scope}
		
		\begin{scope}
		\draw[dashed, color=blue, rounded corners=5mm] (6.4,-1) -- (5.5, 3.75) -- (10.6, 1.75);
		\node[color=blue] at (6,-0.5) {$X^2$};
		\draw[dashed, color=blue, opacity=0.3] (11.5,-1) -- (11.5, 4.5);
		\node[color=blue, opacity=0.3] at (11,-0.5) {$X^1$};
		\end{scope}
		\end{tikzpicture}
	}
\end{figure}
\vfill
Berechne auslastungsminimalen Fluss $x'$ und dünnsten Schnitt.\\
{\color{red}Kapazität $u$}\\
{\color{blue}Balance $b$}\\
{\color{darkgreen}Fluss $x'$}
\end{frame}

\begin{frame}[t]{Berechnung schmaler Flüsse ohne Zurücksetzen (5)}
\begin{figure}
\small
\centering
\makebox[0pt]{
	
	\begin{tikzpicture}[lul/.style={draw,
		ellipse,
		align=center,
		inner sep=0pt,
		outer sep=4pt,
		text width=7mm,
		minimum height=1.5cm
	},
	scale=0.8]
	
	\newnode{0}{(0,2)}{$a$}{$32$}
	\newnode{1}{(3,2)}{$b$}{$-10$}
	\newnode{2}{(5,0)}{$c$}{$-12$}
	\transnode{3}{(7,2)}{$d$}{$-12$}
	\newnode{4}{(10,3)}{$e$}{$-10$}
	\transnode{5}{(10,1)}{$f$}{$-8$}
	\transnode{6}{(13,2)}{$g$}{$-16$}
	
	\begin{scope}[-Latex]
	\path [-Latex] (0) edge node[above] {\edgefu{$10$}{$4$}} (1);
	\path [-Latex] (0) edge[bend right] node[above right] {\edgefu{$12$}{$4$}} (2);
	\path [-Latex] (0) edge[bend left] node[above] {\edgefu{$10$}{$4$}} (4);
	\path [-Latex] (1) edge node[above right] {\edgefu{$0$}{$2$}} (2);
	\path [-Latex, opacity=0.3] (1) edge[snake=snake,segment amplitude=.4mm,segment length=2mm,line after snake=2mm] node[above] {\edgefu{$10$}{$2$}} (3);
	\path [-Latex, opacity=0.3] (2) edge[snake=snake,segment amplitude=.4mm,segment length=2mm,line after snake=2mm] node[below right] {\edgefu{$10$}{$2$}} (3);
	\path [-Latex, opacity=0.3] (3) edge node[above] {\edgefu{$0$}{$1$}} (4);
	\path [-Latex, opacity=0.3] (3) edge node[above] {\edgefu{$8$}{$2$}} (5);
	\path [-Latex, opacity=0.3] (4) edge[snake=snake,segment amplitude=.4mm,segment length=2mm,line after snake=2mm] node[above] {\edgefu{$8$}{$1$}} (6);
	\path [-Latex, opacity=0.3] (5) edge[snake=snake,segment amplitude=.4mm,segment length=2mm,line after snake=2mm] node[above] {\edgefu{$8$}{$1$}} (6);
	\end{scope}
	
	\begin{scope}
	\draw[dashed, color=blue, rounded corners=5mm, opacity=0.3] (6.4,-1) -- (5.5, 3.75) -- (10.6, 1.75);
	\node[color=blue, opacity=0.3] at (6,-0.5) {$X^2$};
	\draw[dashed, color=blue, opacity=0.3] (11.5,-1) -- (11.5, 4.5);
	\node[color=blue, opacity=0.3] at (11,-0.5) {$X^1$};
	\end{scope}
	\end{tikzpicture}
}
\end{figure}
\vfill
Berechne induzierte kleinere Probleminstanz.\\
{\color{red}Kapazität $u$}\\
{\color{blue}Balance $b$}\\
{\color{darkgreen}Fluss $x'$}
\end{frame}


\begin{frame}[t]{Berechnung schmaler Flüsse ohne Zurücksetzen (6)}
\begin{figure}
	\small
	\centering
	\makebox[0pt]{
		
		\begin{tikzpicture}[lul/.style={draw,
			ellipse,
			align=center,
			inner sep=0pt,
			outer sep=4pt,
			text width=7mm,
			minimum height=1.5cm
		},
		scale=0.8]
		
		\newnode{0}{(0,2)}{$a$}{$32$}
		\newnode{1}{(3,2)}{$b$}{$-10$}
		\newnode{2}{(5,0)}{$c$}{$-12$}
		\transnode{3}{(7,2)}{$d$}{$-12$}
		\newnode{4}{(10,3)}{$e$}{$-10$}
		\transnode{5}{(10,1)}{$f$}{$-8$}
		\transnode{6}{(13,2)}{$g$}{$-16$}
		
		\begin{scope}[-Latex]
		\path [-Latex] (0) edge[snake=snake,segment amplitude=.4mm,segment length=2mm,line after snake=2mm] node[above] {\edgefu{$11$}{$4$}} (1);
		\path [-Latex] (0) edge[bend right, decorate, decoration=snake, segment amplitude=.4mm,segment length=2mm,line after snake=2mm] node[above right] {\edgefu{$11$}{$4$}} (2);
		\path [-Latex] (0) edge[bend left] node[above] {\edgefu{$10$}{$4$}} (4);
		\path [-Latex] (1) edge node[above right] {\edgefu{$1$}{$2$}} (2);
		\path [-Latex, opacity=0.3] (1) edge[snake=snake,segment amplitude=.4mm,segment length=2mm,line after snake=2mm] node[above] {\edgefu{$10$}{$2$}} (3);
		\path [-Latex, opacity=0.3] (2) edge[snake=snake,segment amplitude=.4mm,segment length=2mm,line after snake=2mm] node[below right] {\edgefu{$10$}{$2$}} (3);
		\path [-Latex, opacity=0.3] (3) edge node[above] {\edgefu{$0$}{$1$}} (4);
		\path [-Latex, opacity=0.3] (3) edge node[above] {\edgefu{$8$}{$2$}} (5);
		\path [-Latex, opacity=0.3] (4) edge[snake=snake,segment amplitude=.4mm,segment length=2mm,line after snake=2mm] node[above] {\edgefu{$8$}{$1$}} (6);
		\path [-Latex, opacity=0.3] (5) edge[snake=snake,segment amplitude=.4mm,segment length=2mm,line after snake=2mm] node[above] {\edgefu{$8$}{$1$}} (6);
		\end{scope}
		
		\begin{scope}
		\draw[dashed, color=blue, rounded corners=5mm] (0.2, 0) -- (3, 3.5) -- (6, 0.5);
		\node[color=blue] at (0.2,0.6) {$X^3$};
		\draw[dashed, color=blue, rounded corners=5mm, opacity=0.3] (6.4,-1) -- (5.5, 3.75) -- (10.6, 1.75);
		\node[color=blue, opacity=0.3] at (6,-0.5) {$X^2$};
		\draw[dashed, color=blue, opacity=0.3] (11.5,-1) -- (11.5, 4.5);
		\node[color=blue, opacity=0.3] at (11,-0.5) {$X^1$};
		\end{scope}
		\end{tikzpicture}
	}
\end{figure}
\vfill
Berechne auslastungsminimalen Fluss $x'$ und dünnsten Schnitt.\\
{\color{red}Kapazität $u$}\\
{\color{blue}Balance $b$}\\
{\color{darkgreen}Fluss $x'$}
\end{frame}



\begin{frame}[t]{Berechnung schmaler Flüsse ohne Zurücksetzen (7)}
\begin{figure}
	\small
	\centering
	\makebox[0pt]{
		
		\begin{tikzpicture}[lul/.style={draw,
			ellipse,
			align=center,
			inner sep=0pt,
			outer sep=4pt,
			text width=7mm,
			minimum height=1.5cm
		},
		scale=0.8]
		
		\newnode{0}{(0,2)}{$a$}{$10$}
		\transnode{1}{(3,2)}{$b$}{$-10$}
		\transnode{2}{(5,0)}{$c$}{$-12$}
		\transnode{3}{(7,2)}{$d$}{$-12$}
		\newnode{4}{(10,3)}{$e$}{$-10$}
		\transnode{5}{(10,1)}{$f$}{$-8$}
		\transnode{6}{(13,2)}{$g$}{$-16$}
		
		\begin{scope}[-Latex]
		\path [-Latex, opacity=0.3] (0) edge[snake=snake,segment amplitude=.4mm,segment length=2mm,line after snake=2mm] node[above] {\edgefu{$11$}{$4$}} (1);
		\path [-Latex, opacity=0.3] (0) edge[bend right, decorate, decoration=snake, segment amplitude=.4mm,segment length=2mm,line after snake=2mm] node[above right] {\edgefu{$11$}{$4$}} (2);
		\path [-Latex] (0) edge[bend left] node[above] {\edgefu{$10$}{$4$}} (4);
		\path [-Latex, opacity=0.3] (1) edge node[above right] {\edgefu{$1$}{$2$}} (2);
		\path [-Latex, opacity=0.3] (1) edge[snake=snake,segment amplitude=.4mm,segment length=2mm,line after snake=2mm] node[above] {\edgefu{$10$}{$2$}} (3);
		\path [-Latex, opacity=0.3] (2) edge[snake=snake,segment amplitude=.4mm,segment length=2mm,line after snake=2mm] node[below right] {\edgefu{$10$}{$2$}} (3);
		\path [-Latex, opacity=0.3] (3) edge node[above] {\edgefu{$0$}{$1$}} (4);
		\path [-Latex, opacity=0.3] (3) edge node[above] {\edgefu{$8$}{$2$}} (5);
		\path [-Latex, opacity=0.3] (4) edge[snake=snake,segment amplitude=.4mm,segment length=2mm,line after snake=2mm] node[above] {\edgefu{$8$}{$1$}} (6);
		\path [-Latex, opacity=0.3] (5) edge[snake=snake,segment amplitude=.4mm,segment length=2mm,line after snake=2mm] node[above] {\edgefu{$8$}{$1$}} (6);
		\end{scope}
		
		\begin{scope}
		\draw[dashed, color=blue, rounded corners=5mm, opacity=0.3] (0.2, 0) -- (3, 3.5) -- (6, 0.5);
		\node[color=blue, opacity=0.3] at (0.2,0.6) {$X^3$};
		\draw[dashed, color=blue, rounded corners=5mm, opacity=0.3] (6.4,-1) -- (5.5, 3.75) -- (10.6, 1.75);
		\node[color=blue, opacity=0.3] at (6,-0.5) {$X^2$};
		\draw[dashed, color=blue, opacity=0.3] (11.5,-1) -- (11.5, 4.5);
		\node[color=blue, opacity=0.3] at (11,-0.5) {$X^1$};
		\end{scope}
		\end{tikzpicture}
	}
\end{figure}
\vfill
Berechne induzierte kleinere Probleminstanz.\\
{\color{red}Kapazität $u$}\\
{\color{blue}Balance $b$}\\
{\color{darkgreen}Fluss $x'$}
\end{frame}

\begin{frame}[t]{Berechnung schmaler Flüsse ohne Zurücksetzen (8)}
\begin{figure}
	\small
	\centering
	\makebox[0pt]{
		\begin{tikzpicture}[lul/.style={draw,
			ellipse,
			align=center,
			inner sep=0pt,
			outer sep=4pt,
			text width=7mm,
			minimum height=1.5cm
		},
		scale=0.8]
		
		\newnode{0}{(0,2)}{$a$}{$0$}
		\newnode{1}{(3,2)}{$b$}{$2{,}75$}
		\newnode{2}{(5,0)}{$c$}{$2{,}75$}
		\newnode{3}{(7,2)}{$d$}{$5$}
		\newnode{4}{(10,3)}{$e$}{$2{,}5$}
		\newnode{5}{(10,1)}{$f$}{$5$}
		\newnode{6}{(13,2)}{$g$}{$8$}
		
		\begin{scope}[-Latex]
		\path [-Latex] (0) edge node[above] {\edgefu{$11$}{$4$}} (1);
		\path [-Latex] (0) edge[bend right] node[above right] {\edgefu{$11$}{$4$}} (2);
		\path [-Latex] (0) edge[bend left] node[above] {\edgefu{$10$}{$4$}} (4);
		\path [-Latex] (1) edge node[above right] {\edgefu{$1$}{$2$}} (2);
		\path [-Latex] (1) edge node[above] {\edgefu{$10$}{$2$}} (3);
		\path [-Latex] (2) edge node[below right] {\edgefu{$10$}{$2$}} (3);
		\path [-Latex] (3) edge node[above] {\edgefu{$0$}{$1$}} (4);
		\path [-Latex] (3) edge node[above] {\edgefu{$8$}{$2$}} (5);
		\path [-Latex] (4) edge node[above] {\edgefu{$8$}{$1$}} (6);
		\path [-Latex] (5) edge node[above] {\edgefu{$8$}{$1$}} (6);
		\end{scope}
		\end{tikzpicture}
	}
\end{figure}
\vfill
Erhalte schmalen Fluss $x'$ ohne Zurücksetzen.\\
{\color{red}Kapazität $u$}\\
{\color{blue}Knotenauslastung $l'$}\\
{\color{darkgreen}Fluss $x'$}
\end{frame}

\begin{frame}{Berechnung schmaler Flüsse ohne Zurücksetzen (9)}
	\begin{theorem}
		Auf rationalen Instanzen können schmale Flüsse ohne Zurücksetzen mit $\bigO((\size{b} + \size{u}) n^2 m L)$ arithmetischen Operationen berechnet werden, wobei $L\coloneq \abs{\{ l_v \mid v\in V \}}$ die Anzahl der verschiedenen Knotenauslastungen eines schmalen Flusses ohne Zurücksetzen ist.
	\end{theorem}
\end{frame}

\subsection{Berechnung schmaler Flüsse mit Zurücksetzen}
\newcommand{\Brouwer}{\textsc{Brouwer}}
\newcommand{\epsTFwR}{\textsc{$\varepsilon$\nobreakdash-TFwR}}
\newcommand{\PPAD}{\textbf{PPAD}}
\begin{frame}{Berechnung schmaler Flüsse mit Zurücksetzen}
	\pause\begin{center}
		\begin{framed}
			\centering
			\emph{\Brouwer} \\[1em]
			\begin{tabular}{rl}
				{\bfseries Input}: &Polynomiell ber. Funktion $F: [0,1]^m \rightarrow [0, 1]^m$,\\
				& eine Konstante $K$ mit $\norm{F(x) - F(y)} \leq K \norm{x - y}$\\
				&für alle $x,y\in[0,1]^m$ sowie die Genauigkeit $\varepsilon>0$.\\
				{\bfseries Output}: & Ein Punkt $x\in[0,1]^m$ mit $\norm{F(x) - x} \leq \varepsilon$.
			\end{tabular}
		\end{framed}
	\end{center}

	\pause\begin{center}
		\begin{framed}
			\centering
			\emph{Approximate Thin Flow with Resetting: \epsTFwR} \\[1em]
			\begin{tabular}{rl}
				{\bfseries Input}: &Azykl. Netzwerk $(V, E, u\in\Q^E_{>0})$ mit Zurücksetzen\\
				&auf $E_1\subseteq E$, Balancen $b\in \Q^V$ und $\varepsilon > 0$.\\
				{\bfseries Output}: &Ein schmaler $b'$-Fluss mit Zurücksetzen auf $E_1$\\ &und $\norm{b' - b} \leq \varepsilon$.
			\end{tabular}
		\end{framed}
	\end{center}
\end{frame}

\begin{frame}{Berechnung schmaler Flüsse mit Zurücksetzen}
	\begin{lemma}
		Liegt das Problem \Brouwer\ sogar bei Eingabe von Funktionen, die auf einem Polytop in Normalform definiert sind, in~\PPAD, so ist auch das Problem \epsTFwR\ in \PPAD\ enthalten.
	\end{lemma}
\end{frame}
