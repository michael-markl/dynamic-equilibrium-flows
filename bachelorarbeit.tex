\newcommand*{\DocType}{scrreprt}
%\renewcommand*\DocType{article} % Uncomment for screen optimization

\newcommand*\ClassList{scrreprt,article}
\documentclass[\DocType, paper=a4,fontsize=11pt,titlepage]{generalclass}

\usepackage{framed}
\usepackage[utf8]{inputenc}
\usepackage[automark]{scrpage2}	        % Seiten-Stil für scrartcl
% Mathematische Zeichensätze und Umgebungen
\usepackage{amsfonts, amssymb}	        % Definition einer Liste mathematischer Fontbefehle und Symbole
\usepackage[intlimits,sumlimits]{amsmath} % Integral-/Summationsgrenzen über/unter Zeichen
\usepackage{tabto}
\usepackage{mathabx}
\usepackage{mathtools}
\usepackage{subcaption}
% mathematische Verbesserungen
\usepackage{amsthm}	                    % spezielle theorem Stile
\usepackage{aliascnt} 
\usepackage{array}		                % erweiterte Tabellen
% Schriftzeichen, Format
\usepackage{latexsym}		            % Latex-Symbole
\usepackage[english, german, ngerman]{babel} % Mehrsprachenumgebung
% Layout
\usepackage{geometry}                   % Seitenränder
\usepackage{xcolor}                     % Farben
\usepackage{bbm}
% Tabellen und Listen
\usepackage{float}		                % Gleitobjekte 
\usepackage[flushright]{paralist}       % Bessere Behandlung der Auflistungen
\usepackage{datetime}
% Bilder
\usepackage[final]{graphicx}            % Graphiken einbinden
\usepackage{caption}                    % Beschriftungen
\usepackage{subcaption}                 % Beschriftungen für Unterteilung
\usepackage{tikz}
\usepackage[chapter]{algorithm}

\usepackage{algpseudocode}
% Interaktive Referenzen, und PDF-Keys
\usepackage{xr-hyper}  
\usepackage[pagebackref,pdftex, plainpages=false]{hyperref} % Rückreferenz im Literaturverzeichnis, Treiber für ps zu pdf ; für direkt nach pdf: pdftex
\usepackage{enumitem}

\title{Nash Gleichgewichte in Dynamischen Flüssen}
\date{\todo{Get it done!}}
\author{Michael Markl}

%-------------------------------------------------------------------------------
% Kopf-Zeilen
%-------------------------------------------------------------------------------

\pagestyle{scrheadings}		     % Kopfzeilen nach scr-Standard		
\ifx\chapter\undefined 		     % falls Kapitel nicht definiert sind
  \automark[subsection]{section} % Kopf- und Fusszeilen setzen
\else				             % Kapitel sind definiert
  \automark[section]{chapter}	 % Kopf- und Fusszeilen setzen
\fi

%-------------------------------------------------------------------------------
%   Maske für Überschrift 
%-------------------------------------------------------------------------------
% Belegung der notwendigen Kommandos für die Titelseite
\newcommand{\autor}{Markl, Michael} 		% bearbeitender Student
\newcommand{\veranstaltung}{Seminar zur Optimierung und Spieltheorie} 	% Titel des ganzen Seminars
\newcommand{\uni}{Institut für Mathematik der Universität Augsburg} % Universit\"at
\newcommand{\lehrstuhl}{Diskrete Mathematik, Optimierung und Operations Research} % Lehrstuhl
\newcommand{\semester}{Sommersemester 2019}	% Winter-/Sommersemester mit Jahr
\newcommand{\datum}{27.06.2019} 			% Datumsangabe
\newcommand{\thema}{TODO}  		% Titel der Seminararbeit

\newcommand{\ownline}{\vspace{.7em}\hrule\vspace{.7em}} % horizontale Linie mit Abstand

\newcommand{\seminarkopf}{	% Befehl zum Erzeugen der Titelseite 
 \textsc{\autor}  \hfill{\datum} \\ 
\textbf{\veranstaltung} \\ 
\uni \\ 
\lehrstuhl \\
\semester
\ownline 

\begin{center}
{\LARGE \textbf{\thema}}
\end{center}

\ownline
}			% Befehle und Pakete für Titelseite


\DeclareMathOperator{\e}{ex}
\DeclareMathOperator{\ma}{mate}
\DeclareMathOperator{\Ex}{Ex}

%-------------------------------------------------------------------------------
%   Befehle für Nummerierung der Ergebnisse
%   fortlaufend innerhalb eines Abschnittes
%-------------------------------------------------------------------------------
\theoremstyle{plain}            % normaler Stil
\newtheorem{theorem}{Theorem}
% Lemma
\newaliascnt{lemma}{theorem}
\newtheorem{lemma}[lemma]{Lemma}
\aliascntresetthe{lemma}
% Satz
\newaliascnt{satz}{theorem}
\newtheorem{satz}[satz]{Satz}
\aliascntresetthe{satz}
% Korollar
\newaliascnt{corollary}{theorem}
\newtheorem{corollary}[corollary]{Korollar}
\aliascntresetthe{corollary}
% Proposition
\newaliascnt{proposition}{theorem}
\newtheorem{proposition}[proposition]{Proposition}
\aliascntresetthe{proposition}
%-------------------------------------------------------------------------------
\theoremstyle{definition}	% Definitionsstil
% Definition
\newaliascnt{definition}{theorem}
\newtheorem{definition}[definition]{Definition}
\aliascntresetthe{definition}
% Beispiel
\newaliascnt{example}{theorem}
\newtheorem{example}[example]{Beispiel}
\aliascntresetthe{example}
% Problem
\newaliascnt{problem}{theorem}
\newtheorem{problem}[problem]{Problem}
\aliascntresetthe{problem}
% Algorithmus
\newaliascnt{algorithmus}{theorem}
\newtheorem{algorithmus}[algorithmus]{Algorithmus}
\aliascntresetthe{algorithmus}
%-------------------------------------------------------------------------------
\theoremstyle{remark}		% Bemerkungsstil
% Bemerkung
\newaliascnt{remark}{theorem}
\newtheorem{remark}[remark]{Bemerkung}
\aliascntresetthe{remark}
% Vermutung
\newaliascnt{conjecture}{theorem}
\newtheorem{conjecture}[conjecture]{Vermutung}
\aliascntresetthe{conjecture}
% Notation
\newaliascnt{notation}{theorem}
\newtheorem{notation}[notation]{Notation}
\aliascntresetthe{notation}

%-------------------------------------------------------------------------------
% automatische Referenzen mit interaktiven Text
%-------------------------------------------------------------------------------

% Texte
\renewcommand{\theoremautorefname}{Theorem}
\newcommand{\lemmaautorefname}{Lemma}
\newcommand{\satzautorefname}{Satz}
\newcommand{\korollarautorefname}{Korollar}
\newcommand{\propositionautorefname}{Proposition}

\newcommand{\definitionautorefname}{Definition}
\newcommand{\beispielautorefname}{Beispiel}
\newcommand{\problemautorefname}{Problem}
\newcommand{\algorithmusautorefname}{Algorithmus}

\newcommand{\bemerkungautorefname}{Bemerkung}
\newcommand{\vermutungautorefname}{Vermutung}
\newcommand{\notationautorefname}{Notation}

%-------------------------------------------------------------------------------
% Nummerierung der Gleichungen innerhalb der obersten Ebene
%-------------------------------------------------------------------------------
\ifx\chapter\undefined 			% Kapitel sind definiert
  \numberwithin{equation}{section}	% Gleichungsnummern in Section
\else					% Kapitel sind nicht definiert
  \numberwithin{equation}{chapter}	% Gleichungsnummern in Kapiteln
\fi
			% Mathematische Befehle und Pakete

% Literatur-Bibliothek
\bibliographystyle{alphadin-mod}               % deutscher Bibliotheksstil

% Erweiterte Einstellungen zu hyperref

\hypersetup{
        breaklinks=true,        % zu lange Links unterbrechen
        colorlinks=true,        % Färben von Referenzen
        citecolor=black,        % Farbe der Zitate
        linkcolor=black,        % Farbe der Links
        extension=pdf,          % Externe Dokumente können eingebunden werden.
        ngerman,		
	pdfview=FitH,
	pdfstartview=FitH,		
	bookmarksnumbered=true,     % Anzeige der Abschnittsnummern	% pdf-Titel
	pdfauthor={\autor}          % pdf-Autor
}

% Namen für Referenzen 

\newcommand{\ownautorefnames}{
  \renewcommand{\sectionautorefname}{Kapitel}
  \renewcommand{\subsectionautorefname}{Unterkapitel}
  \renewcommand{\subsubsectionautorefname}{\subsectionautorefname}
  \renewcommand{\appendixautorefname}{Anhang}
  \renewcommand{\figureautorefname}{Abbildung}
}

% Rückreferenzentext zur Literatur
\def\bibandname{und}%
\renewcommand*{\backref}[1]{}
\renewcommand*{\backrefalt}[4]{%
\ifcase #1 %
 (Nicht zitiert, also Ergänzungsliteratur.)%
\or
 (Zitiert auf Seite #2.)%
\else
 (Zitiert auf den Seiten #2.)%
\fi
}
\renewcommand{\backreftwosep}{ und~} % seperate 2 pages
\renewcommand{\backreflastsep}{ und~} % seperate last of longer 

			% Befehle und Pakete für Referenzen


\IfClass{article}{ % Optimize for screen
	\geometry{papersize={160mm,161.5mm},margin=5mm} 
}

\renewcommand{\descriptionlabel}[1]{\hspace{\labelsep}\textit{#1}}
\floatname{algorithm}{Algorithmus}
\newcommand{\todo}[1]{{\color{red}#1}}

\numberwithin{figure}{section}	% Abbildungsnummern in Section

\newcommand{\R}{\mathbb{R}}
\newcommand{\Q}{\mathbb{Q}}
\newcommand{\Z}{\mathbb{Z}}
\newcommand{\N}{\mathbb{N}}
\newcommand*\diff{\mathop{}\!\mathrm{d}}
\newcommand{\bigO}{\mathcal{O}}
\renewcommand{\]}{\end{equation*}}

\DeclarePairedDelimiter\abs{\lvert}{\rvert}
\newcommand{\size}[1]{\langle#1\rangle}
\DeclareMathOperator{\rank}{rang}

%%%%%%%%%%%%%%%%%%%%%%%%%%%%%%%%%%%%%%%%%%%%%%%%%%%%%%%%%%%%%%%%%%%%%%%%%%%%%%%%
% Start des Dokuments
\begin{document}

\ownautorefnames		% Änderung einiger automatischen Texte von hyperref (wie in referenz.tex definiert)

\thispagestyle{empty}
\newcommand{\Universitat}{Universität Augsburg}
\begin{titlepage}
  	\begin{center}
		\LARGE \textbf{\Universitat} \\
		\uni \\
		\makebox[0pt]{\text{\lehrstuhl}} \\
	\end{center}
	\vspace*{1em}
	\begin{center}
		\includegraphics[width=5cm,height=5cm]{UNI}
	\end{center}
	
	\vspace*{1em}
	
	\begin{center}
		\Huge\textsc{Bachelorarbeit}
	\end{center}
	
	\vspace*{1cm}
	\ownline
	\begin{center}
		\setstretch{2}{
		{\huge \textbf{\thema}}\par
		}
	\end{center}
	\ownline
	\vspace*{1cm}
	\begin{center}
		\LARGE\textbf{Michael Markl}
	\end{center}
    \vfill
\end{titlepage}

\thispagestyle{empty}
\tableofcontents        % Inhaltsverzeichnis
%\listoffigures         % Abbildungsverzeichnis (eventuell einfügen)
%\listoftables          % Tabellenverzeichnis (eventuell einfügen)
\setcounter{page}{0}    % Eigentlicher Inhalt beginnt auf Seite 1
\clearpage              % neue Seite für eigentlichen Inhalt

\chapter{Einführung}\label{introduction}

Routenplanungsspiele haben in der Analyse und Optimierung von Verkehrs- und Kommunikationsnetzwerken eine große Bedeutung.
Bei der Modellierung solcher Netzwerke wurden dabei in der Vergangenheit häufig statische Flüsse zu Rate gezogen, welche jedoch nicht abbilden können, wie sich Veränderungen des Flusses im Laufe der Zeit innerhalb des Netzwerks, wie sie beispielsweise in Straßennetzen auftreten, auf das System auswirken.
Daher geht man zu dynamischen Flüssen (engl. flows over time) über, welche den zeitlichen Verlauf der Kantenbelegung erfassen.

In dieser Arbeit unterwirft man dynamische Flüsse dem Modell der deterministischen Warteschlangen (engl. deterministic queuing model):
Sobald Partikel eine Kante durchlaufen möchten, reihen sich diese zunächst in eine Warteschlange der Kante ein, die mit einer gewissen Kapazität abgearbeitet wird, um dann nach einer zusätzlichen Verzögerung am Zielknoten anzukommen.
Dabei kann man sich die Kanten eines Netzwerk als Transportbänder vorstellen, die eine bestimmte Kapazität und eine bestimmte Verzögerung haben.
Die Kapazität entspricht der Breite des Bands, welche die Gütermenge, die das Band pro Zeiteinheit transportieren kann,  beschränkt, und die Verzögerung entspricht der Transportzeit, die das Band benötigt, um Güter vom Startknoten zum Zielknoten zu tragen.
Weiter gibt es im Netzwerk eine Quelle, also einen Knoten, an dem kontinuierlich Partikel entstehen, und eine Senke, in dem Partikel aus dem Netzwerk abfließen.

Diesen Flüssen gibt man nun die folgende spieltheoretische Interpretation:
Jedes infinitesimal kleine Partikel, das in einem Fluss an der Quelle zu einem bestimmten Zeitpunkt entsteht, wird als Spieler in einem nicht-atomaren Auslastungsspiel interpretiert:
Dabei verfolgt jeder Spieler das Ziel, in möglichst kurzer Zeit nach Erscheinen an der Quelle zur Senke des Netzwerks zu gelangen.
Jeder Spieler besitzt dabei bereits zur Zeit seines Entstehens über alle nötigen Informationen, die seine tatsächliche Ankunftszeit an der Senke beeinflussen - das heißt, er weiß über den künftigen Verlauf der Warteschlangen an den Kanten Bescheid --, und kann diese nutzen um einen tatsächlich zur aktuellen Zeit kürzesten Quelle-Senke-Pfad zu ermitteln.
Da es sich bei den Spielern um infinitesimal kleine Partikel handelt, verändert sich der dynamische Fluss durch die Entscheidung eines einzelnen Partikels nicht.

Ein Nash-Gleichgewicht wird für eine endliche Anzahl an Spielern für gewöhnlich als eine Strategiewahl charakterisiert, in der kein Spieler seine Strategie bei Beibehalten der Wahl der anderen Spieler ändern kann, um seine eigenen Kosten echt zu verringern.
Im Fall dynamischer Flüsse wird dieser Begriff etwas aufgeweicht:
So wird ein dynamischer Fluss ein Nash-Gleichgewicht genannt, falls eine Kante 
fast nur zu Zeitpunkten genutzt wird, zu denen die Kante den Zielknoten unter allen Pfaden in der kürzest-möglichen Zeit erreicht.

Diese Arbeit gibt zunächst eine formale Definition von dynamischen Flüssen in Kapitel~\ref{chapter-dynamic-flows} sowie eine Charakterisierung von Nash-Gleichgewichten in Kapitel~\ref{chapter-nash-flows}.
Danach widmet sie sich der Berechnung ebensolcher Gleichgewichte für den Fall, dass der Netzwerkzufluss an der Quelle konstant ist.
Dazu werden jedoch einige Kenntnisse über zwei Klassen von statischen Flüssen benötigt:
So werden in Kapitel~\ref{chapter-min-con-flows} zunächst Resultate der Analyse von auslastungsminimalen Flüssen zusammengetragen:
Dies sind $b$-Flüsse, die die maximale Auslastung aller Kanten, d.h. das Verhältnis von Fluss und Kapazität einer Kante, minimieren.
Es wird gezeigt, dass das dazu duale Problem ist, einen sogenannten dünnsten Schnitt zu finden, der das Verhältnis vom Nettoangebot und der Kapazität der ausgehenden Kanten des Schnittes maximiert.
Außerdem wird ein Verfahren vorgestellt, mit dem man auslastungsminimale Flüsse in polynomieller Zeit exakt bestimmen kann.
Weiter führt die Analyse der Ableitung dynamischer Nash-Flüssen zum Konzept sogenannter schmaler Flüsse, die in Kapitel~\ref{chapter-thin-flows}.
Dies sind $b$-Flüsse in einem Netzwerk mit einer einzigen Quelle $s$ und sogenannten zurücksetzenden Kanten, in denen alle $s$-$v$-Pfade, die in der Quelle starten und positiven Fluss haben, minimale Auslastung haben.
Nach einer Charakterisierung dieser Flüsse wird ein polynomieller Algorithmus erarbeitet, der schmale Flüsse in Netzwerken ohne zurücksetzende Kanten berechnet.
Außerdem wird diskutiert, ob das allgemeine Problem mit zurücksetzenden Kanten in der Komplexitätsklasse \PPAD\ liegt.
Schließlich leitet man mithilfe der Existenz solcher schmaler Flüsse ein mögliches Verfahren ab, mit dem Nash-Flüsse bei konstantem Netzwerkzufluss berechnet werden sollen.

\chapter{Dynamische Flüsse}\label{chapter-dynamic-flows}

\section{Grundlegende Definitionen}

Zunächst werden einige grundlegende Begriffe eingeführt.
In der gesamten Arbeit werden grundsätzlich nur gerichtete Graphen mit endlicher Knoten- und Kantenmenge betrachtet.
Dabei bezeichne $\tail(e)$ den Start- und $\head(e)$ den Zielknoten einer Kante $e$.
Außerdem sind parallele Kanten zwischen Knoten stets erlaubt, obwohl häufig die vereinfachte Schreibweise $vw\coloneq (v,w)\coloneq e$ mit $v=\tail(e)$ und $w=\head(e)$ benutzt wird.

Für eine Menge $X$ an Knoten bezeichne $\delta^+(X)\coloneq \{ e\in E \mid \tail(e) \in X \notni \head(e) \}$ die Menge der ausgehenden Kanten von~$X$ und analog bezeichne $\delta^-(X)$ die Menge der eingehenden Kanten in~$X$.
Ist der zugehörige Graph unklar, so schreibt man $\delta^+_M(X)$ bzw. $\delta^-_M(X)$ für ein Netzwerk, einen Graphen oder eine Kantenmenge $M$.
Für die ausgehenden oder eingehenden Kanten eines einzelnen Knotens schreibt man verkürzend $\delta^+(v)$ bzw. $\delta^-(v)$.

Ist weiter $f: E \rightarrow \R$ eine Kantenbewertung und $b: V \rightarrow \R$ eine Knotenbewertung, so schreibt man meist $f_e$ statt $f(e)$ und $b_v$ statt $b(v)$ und für Teilmengen $E'\subseteq E$ und $X\subseteq V$ abkürzend
\[ 
	f(E')\coloneq \sum_{e \in E'} f_e \text{~~~ und ~~~} b(X)\coloneq\sum_{v\in X} b_v.
\]

Ein \emph{Pfad} $P=(e_1, \dots, e_k)$ ist eine Aneinanderreihung von Kanten, das heißt, es gilt $\head(e_i) = \tail(e_{i+1})$ für alle $i\in[k-1]$.
Dabei bezeichnet $[n]$ die Menge der ersten $n$ natürlichen Zahlen, also $\{ 1,\dots, n \}$.
Die Kanten eines Pfades werden in der Menge $E(P)$, die Knoten in der Menge $V(P)$ gesammelt.
Ein Pfad heißt \emph{Weg}, falls kein Knoten mehrmals besucht wird. 
Ein Pfad heißt \emph{Kreis}, falls $\tail(e_1) = \head(e_k)$ gilt, und \emph{Zyklus} oder \emph{elementarer Kreis}, falls zusätzlich $\tail(e_1)$ genau zweimal und sonst kein Knoten mehrmals besucht werden.
Ein Weg oder Zyklus $P$ wird oft als Vektor in $\R^E$ aufgefasst, der als Einträge $1$-en auf Kanten in $P$ und sonst $0$-en enthält.

\begin{definition}
	Ein \emph{Netzwerk $(V, E, u)$} ist ein gerichteter, endlicher Graph $(V, E, u)$ mit \emph{Kapazitäten} $u: E \to \R_{>0}$.
	Im Falle statischer Flüsse ist ein Netzwerk häufig mit Balancen $b:V\to\R$ ausgestattet.
	
	Ein \emph{dynamisches Netzwerk} $(V, E, u, s, t, \tau)$ ist ein Netzwerk $(V, E, u)$, in dem $s$ alle Knoten in $V$ erreicht.
	Dabei heißen $s\in V$ die \emph{Quelle} und $t\in V$ die \emph{Senke} des Netzwerks.
	Jeder Kante $e\in E$ wird zusätzlich eine \emph{Verzögerungszeit} $\tau_e\geq 0$ zugeordnet, wobei alle Zyklen $C$ eine positive Gesamtverzögerung $\tau(E(C))$ haben.
\end{definition}
\begin{definition}[Statischer Fluss]
	Eine Kantenbewertung $f\in\R_{\geq 0}^E$ in einem gerichteten Graphen $(V, E)$ heißt \emph{statischer Fluss}.
	Sein \emph{Balancevektor $b\in \R^V$} ist gegeben durch
	\[ b_v \coloneq f(\delta^+(v)) - f(\delta^-(v)) \text{~~~ für $v\in V$}. \]
	Man nennt $f$ auch einen \emph{$b$-Fluss} oder sagt, \emph{$f$ gewähre Flusserhaltung bzgl. $b$}.
	Gibt es zwei Knoten $s$ und $t$, sodass $b_v$
	für alle Knoten $v\in V\setminus\{ s, t \}$ verschwindet, für $s$ nicht-negativ und für $t$ nicht-positiv ist, so ist $f$ ein \emph{statischer $s$-$t$-Fluss mit Wert $b_s$}.
	Ist $b$ der Nullvektor, so nennt man $f$ auch eine \emph{Strömung}.
\end{definition}

Man bemerke, dass ein statischer Fluss keine Kapazitätsbedingung erfüllen muss.
Außerdem sind für eine Strömung nur nichtnegative Werte zugelassen.
Eine wichtige Aussage bei der Anaylse von statischen Flüssen liefert der Dekompositionssatz (siehe \cite[Satz 8.8]{Korte2012}):
\begin{theorem}[Dekompositionssatz]\label{thm-decomposition}
	Ein statischer $s$-$t$-Fluss $f$ besitzt eine Dekomposition in elementare $s$-$t$-Wege und Zyklen, das heißt es existieren $s$-$t$-Wege $P_1,\dots,P_k$ und Zyklen $C_1, \dots, C_l$ sowie $\lambda_1,\dots,\lambda_k,\mu_1,\dots,\mu_l > 0$ mit \[
		f = \sum_{i\in[k]}\lambda_i P_i + \sum_{i\in[l]} \mu_i C_i.
	\]
	Ist $f$ eine Strömung, so gibt es eine Dekomposition in Zyklen.
\end{theorem}

Die Einführung dynamischer Flüsse und kürzester Wege folgt im Wesentlichen der Darstellung von Cominetti, Correa und Larré aus~\cite{Cominetti2015} mit Ergänzungen von Ronald Koch und Martin Skutella aus~\cite{Koch2011}.
So haben Cominetti u. a. dynamische Flüsse beispielsweise statt mit Lebesgue-integrierbarer Funktionen mit lokal Lebesgue-integrierbaren Funktionen ausgestattet, die den Vorteil bieten, über den gesamten Zeitraum $[0, \infty)$ unendlich viel Fluss schicken zu können.
Dazu wird der folgende Funktionenraum eingeführt:

\begin{definition}
	Der Raum $\mathfrak{F}_0$ sei die Menge der Funktionen $g: \R \to \R_{\geq 0}$, die lokal integrierbar bzgl. des Lebesgue-Maßes sind und auf der negativen Achse verschwinden, die also $\int_a^b |g(t)| \diff t< \infty$ für beliebige beschränkte Intervalle $(a,b)$ und $g(t)=0$ für $t<0$ erfüllen.
\end{definition}

\begin{definition}[Dynamischer Fluss]
	Ein \emph{dynamischer Fluss $f=(f^+, f^-)$} ist ein Paar zweier über die Kanten $E$ eines dynamischen Netzwerks indizierter Familien mit $f^+_e,f^-_e\in\mathfrak F_0$ für alle $e\in E$.
	Dabei bezeichnen $f_e^+(\theta)$ und $f_e^-(\theta)$ die \emph{Zu- bzw. Abflussrate an Kante $e\in E$ zum Zeitpunkt $\theta\in\R$}.
	
	Der (kumulative) \emph{Zu- bzw. Abfluss an einer Kante $e$ bis zum Zeitpunkt $\theta$} sei definiert durch $F^+_e(\theta)\coloneq\int_0^\theta f^+_e(t) \diff t<\infty$ bzw. $F^-_e(\theta)\coloneq\int_0^\theta f^-_e(t) \diff t<\infty$.
	
	Die \emph{(Länge der) Warteschlange $z_e(\theta)$ und die Wartezeit $q_e(\theta)$ an einer Kante $e\in E$ zum Zeitpunkt $\theta\in\R$} seien gegeben durch $z_e(\theta)\coloneq F_e^+(\theta) - F_e^-(\theta + \tau_e)$ und $q_e(\theta) \coloneq z_e(\theta) / u_e$.
	
	Man bezeichne die \emph{Austrittszeit $T_e(\theta)$ aus einer Kante $e\in E$ bei Eintrittszeit $\theta$}, zu der ein Partikel eine Kante verlässt, die es zum Zeitpunkt $\theta$ betreten hat, als $T_e(\theta)\coloneq\theta + q_e(\theta) + \tau_e$.
\end{definition}

Der Definition der Austrittszeit kann man bereits entnehmen, wie sich Partikel, die an einer Kante $e$ ankommen, verhalten sollen:
Nachdem sie zur Zeit $\theta$ die Kante betreten haben, müssen sie sich zunächst in eine Warteschlange einreihen, welche mit der Kapazität $u_e$ nach dem FIFO-Prinzip (First-In-First-Out-Prinzip) abgebaut wird.
Nachdem diese Wartezeit $q_e(\theta)$ vorüber ist, vergeht eine weitere konstante Verzögerungszeit $\tau_e$, bevor sie wieder aus der Kante austreten.
Des Weiteren müssen sich die Partikel bereits sofort bei der Ankunft an einem Knoten entscheiden, in welche Kante sie eintreten wollen, und können nicht an einem Knoten verweilen.
Um das beschriebene Verhalten zu gewährleisten, führt man im Folgenden die Zulässigkeit dynamischer Flüsse ein:

\begin{definition}[Zulässiger dynamischer Fluss]
	Ein dynamischer Fluss $f$ heißt \emph{zulässig}, falls er die folgenden Eigenschaften erfüllt:
	\begin{enumerate}[label=(F\arabic*)]
		\item\label{def-feasible-flow-capacity} Keine Abflussrate übersteigt die Kapazität, d.h. $\forall e\in E, \theta\in\R: f_e^-(\theta)\leq u_e$.
		\item\label{def-feasible-flow-no-negative-flow} Fluss verlässt eine Kante nur, falls er sie zuvor betreten hat,\\ d.h. $\forall e\in E, \theta\in\R: F_e^+(\theta) \geq F_e^-(\theta + \tau_e).$
		\item\label{def-feasible-flow-no-flow-at-node} Bis auf Quelle und Senke erfüllt jeder Knoten $v$ Flusserhaltung,\\
		d.h. $\forall\theta\in\R: \sum_{e\in\delta^+(v)}f^+_e(\theta) - \sum_{e\in\delta^-(v)} f_e^-(\theta) = 0$.\\
		Für die Senke $t$ muss dieser Wert nicht-positiv und für die Quelle $s$ nicht-negativ sein. 
		Für $s$ bezeichnet er den \emph{Zufluss $d(\theta)$ in das Netzwerk}.
		\item\label{def-feasible-flow-queue-with-capacity} Warteschlangen werden mit der Kapazität der Kante abgebaut,\\ d.h. $\forall e\in E, \theta\in\R: z_e(\theta) > 0 \implies f_e^-(\theta + \tau_e) = u_e$.
	\end{enumerate}
\end{definition}

Die folgende Proposition beschreibt wichtige Folgerungen über zulässige dynamische Flüsse:

\begin{proposition}\label{prop-feasible-flow}
	Für eine Kante $e\in E$ und einen zulässigen dynamischen Fluss $f$ gilt:
	\begin{enumerate}[label=(\roman*)]
		\item\label{prop-feasible-flow-T-mon-inc-cont} Die Funktion $\theta \mapsto \theta + q_e(\theta)$ ist monoton wachsend und stetig.
		\item\label{prop-feasible-flow-positive-queue} Für alle $\theta\in\R$ ist die Warteschlange $z_e$ auf dem Intervall $(\theta, \theta + q_e(\theta))$ positiv.
		\item\label{prop-feasible-flow-det-outflow} Zu jeder Zeit $\theta\in\R$ gilt $F_e^+(\theta) = F_e^-(T_e(\theta))$.
		\item\label{prop-feasible-flow-queue-delay} Für alle $\theta_1 \leq \theta_2$ mit $\int_{\theta_1}^{\theta_2} f^+_e(t) \diff t = 0$ und $q_e(\theta_2)>0$ gilt $\theta_1 + q_e(\theta_1) = \theta_2 + q_e(\theta_2)$.
	\end{enumerate}
\end{proposition}
\begin{proof}
	In~\ref{prop-feasible-flow-T-mon-inc-cont} folgt die Stetigkeit bereits aus der Stetigkeit von $F_e^+$ und $F_e^-$.
	Um zu zeigen, dass die Funktion monoton wachsend ist, seien $\theta_1 \leq \theta_2$ gegeben.
	Mit $F_e^-(\theta_2 + \tau_e) = F_e^-(\theta_1+\tau_e) + \int_{\theta_1+\tau_e}^{\theta_2+\tau_e} f_e^-(t)\diff t\leq F_e^-(\theta_1 + \tau_e) + (\theta_2 - \theta_1)u_e$ und mit der Monotonie von $F_e^+$ folgt: 
	\[
		\theta_1 + q_e(\theta_1)
		= \theta_1 + \frac{F_e^+(\theta_1) - F_e^-(\theta_1 + \tau_e)}{u_e}
		\leq \theta_1 + \frac{F_e^+(\theta_2) - F_e^-(\theta_1+\tau_e)}{u_e} \leq \theta_2 + q_e(\theta_2).
	\]
	
	Für $\theta'\in (\theta, \theta+q_e(\theta))$ gilt also $\theta' + q_e(\theta') \geq \theta + q_e(\theta)$, womit $q_e(\theta') \geq \theta + q_e(\theta) - \theta' > 0$ gerade Aussage (ii) beweist.
	
	Aussage (iii) folgt dann mit~\ref{def-feasible-flow-queue-with-capacity} und Aussage~(ii), weil daraus
	\[ 
	\int_{\theta}^{\theta + q_e(\theta)}f_e^-(t + \tau_e) \diff t = q_e(\theta)  u_e = z_e(\theta)
	\]
	folgt, weshalb $F_e^-(T_e(\theta)) = F_e^-(\theta+\tau_e) + \int_{\theta+\tau_e}^{\theta+\tau_e+q_e(\theta)}f_e^-(t)\diff t = F_e^+(\theta)$ gilt.
	
	Zu Aussage (iv):
	Für alle $\theta'\in [\theta_1, \theta_2]$ gilt $F_e^+(\theta') = F_e^+(\theta_2)$.
	Also ist die Warteschlange $z_e(\theta') = F_e^+(\theta_2) - F_e^-(\theta' + \tau_e) \geq z_e(\theta_2)$ positiv und nach~\ref{def-feasible-flow-queue-with-capacity} gilt $f_e^-(\theta' + \tau_e)=u_e$.
	Die Differenz der Warteschlangen erfüllt
	\[
	z_e(\theta_1)-z_e(\theta_2)=-F^-_e(\theta_1 + \tau_e) + F^-_e(\theta_2 + \tau_e) = (\theta_2 - \theta_1)u_e,
	\]
	was $q_e(\theta_1) - q_e(\theta_2) = \theta_2 - \theta_1$ impliziert.
\end{proof}

\section{Kürzeste Wege}\label{sec-travel-times}

In diesem Abschnitt wird der Begriff der frühesten Ankunftszeit an einem Knoten eingeführt und erörtert, wann eine Kante $vw$ in einem kürzesten $s$-$w$-Pfad liegt.

\begin{definition}
	Für einen dynamischen Fluss $f$ und einen Pfad $P=(e_1,\dots,e_k)$ definiere $l^P(\theta)\coloneq T_{e_k}\circ\dots\circ T_{e_1}(\theta)$ den Zeitpunkt, an dem ein Partikel den Endknoten des Pfads erreicht, falls es den Pfad zum Zeitpunkt $\theta$ betritt.
	
	Für einen Knoten $w\in V$ beschreibe $\mathcal{P}_w$ die Menge aller $s$-$w$-Pfade.
	Dann ist die früheste Ankunft eines Partikels, das zur Zeit $\theta$ bei $s$ startet, gegeben durch $l_w(\theta)\coloneq \min_{P\in\mathcal{P}_w}l^P(\theta)$.
	Ein Pfad $P\in \mathcal{P}_w$ heißt \emph{kürzester $s$-$w$-Pfad zur Zeit $\theta$}, falls er $l_w(\theta)=l^P(\theta)$ erfüllt.
\end{definition}

\begin{proposition}\label{prop-abs-cont-sur}
	Für einen zulässigen Fluss $f$ sind die Funktionen $F_e^+$ und $F_e^-$ für alle $e\in E$ lokal absolut stetig.
	Die Funktionen $T_e$, $l^P$ sowie $l_v$ sind dabei für alle Kanten $e\in E$, Pfade $P$ in G und Knoten $v\in V$ monoton wachsend, lokal absolut stetig und surjektiv.
\end{proposition}
\begin{proof}
	Nach dem Hauptsatz der Differential- und Integralrechnung für das Lebes\-gue-Inte\-gral (siehe \cite[Kap. VII, Satz 4.14]{Elstrodt2011}) ist $G: [a,b] \to \R, x\mapsto \int_a^x g(t) \diff t$ für eine Lebesgue-integrierbare Funktion $g: [a,b] \to \R$ absolut stetig.
	Insbesondere sind also $F_e^+$ sowie $F_e^-$ und damit auch $q_e$ und $T_e$ lokal absolut stetig.
	Als Komposition bzw. punktweises Minimum endlich vieler lokal absolut stetiger Funktionen sind auch $l^P$ und $l_v$ für alle Pfade $P$ und Knoten $v$ lokal absolut stetig.
	Nach Proposition~\ref{prop-feasible-flow}~\ref{prop-feasible-flow-T-mon-inc-cont} ist die Monotonie von $T_e$ bereits gegeben, welche auch die Monotonie von $l^P$ und $l_v$ impliziert.
	Wegen $f_e^+, f_e^-\in\mathfrak{F_0}$ gilt $q_e(\theta)=0$ für $\theta\leq 0$, wodurch auch $\lim_{\theta\to-\infty} T_e(\theta) = - \infty$ folgt.
	Mit $T_e(\theta)\geq \theta$ ergibt sich die Surjektivität von $T_e$.
	Daher sind auch $l^P$ und $l_v$ surjektiv.
\end{proof}

Die Monotonie lässt sich spieltheoretisch interpretieren:
Partikel, die zur Zeit $\theta$ in $s$ starten und sich auf einem kürzesten Pfad zu $t$ begeben, kommen zur Zeit $l_t(\theta)$ in $t$ an.
Partikel, die später starten, können also nicht früher in $t$ ankommen.
Wie im statischen Szenario von kürzesten Pfaden, gilt auch hier die Dreiecksungleichung: 

\begin{lemma}\label{lemma-dreicksungl}
	Für einen zulässigen dynamischen Fluss gilt 
	$T_{vw}(l_v(\theta)) \geq l_w(\theta)$ für alle Kanten $vw\in E$.
\end{lemma}
\begin{proof}
	Sei ein kürzester $s$-$v$-Pfad $P$ zum Zeitpunkt $\theta$ gegeben.
	Hängt man an $P$ die Kante $vw$ an, erhält man einen $s$-$w$-Pfad, der zur Eintrittszeit $\theta$ die Ankunftszeit $T_{vw}(l_v(\theta))$ liefert.
	Da $l_w(\theta)$ das Minimum über die Ankunftszeit aller $s$-$w$-Pfade ist, gilt die Behauptung.
\end{proof}

\begin{definition}
	Man bezeichne eine Kante $vw\in E$ als \emph{aktiv zum Zeitpunkt $\theta$}, falls sie auf einem zur Zeit $\theta$ kürzesten $s$-$w$-Pfad liegt; sonst nennt man sie \emph{inaktiv zum Zeitpunkt $\theta$}.
	Es bezeichne $\Theta_e$ die Menge aller Zeitpunkte, zu denen die Kante $e$ aktiv ist, und $G_\theta \coloneq (V, E_\theta)$ den durch die zur Zeit $\theta$ aktiven Kanten induzierten Teilgraphen.
\end{definition}

\begin{proposition}
	Für einen zulässigen Fluss ist eine Kante $vw$ genau dann aktiv zum Zeitpunkt $\theta$, falls $T_{vw}(l_v(\theta)) = l_w(\theta)$ gilt.
	Außerdem ist die Menge $\Theta_{vw}$ abgeschlossen.
\end{proposition}
\begin{proof}
	Ist $vw$ aktiv zum Zeitpunkt $\theta$, existiert ein zur Zeit $\theta$ kürzester $s$-$w$-Pfad $P$, der die Kante $vw$ benutzt.
	Da Zyklen nach Voraussetzung eine positive Gesamtverzögerung haben, ist $P$ ein $s$-$w$-Weg, dessen letzte Kante gerade $vw$ ist.
	Sei also $Q$ das Anfangsstück von $P$ bis zum Knoten $v$.
	Dann gilt aufgrund der Monotonie $
	T_{vw}(l_v(\theta)) \leq T_{vw}( l^Q(\theta) ) = l_w(\theta),
	$
	sodass mit Lemma~\ref{lemma-dreicksungl} sogar Gleichheit gilt.
	
	Gilt umgekehrt $T_{vw}(l_v(\theta)) = l_w(\theta)$ und sei $Q$ ein kürzester $s$-$v$-Pfad, so ist der Pfad $P$, der an $Q$ noch die Kante $vw$ anhängt, ein kürzester $s$-$w$-Pfad zur Zeit $\theta$, der die Kante $vw$ benutzt.
	
	Aufgrund der Stetigkeit von $T_{vw}$ und $l_v$ ist die Menge $\Theta_{vw}$ abgeschlossen.
\end{proof}


Man beachte, dass Teilpfade kürzester Pfade im statischen Sinne wieder kürzeste Pfade sind; im dynamischen Sinne gilt dies nicht unbedingt, jedoch aber in folgendem Teilgraph:

\begin{lemma}\label{lemma-shortest-path-using-active-edges}
	Für einen zulässigen Fluss ist $G_\theta$ zu jeder Zeit $\theta$ ein azyklischer Graph, in dem $s$ jeden Knoten $v\in V$ erreichen kann.
\end{lemma}
\begin{proof}
	Angenommen, es existiere ein Zyklus $C=(e_1, \dots, e_n)$ mit ausschließlich aktiven Kanten.
	Es ist $l^C(\theta) > \theta$, da für Zyklen eine positive Gesamtverzögerung vorausgesetzt ist.
	Setzt man $v:= \tail(e_1)$, so erzeugt $l_{v}(\theta) = l^C(l_{v}(\theta)) > l_{v}(\theta)$ einen Widerspruch aufgrund der Aktivität aller Kanten.
	
	Für jeden Knoten $w\neq s$ existiert mindestens eine eingehende aktive Kante -- zum Beispiel die letzte Kante eines kürzesten $s$-$w$-Pfades, welcher wiederum existiert, weil $w$ von $s$ aus in $G$ erreichbar ist.
	Daher ist $w$ von $s$ aus auch in $G_\theta$ erreichbar.
\end{proof}

\begin{proposition}\label{prop-arrival-times-vector}
	Für einen zulässigen dynamischen Fluss $f$ ist $(l_v(\theta))_{v\in V}$ die eindeutige Lösung des Gleichungssystems
	\[ \tilde{l}_w = \begin{cases}
	\theta, & \text{falls } w=s, \\
	\min\limits_{vw\in \delta^-(w)} T_{vw}(\tilde{l}_v), & \text{sonst}.
	\end{cases} \]
\end{proposition}
\begin{proof}
	Offenbar löst $(l_v(\theta))_{v\in V}$ dieses System, da jeder Knoten $w\neq s$ eine eingehende Kante hat, welche $T_{vw}(l_v) = l_w$ erfüllt.
	Für eine Lösung $(\tilde{l}_v)_{v\in V}$ des Gleichungssystems zeige man $l_w(\theta) = \tilde{l}_w$ für jeden Knoten $w\in V$.
	Dabei ist der Teilgraph $G'=(V, E')$ mit
	\[ E' \coloneq  \{ vw \in E \mid T_{vw}(\tilde{l}_v ) = \tilde{l}_w \} \]
	ein azyklischer Graph, in dem $s$ jeden Knoten $w\in V$ erreichen kann:
	Zyklen können wegen der positiven Gesamtverzögerung nicht entstehen und jeder Knoten $w\neq s$ hat mindestens eine eingehende Kante $vw$ mit $T_{vw}(\tilde{l}_v) = \tilde{l}_w$.
	Daher ist $\tilde{l}_w$ bereits durch einen $s$-$w$-Pfad $P$ in $G'$ festgelegt auf $l^P(\theta)\geq l_w(\theta)$.
	Für einen zur Zeit $\theta$ kürzesten $s$-$w$-Pfad $Q$ gilt außerdem $\tilde{l}_w \leq T^Q(\tilde{l}_s) = T^Q(\theta) = l_w(\theta)$.
\end{proof}

Um den Vektor $(l_v(\theta))_{v\in V}$ für alle $\theta\in\R$ gleichzeitig zu berechnen, kann der Bellman-Ford-Algorithmus auf den Distanzvektor-Funktionen $(l_v)_{v\in V}$ genutzt werden:
Dazu wird in jeder der $n-1$ Iterationen für jede Kante das punktweise Minimum $l_w \coloneq  \min\{ l_w, T_{vw}\circ l_v \}$ gebildet.
Sind Operationen auf Funktionen nicht möglich oder zu teuer, so kann $(l_v(\theta))_{v\in V}$ für ein spezielles $\theta\in\R$ mit dem Dijkstra-Algorithmus ermittelt werden, wobei man die Kosten einer Kante $vw$ erst bei Scanning von $v$ in der Form $q_{vw}(l_v(\theta)) + \tau_{vw}$ berechnet.
\chapter{Dynamische Nash-Flüsse}\label{chapter-nash-flows}

Dieser Abschnitt dient dazu, Nash Gleichgewichte im Kontext dynamischer Flüsse einzuführen.
Dabei hilft die Anschauung, dass Partikel, die zur Zeit $\theta$ an der Quelle erscheinen, in einem Nash Gleichgewicht möglichst früh, also zum Zeitpunkt $l_t(\theta)$, an der Senke ankommen.

\section{Charakterisierung dynamischer Nash-Flüsse}


Für die formale Einführung dynamischer Nash-Flüsse benötigt man weitere Definitionen: 

\begin{definition}
	Für eine Kante $vw\in E$ bezeichne $x_{vw}^+(\theta):= F_{vw}^+(l_v(\theta))$ den Zufluss bis zur frühestmöglichen Ankunftszeit von Partikeln in $v$, die zur Zeit $\theta$ in $s$ starten.\\
	Dagegen bezeichne $x_{vw}^-(\theta):= F^-_{vw}(l_w(\theta))$ den Abfluss bis zur frühestmöglichen Ankunftszeit von Partikeln in $w$, die zur Zeit $\theta$ in $s$ starten.
	
	Für einen Knoten $v\in V$ sei $b_v(\theta):=\sum_{e\in\delta^+(v)} x_e^+(\theta) - \sum_{e\in\delta^-(v)} x_e^-(\theta)$ die Balance des Knoten $v$ zum Zeitpunkt $\theta$.
\end{definition}

\begin{remark}\label{remark-x^-leqx^+}
	In einem zulässigen Fluss gilt nach Proposition~\ref{prop-feasible-flow}~\ref{prop-feasible-flow-det-outflow} und mit der Monotonie von $F_{vw}^-$ bereits 
	\[
	x_{vw}^-(\theta) = F_{vw}^-(l_w(\theta)) \leq F_{vw}^-(T_{vw}(l_v(\theta)))=F_{vw}^+(l_v(\theta)) = x_{vw}^+(\theta).
	\]
\end{remark}

\begin{lemma}\label{lemma-balance-0}
	Für einen zulässigen dynamischen Fluss $f$ gilt $b_v(\theta)=0$ für alle Knoten $v\in V\setminus\{ s,t \}$ und alle $\theta\in\R$.
\end{lemma}
\begin{proof}
	Unter Benutzung der Voraussetzung~\ref{def-feasible-flow-no-flow-at-node} folgere man für $v\in V\setminus \{ s, t\}, \theta\in\R$:
	\[ \sum_{e\in\delta^-(v)} x_e^-(\theta) = \int_{0}^{l_v(\theta)} \sum_{e\in\delta^-(v)} f_e^-(t) \diff t = \int_{0}^{l_v(\theta)} \sum_{e\in\delta^+(v)} f_e^+(t) \diff t = \sum_{e\in\delta^+(v)}x_e^+(\theta). \]
\end{proof}

\begin{notation}
	 $M^c:= \R\setminus M$ bezeichne das Komplement von $M\subseteq\R$, $\overline{M}$ den Abschluss.
\end{notation}

\begin{definition}\label{def-flow-along-active-edges}
	Man sage, der Fluss $f$ \emph{fließe nur entlang aktiver Kanten}, falls $f_{vw}^+$ fast überall auf $l_v(\Theta_{vw}^c)$ verschwindet für alle Kanten $vw\in E$.
\end{definition}

\begin{remark}
	Diese Definition weicht von der Definition von Koch und Skutella ab und entspricht derjenigen aus~\cite[Definition 1]{Cominetti2015}:
	Nach \cite[Definition 2]{Koch2011} sagt man, $f$ \emph{sende Fluss nur entlang aktuell kürzester Pfade}, falls $f_{vw}^+\circ l_v$ fast überall auf $\Theta_{vw}^c$ verschwindet für alle Kanten $vw$.

	Entspricht $f$ dieser Definition, so auch Definition~\ref{def-flow-along-active-edges}: 
	Da $l_v$ nach Proposition~\ref{prop-abs-cont-sur} absolut stetig ist, bildet es nach~\cite[Aufgabe 4.9]{Elstrodt2011Abs} Nullmengen wieder auf Nullmengen ab, weshalb folgende Menge eine Nullmenge ist: \[ l_v(\{ \theta \in \Theta_{vw}^c \mid f_{vw}^+ (l_v(\theta)) > 0 \}) = \{ \xi \in l_v(\Theta_{vw}^c) \mid f_{vw}^+ (\xi) > 0 \}. \]
	 
	Koch und Skutella zeigen im Beweis von~\cite[Lemma 1]{Koch2011} die entsprechende Äquivalenz von Lemma~\ref{lemma-only-active-edges} (i) und (iii) -- jedoch in (i) unter Verwendung ihrer Definition -- und
	verwenden bei der Implikation (iii)$\Rightarrow$(i) das Argument, dass für jede Kante $vw\in E$ und alle $\theta\in \Theta_{vw}^c$ eine Umgebung $U$ von $\theta$ existiert, sodass $f_{vw}^+$ fast überall in $l_v(U)$ verschwindet.
	Dies reicht aber nicht aus, um zu zeigen, dass $f_{vw}^+(l_v(\theta))=0$ für fast alle $\theta\in\Theta_{vw}^c$ gilt:
	So kann $f_{vw}^+(l_v(\theta))$ für ein $\theta\in\Theta_{vw}^c$ positiv sein und $l_v$ konstant in einer Umgebung um $\theta$.
	Dann ist $f_{vw}^+ \circ l_v$ in einer Umgebung um $\theta$ positiv, was im Widerspruch zur Forderung ist.
	
	Dies wurde in~\cite[Example 2]{Cominetti2015} ausgenutzt, um einen Beispielfluss anzugeben, der beweist, dass die Forderung von Koch und Skutella sogar echt stärker ist.
\end{remark}

Für eine äquivalente Umschreibung dieser Definition, benötigen wir folgendes Lemma der Maßtheorie:

\begin{lemma}\label{lemma-vanishes-intervals}
	Seien $g: \R \to \R_{\geq 0}$ eine lokal Lebesgue-integrierbare Funktion und $((a_i, b_i))_{i\in I}$ eine Familie offener Intervalle.
	Dann verschwindet $g$ fast überall auf $\Theta:=\bigcup_{i\in I} (a_i, b_i)$ genau dann, wenn es für alle $i\in I$ fast überall auf $(a_i, b_i)$ verschwindet.
\end{lemma}
\begin{proof}
	Verschwindet $g$ fast überall auf $\Theta$, so erst recht auf jedem Intervall $(a_i, b_i)$.
	Für die andere Richtung definiert die Funktion $\mu(A):= \int_A g \diff \lambda$ ein Maß auf den Borelmengen~$\mathfrak{B}$.
	Da jede offene Menge $O\subseteq\R$ $\sigma$-kompakt ist, also eine Darstellung als abzählbare Vereinigung kompakter Mengen -- hier $O=\bigcup_{n\in\N}(\{ x \in\R \mid d(x, O^c) \geq 1/n \} \cap [-n, n] )$ -- besitzt, ist jede offene Menge nach~\cite[1.2 Folgerungen (e)]{Elstrodt2011Top} innen regulär.
	Das heißt, es gilt
	\[ \mu(O)=\sup\{ \mu(K) \mid K\subseteq O \text{ kompakt} \} \]
	für offene Mengen $O\subseteq\R$.
	Für ein kompaktes $K\subseteq \Theta$ existiert eine endliche Teil\-über\-deckung $\bigcup_{i=1}^n (a_i, b_i) \supseteq K$, für die $\mu(K) \leq \sum_{i=1}^{n} \mu((a_i, b_i)) = \sum_{k=1}^{n} \int_{a_i}^{b_i} g(t) \diff t = 0$ gilt.
	Also ist auch $\mu(\Theta)=0$.
\end{proof}

\begin{lemma}\label{lemma-only-active-edges}
	Für einen zulässigen Fluss $f$ sind folgende Aussagen äquivalent:
	\begin{enumerate}[label=(\roman*)]
		\item Der Fluss $f$ fließt nur entlang aktiver Kanten.
		\item Für jede Kante $vw\in E$ und für fast alle $\xi\in\R$ mit	$f_{vw}^+(\xi)>0$ gilt $\xi \in l_v(\Theta_{vw})$.
		\item Für jede Kante $e\in E$ und für alle $\theta\in\R$ gilt $x_e^+(\theta) = x_e^-(\theta)$.
	\end{enumerate}
\end{lemma}
\begin{proof}
	$(i) \Leftrightarrow (ii)$: Bedingung~(ii) gilt genau dann, wenn $f_{vw}^+$ fast überall auf $l_v(\Theta_{vw})^c$ verschwindet.
	Daher genügt es, zu zeigen, dass sich $l_v(\Theta_{vw})^c$ und $l_v(\Theta_{vw}^c)$ nur um eine Nullmenge voneinander unterscheiden.
	Mit der Surjektivität von $l_v$ gilt $l_v(\Theta_{vw})^c\subseteq l_v(\Theta_{vw}^c)$.
	
	Des Weiteren ist $S:=l_v(\Theta_{vw}^c)\setminus l_v(\Theta_{vw})^c = l_v(\Theta_{vw}^c)\cap l_v(\Theta_{vw})$ eine Teilmenge von $l_v(\Q)$:
	Für ein $\xi\in S$ gibt es $\theta\in\Theta_{vw}^c$ und $\theta'\in\Theta_{vw}$ mit $l_v(\theta)=\xi=l_v(\theta')$.
	Da $\theta\neq\theta'$ ist, existiert ein $\theta_q\in\Q\cap(\theta,\theta')$.
	Wegen der Monotonie von $l_v$ gilt $l_v(\theta_q)=\xi$, womit $\xi\in l_v(\Q)$ folgt.
	Also unterscheiden sich die beiden Mengen nur um eine abzählbare Menge.
	
	$(i)\Leftrightarrow (iii)$: Sei eine Kante $vw\in E$ gegeben.
	Für ein $\theta\in\R$ bezeichne $\omega_\theta\leq \theta$ den spätesten Startzeitpunkt, sodass man unter Benutzung von $vw$ zum Zeitpunkt $l_w(\theta)$ zu $w$ gelangt:
	\[ \omega_\theta:=\max\{ \omega\leq\theta \mid l_w(\theta) = T_{vw}(l_v(\omega)) \}. \]
	
	Es gilt $\Theta_{vw}^c = \bigcup_{\theta\in\R} (\omega_\theta, \theta)$:
	Für $\theta\in\Theta_{vw}^c$ gilt $T_{vw}(l_v(\theta)) > l_w(\theta)$.
	Aufgrund der Stetigkeit von $T_{vw}\circ l_v$ und von $l_w$ existiert ein $\varepsilon>0$, sodass $T_{vw}(l_v(\theta')) > l_w(\theta+\varepsilon)$ für $\theta'\in[\theta,\theta+\varepsilon]$ gilt.
	Also ist $\theta\in(\omega_{\theta+\varepsilon}, \theta+\varepsilon)$.
	Ist umgekehrt $\theta'\in (\omega_\theta,\theta)$, so ist aufgrund der Monotonie $T_{vw}(l_v(\theta'))\geq T_{vw}(l_v(\omega_\theta)) = l_w(\theta)\geq l_w(\theta')$.
	Die erste Ungleichung kann nicht mit Gleichheit erfüllt sein, da $\omega_\theta$ maximal mit der Eigenschaft $T_{vw}(l_v(\omega)) = l_w(\theta)$ ist, wodurch $\theta'\in\Theta_{vw}^c$ folgt.
	
	Mit $l_v(\Theta_{vw}^c) = \bigcup_{\theta\in\R}(l_v(\omega_\theta),l_v(\theta))$ verschwindet $f_{vw}^+$ nach Lemma~\ref{lemma-vanishes-intervals} genau dann fast überall auf $l_v(\Theta_{vw}^c)$, wenn es für alle $\theta\in\R$ fast überall auf $(l_v(\omega_\theta),l_v(\theta))$ verschwindet.
	Dies ist nach Proposition~\ref{prop-feasible-flow}~\ref{prop-feasible-flow-det-outflow} wiederum äquivalent zu
	$F_{vw}^+(l_v(\theta))-F_{vw}^-(l_w(\theta))=0$ für alle $\theta\in\R$.
\end{proof}

\begin{definition}
	Man sage, ein zulässiger dynamischer Fluss $f$ \emph{fließe ohne Über\-holungs\-möglichkeiten}, falls $b_s(\theta) = -b_t(\theta)$ für alle $\theta\in\R$.
\end{definition}

Dabei betrachte man folgende Intuition:
 Partikel, die zur Zeit $\theta\in\R$ bei $s$ starten und sich auf einem kürzesten Weg zu $t$ bewegen  -- also zur Zeit $l_t(\theta)$ in $t$ ankommen --, überholen andere Partikel, falls $b_s(\theta) > -b_t(\theta)$.
Falls jedoch $b_s(\theta) < - b_t(\theta)$ gilt, wurde das Partikel bereits von anderen überholt.
Ein Nash-Gleichgewicht sollte diese Eigenschaft daher erfüllen.

\begin{definition}
	Seien ein statischer Fluss $f \in \R^E$ in einem Graphen $G=(V,E)$ mit Kapazitäten $u\in \R_+^E$ und ein Balancevektor $b\in\R^V$ mit $\sum_{v\in V} b_v = 0$ gegeben.
	Der Fluss $f$ heißt \emph{$b$-Fluss}, falls er Flusserhaltung bzgl. $b$ gewährt, d.h. falls alle $v\in V$ die Bedingung $\sum_{e\in\delta^+(v)}f_e - \sum_{e\in\delta^-(v)}f_e = b_v$ erfüllen.
\end{definition}

\newcommand{\newv}{\mathbf{v}}
\begin{lemma}\label{lemma-b-graph}
	Seien ein dynamischer Fluss $f$ in einem Graphen $G=(V,E)$ und ein Zeitpunkt $\theta\in\R$ gegeben.
	Der Graph $H$ entstehe aus $G$, indem man jede Kante $vw\in E$ aus $G$ durch einen neuen Knoten $\newv_{vw}$ und zwei Kanten $v\newv_{vw}$ und $\newv_{vw}w$ ersetze.
	Der statische Fluss $g$ auf $H$ sei definiert durch
	\[ g_{v\newv_{vw}} := x_{vw}^+(\theta) \text{ und } g_{\newv_{vw}w} := x_{vw}^-(\theta) \text{ für alle $vw\in E$} \]
	und die Balance $b$ auf $H$ sei gegeben durch $b_v:= b_v(\theta)$ für $v\in V$ und $b_{\newv_e}:= x_e^-(\theta) - x_e^+(\theta)$ für $e\in E$.
	Dann gelten die folgenden Aussagen:
	
	\begin{enumerate}[label=(\roman*)]
		\item Der Fluss $g$ ist ein statischer $b$-Fluss.
		\item\label{lemma-b-graph-imp} Ist $f$ zulässig, so gilt $\forall e\in E : x_e^+(\theta) = x_e^-(\theta)\iff b_s(\theta) + b_t(\theta) = 0$.
	\end{enumerate}
\end{lemma} 
\begin{proof}
	$(i)$: Um zu zeigen, dass die Summe über die Balanceeinträge verschwindet, erkenne man, dass der Anteil einer Kante $e\in E$ in $\sum_{v\in V} b_v$ gerade $x_e^+(\theta) - x_e^-(\theta)$ ist.
	Damit gilt:
		\[ \sum_{v\in V}b_v + \sum_{e\in E} b_{\newv_e} = \sum_{e\in E}  (x_e^+(\theta) - x_e^-(\theta) + x_e^-(\theta) - x_e^+(\theta)) = 0. \]
		Es bleibt zu zeigen, dass $g$ bezüglich $b$ Flusserhaltung gewährt.
		Für die Knoten der Form $\newv_{vw}$ gilt dies, da $g_{\newv_{vw}w} - g_{v\newv_{vw}} = x_{vw}^-(\theta) - x_{vw}^+(\theta) = b_{\newv_{vw}}$.
		Für $v\in V$ gilt nach Konstruktion
		\[ b_v =
		\sum_{e\in\delta^+_G(v)} x_{e}^+(\theta) - \sum_{e\in\delta^-_G(v)} x_{e}^-(\theta) =
	\sum_{e\in\delta_H^+(v)} g_e - \sum_{e\in\delta^-_H(v)}g_e
		. \]
	
	$(ii)$: Tatsächlich benötigt man aus $(i)$ nur die Eigenschaft, dass die Summe über die Einträge des Balancevektors verschwindet.
	Mit Lemma~\ref{lemma-balance-0} gilt wegen der Zulässigkeit von $f$ sogar $b_s(\theta)+b_t(\theta) + \sum_{e\in E} b_{\newv_e} = 0$.
	
	Angenommen, es gelte $x_e^+(\theta) = x_e^-(\theta)$ für alle $e\in E$.
	Dann sind auch alle $b_{\newv_e} = 0$ und es gilt $b_s(\theta) + b_t(\theta) = 0$.
	Setzt man $b_s(\theta) + b_t(\theta) = 0$ voraus, so ist $\sum_{e\in E} b_{\newv_e} = 0$ und, da $f$ zulässig ist, gilt $x_e^-(\theta)\leq x_e^+(\theta)$ nach Bemerkung~\ref{remark-x^-leqx^+}.
	Daher gilt $b_{\newv_e}\leq 0$ für alle $e\in E$, weshalb bereits alle $b_{\newv_e} = 0$ sein müssen.
\end{proof}

Die Ergebnisse aus Lemma~\ref{lemma-only-active-edges} und Lemma~\ref{lemma-b-graph} werden im folgenden Theorem gesammelt, welches Nash-Gleichgewichte in dynamischen Flüssen charakterisiert:

\begin{theorem}[Charakterisierung dynamischer Nash-Flüsse]\label{thm-equivalencies-nash-flow}
	Ist $f$ ein zu\-läs\-siger dynamischer Fluss, so sind die folgenden Aussagen äquivalent:
	\begin{enumerate}[label=(\roman*)]
		\item Der Fluss $f$ fließt nur entlang aktiver Kanten.
		\item Für alle Kanten $e\in E$ und zu jeder Zeit $\theta\in\R$ gilt $x_e^+(\theta) = x_e^-(\theta)$.
		\item Der Fluss $f$ fließt ohne Überholungsmöglichkeiten.
	\end{enumerate}
	Gilt eine dieser Aussagen, so nennt man $f$ einen \emph{dynamischen Nash-Fluss}.
\end{theorem}

\section{Eigenschaften dynamischer Nash-Flüsse}

In diesem Abschnitt werden einige Ergebnisse über Nash-Flüsse gesammelt, die in Abschnitt~\ref{sec-nash-flow-extension} benötigt werden.

\begin{remark}\label{remark-s-t-flow}
	In einem Nash-Fluss ist der statische Fluss $x(\theta)$ mit $x_e(\theta):=x_e^+(\theta)=x_e^-(\theta)$ nach Lemma~\ref{lemma-balance-0} für alle $\theta\in\R$  ein statischer $s$-$t$-Fluss.
	Wegen der Monotonie von $x_e$ ist auch $x(\theta_2) - x(\theta_1)$ für $\theta_1 \leq \theta_2$ ein statischer $s$-$t$-Fluss, genauso wie $x'(\theta)$, falls $x_e$ für alle $e\in E$ differenzierbar in $\theta$ ist, da Differenzieren die Flusserhaltung erhält und $x_e$ monoton wachsend ist für alle $e\in E$.
\end{remark}

\begin{lemma}\label{lemma-x-locally-constant}
In einem dynamischen Nash-Fluss ist $x_e$ eingeschränkt auf $\overline{\Theta_e^c}$, also dem Abschluss der Menge der inaktiven Zeitpunkte von $e$, für jede Kante $e\in E$ lokal konstant.
\end{lemma}
\begin{proof}
Da $\Theta_{vw}^c$ eine in $\R$ offene Menge ist, hat sie eine Darstellung als abzählbare Vereinigung paarweise disjunkter offener Intervalle.
Innerhalb eines solchen Intervalls $(\theta_1, \theta_2)$ gilt $x_{vw}(\theta_2) - x_{vw}(\theta_1) = \int_{l_v(\theta_1)}^{l_v(\theta_2)} f_{vw}^+(t) \diff t = 0$, da $f$ nur entlang aktiver Kanten fließt.
Der Rest folgt mit der Monotonie und Stetigkeit von $x_{vw}$.
\end{proof}

\begin{lemma}\label{lemma-nash-flow-waiting-queue-implies-active-edge}
	Seien ein dynamischer Nash-Fluss $f$, eine Kante $vw\in E$ und ein Zeitpunkt $\theta\in\R$ gegeben.
	Gilt eine der folgenden Aussagen, so ist $vw$ zum Zeitpunkt $\theta$ aktiv:
	\begin{enumerate}[label=(\roman*)]
		\item Die Ableitung $x_{vw}'(\theta)$ existiert und es gilt $x_{vw}'(\theta)> 0$.
		\item Die Wartezeit $q_{vw}$ an der Kante $vw$ ist zur Zeit $l_v(\theta)$ positiv.
	\end{enumerate}
	Insbesondere verschwindet die Wartezeit $q_{vw}(l_v(\theta))$ für alle $\theta\in\overline{\Theta_{vw}^c}$.
\end{lemma}
\begin{proof}
	Zu Aussage (i): Angenommen, $vw$ wäre zum Zeitpunkt $\theta$ nicht aktiv, so würde wegen der Offenheit von $\Theta_{vw}^c$ und Lemma~\ref{lemma-x-locally-constant} die Ableitung $x_{vw}'(\theta)$ verschwinden.
	
	Für Aussage (ii) zeige man $T_{vw}(l_v(\theta)) \leq l_w(\theta)$.
	Sei $\theta_1$ der früheste Zeitpunkt mit $x_{vw}^+(\theta_1)= x_{vw}^+(\theta)$.
	Dieser existiert, da $l_v$ nach Proposition~\ref{prop-abs-cont-sur} surjektiv ist.
	Dann ist $\theta_1\in \Theta_{vw}$ nach Lemma~\ref{lemma-x-locally-constant}.
	Außerdem ist $\theta_1 \leq \theta$ wegen der Monotonie von $F_{vw}^+ \circ l_v$.
	Nach Aussage (i) gilt nun $T_{vw}(l_v(\theta_1)) = l_w(\theta_1)$.
	Nach Proposition~\ref{prop-feasible-flow}~\ref{prop-feasible-flow-queue-delay} ist $T_{vw}(l_v(\theta_1)) = T_{vw}(l_v(\theta))$ und mit der Monotonie von $l_w$ folgt $T_{vw}(l_v(\theta))\leq l_w(\theta)$.
\end{proof}

\begin{proposition}\label{prop-nash-flow-s-t-path-decomposable}
	Für einen dynamischen Nash-Fluss $f$ und zwei Zeitpunkte $\theta_1 \leq \theta_2$ ist der statische $s$-$t$-Fluss $x(\theta_2) - x(\theta_1)$ eine Komposition von $s$-$t$-Wegen.
\end{proposition}
\begin{proof}
	Sei $\theta$ das Infimum aller Zeitpunkte $\xi\geq\theta_1$, zu denen $x(\xi) - x(\theta_1)$ nicht in $s$-$t$-Wege zerlegbar ist.
	Man nehme $\theta \leq \theta_2$ an.
	Da inaktive Kanten zum Zeitpunkt $\theta$ bereits kurz vor $\theta$ und noch kurz nach $\theta$ inaktiv sind, existiert ein Intervall $[\theta - \varepsilon, \theta + \varepsilon]$, in der keine inaktive Kante aktiv wird.
	Außerdem existiert für $\xi_0 := \max \{ \theta_1, \theta - \varepsilon \}$ eine $s$-$t$-Wegezerlegung von $x(\xi_0) - x(\theta_1)$.
	
	Für einen Pfad $P$ und einen statischen Fluss $g$ sei $g^P := \min_{e\in P} g_e$ der Fluss, der auf dem Pfad $P$ fließt.
	Für einen Zyklus $C$ ist $(x(\xi) - x(\xi_0))^C = 0$ für $\xi\in [\xi_0, \theta+\varepsilon]$, da aufgrund der Azyklizität von $G_{\xi_0}$ eine Kante $e\in C$ des Zyklus existiert, die zur Zeit $\xi_0$ und damit in ganz $[\xi_0, \theta+\varepsilon]$ inaktiv ist, wodurch $x_e(\xi_0) = x_e(\xi)$ nach Lemma~\ref{lemma-x-locally-constant} folgt.
	
	Also hat der $s$-$t$-Fluss $x(\xi) - x(\xi_0)$ für $\xi\in [\xi_0, \theta + \varepsilon]$ keinen Zyklus mit positivem Fluss und besitzt daher eine $s$-$t$-Wegezerlegung.
	Addiert man diese zur $s$-$t$-Wegezerlegung von $x(\xi_0) - x(\theta_1)$, so erhält man eine $s$-$t$-Wegezerlegung von $x(\xi) - x(\theta_1)$, was für $\xi > \theta$ einen Widerspruch zur Definition von $\theta$ darstellt.
\end{proof}

\begin{corollary}
	Für einen dynamischen Nash-Fluss $f$ ist der statische $s$-$t$-Fluss $x(\theta)$ zu jeder Zeit $\theta$ eine Komposition von $s$-$t$-Wegen.
\end{corollary}
\begin{proof}
	Nach Proposition~\ref{prop-abs-cont-sur} existiert ein Zeitpunkt $\xi_0$ mit $l_v(\xi_0) \leq 0$ für alle Knoten $v\in V$.
	Für $\theta \leq \xi_0$ ist $x(\theta)$ der Nullfluss und offenbar in $s$-$t$-Wege zerlegbar, da die Funktionen $f_e^+$ und $f_e^-$ links der $y$-Achse verschwinden.
	Sonst ist $x(\theta)=  x(\theta) - x(\xi_0)$ nach Proposition~\ref{prop-nash-flow-s-t-path-decomposable} in $s$-$t$-Wege zerlegbar.
\end{proof}
\chapter{Auslastungsminimale $b$-Flüsse}

Ein effiziente Berechnung von Nash-Gleichgewichten erfordert zunächst Verständnis von speziellen Klassen von statischen Flüssen.
Eine wichtige Klasse sind die sogenannten auslastungsminimalen $b$-Flüsse:

\begin{definition}[Auslastungsminimaler $b$-Fluss]
	Sei ein $b$-Fluss $f$ auf einem Netzwerk mit Kapazitäten $u\in\R_{>0}^E$ gegeben.
	Dann heißt $f_e/u_e$ die \emph{Auslastung der Kante $e$ durch $f$} und die maximale Kantenauslastung $c(f):=\max_{e\in E} f_e/u_e$ bezeichnet die \emph{Auslastung des Flusses $f$}.
	Eine Kante mit Auslastung $c(f)$ wird auch als \emph{Flaschenhalskante} (engl. bottleneck edge) bezeichnet.
	Ein $b$-Fluss mit minimaler Auslastung wird dann \emph{auslastungsminimaler $b$-Fluss} genannt.
\end{definition}

In diesem Kapitel wird nun ein Optimalitätskriterium und ein effizienter Algorithmus zur Berechnung solcher auslastungsminimaler $b$-Flüsse vorgestellt.

\section{Optimalitätskriterium auslastungsminimaler $b$-Flüsse}

In diesem Abschnitt wird ein hinreichendes sowie notwendiges Kriterium auslastungsminimaler $b$-Flüsse erarbeitet.

Ähnlich zum wohlbekannten Max-Flow-Min-Cut-Theorem von Ford und Fulkerson betrachtet man dabei ein duales Problem, das die Auslastung von Schnitten involviert.

\begin{definition}[Schnitt]
	In einem gerichteten Netzwerk $(V, E, u)$ heißt eine Teilmenge $X\subseteq V$ \emph{Schnitt}, wobei die aus $X$ ausgehenden Kanten mit $\delta^+(X)$ und die in $X$ eingehenden Kanten mit $\delta^-(X)$ bezeichnet werden.
	
	Sind zusätzlich Knotenbalancen $b\in\R^V$ mit $\sum_{v\in V} b_v = 0$ gegeben und ist $\delta^+(X)$ nichtleer, so bezeichne $b(X) / u(\delta^+(X))$ die \emph{Auslastung des Schnittes $X$}.
	Dabei ist $b(X)$ bzw. $u(E')$ eine Kurzschreibweise für $\sum_{v\in X} b_v$ bzw. $\sum_{e\in E'} u_e$.
	Existiert ein Schnitt, dessen Auslastung maximal ist, so nennt man ihn einen \emph{dünnsten Schnitt}.
\end{definition}
Man bemerke, dass ein dünnster Schnitt existiert, wenn die Kantenmenge $E$ nichtleer ist.
\begin{definition}[Doppelgraph]
	Der Doppelgraph $G^\leftrightarrow$ eines Graphen $G=(V,E)$ ist das Paar $(V, \overrightsmallarrow{E}\cup \overleftsmallarrow{E})$, wobei $\overrightsmallarrow{E}:=\{ \overrightsmallarrow{e} \mid e\in E \}$ die Menge der Vorwärtskanten, die die Richtung der Ursprungskante beibehalten, und $\overleftsmallarrow{E}:=\{ \overleftsmallarrow{e} \mid e\in E \}$ die Menge der Rückswärtskanten, die die Richtung der Ursprungskante umkehren, sind.
\end{definition}
\begin{definition}[Residualgraph eines $b$-Flusses]
	Sei ein $b$-Fluss $f$ auf einem Netzwerk $(V, E)$ mit Kapazitäten $u\in\R^E_{>0}$ gegeben.
	Der \emph{Residualgraph von $G$ bezüglich $f$} ist definiert durch $G_f := (V, E_f)$ mit \[
	E_f := \{ \overrightsmallarrow{e}\in \overrightsmallarrow{E} \mid f_e/u_e < c(f) \} \cup \{ \overleftsmallarrow{e} \in \overleftsmallarrow{E} \mid f_e/u_e > 0 \}.
	\]
\end{definition}

\begin{lemma}\label{lemma-min-flow-criterion}
	Ist $f$ ein $b$-Fluss, dessen Residualgraph keine gerichteten Kreise besitzt, die eine Flaschenhalskante als Rückwärtskante benutzen, so ist $f$ auslastungsminimal.
\end{lemma}

Um diese Aussage zu zeigen, benötigt man folgende Hilfsproposition:

\begin{proposition}\label{prop-difference-b-flows-stream}
	Für zwei $b$-Flüsse $f$, $f'$ mit $c(f) \geq c(f')$ ist $f'\Delta f := g\in \R_{\geq0}^{E^\leftrightarrow}$, definiert durch
	\begin{align*}
	g_{\overrightsmallarrow{e}} := \max\{ 0, f_e' - f_e \} \text{~~~und~~~}
	g_{\overleftsmallarrow{e}} := \max\{ 0, f_e - f_e' \} \text{~~~für $e\in E$},
	\end{align*}
	eine Strömung auf $G^\leftrightarrow$, die auf $E^\leftrightarrow \setminus E_f$ verschwindet.
\end{proposition}
\begin{proof}
	Eine Strömung ist eine Kantenbewertung, die in jedem Knoten Flusserhaltung erhält.
	Man zeige also $g(\delta^+_{G^\leftrightarrow}(v)) - g(\delta^-_{G^\leftrightarrow}(v)) = 0$ für alle Knoten $v\in V$.
	Durch Fallunterscheidung erkenne man $g_{\overrightsmallarrow{e}} - g_{\overleftsmallarrow{e}} = f_e' - f_e$ für alle $e\in E$ und man folgere:
	\begin{align*}
	g(\delta^+_{G^\leftrightarrow}(v)) - g(\delta^-_{G^\leftrightarrow}(v))
	&= \left( \sum_{e\in\delta^+_G(v)} g_{\overrightsmallarrow{e}} +  \sum_{e\in\delta^-_G(v)} g_{\overleftsmallarrow{e}} \right)
	- \left(\sum_{e\in\delta^-_G(v)} g_{\overrightsmallarrow{e}} + \sum_{e\in\delta^+_G(v)} g_{\overleftsmallarrow{e}} \right) \\
	&= \sum_{e\in\delta^+_G(v)} (g_{\overrightsmallarrow{e}} - g_{\overleftsmallarrow{e}}) - \sum_{e\in\delta^-_G(v)} (g_{\overrightsmallarrow{e}} - g_{\overleftsmallarrow{e}})\\
	&= \sum_{e\in\delta^+_G(v)} (f_e' - f_e) - \sum_{e\in\delta^-_G(v)} (f_e' - f_e) = b_v - b_v = 0.
	\end{align*}
	Es bleibt also zu zeigen, dass $g$ auf $E^\leftrightarrow \setminus E_f$ verschwindet.
	Sei zunächst eine Vorwärtskante $\overrightsmallarrow{e}$ mit $f_{e}/u_{e} = c(f)$ gegeben.
	Dann gilt nach Voraussetzung auch $f_{e}/u_{e} = c(f) \geq c(f') \geq f'_{e}/u_{e}$, wodurch $g_{\overrightsmallarrow{e}}= 0$ folgt.
	Für eine Rückwärtskante $\overleftsmallarrow{e}$ mit $f_e = 0$ folgt $g_{\overleftsmallarrow{e}} = 0$ direkt.
\end{proof}

Insbesondere existiert für solche Strömungen $f'\Delta f$ eine Dekomposition in Zyklen.
Diese Eigenschaft wird im folgenden Beweis ausgenutzt:

\begin{proof}[Beweis von Lemma~\ref{lemma-min-flow-criterion}]
	Sei $f$ ein $b$-Fluss, dessen Residualgraph keine gerichteten Kreise mit einer Flaschenhalskante besitzt, und sei $e$ eine beliebige Flaschenhalskante.
	Angenommen, es existiere ein $b$-Fluss $f'$ mit geringerer Auslastung; es gilt also insbesondere $f'_e/u_e < f_e/u_e$.
	Die Strömung $g:= f'\Delta f$ besitzt nach Proposition~\ref{prop-difference-b-flows-stream} eine Dekomposition in Zyklen $g = \lambda_1 \cdot C_1 +\dots + \lambda_k \cdot C_k$ mit $\lambda_i > 0$ für $i\in[k]$.
	Da $g_{\overleftsmallarrow{e}}$ positiv ist, gibt es einen Zyklus $C_i$, der ${\overleftsmallarrow{e}}$ enthält.
	Da $g$ außerdem nur auf $E_f$ verläuft, enthält $E_f$ also den Zyklus $C_i$, der die Flaschenhalskante $e$ benutzt, was im Widerspruch zur Voraussetzung steht.
\end{proof}


\begin{lemma}\label{lemma-no-circle-in-res-graph-inclus-min}
	Sei $q^*$ die minimale Auslastung eines $b$-Flusses im Netzwerk $(V, E, u)$.
	Ist $E'\subseteq E$ inklusionsminimal mit der Eigenschaft, dass ein auslastungsminimaler $b$-Fluss $f$ mit Flaschenhalskanten $E'$ existiert, so enthält der Residualgraph $G_f$
	für solche Flüsse $f$ keine gerichteten Kreise, die eine Flaschenhalskante als Rückwärtskante benutzen.
\end{lemma}
\begin{proof}
	\newcommand{\VK}{\text{VK}}
	\newcommand{\RK}{\text{RK}}
	Angenommen, es existiere ein einfacher Kreis $C$, der eine Flaschenhalskante $e\in E$ als Rückwärtskante enthält.
	Es seien $C_\VK$ die Menge der Vorwärtskanten und $C_\RK$ die Menge der Rückwärtskanten in $C$.
	Man setze $\gamma_\VK := \min_{e\in C_\VK} q^*u_e - f_e > 0$ als die minimale Flussmenge, die man jeder Kante in $C_\VK$ zufügen müsste, sodass mindestens eine Kante darin mindestens Auslastung $q^*$ erhält.
	Weiter sei $\gamma_\RK := \min_{e^\leftarrow\in C_\RK} f_e > 0$ die minimale Flussmenge der Rückwärtskanten in $C$.
	Wählt man nun $0 < \gamma < \min\{ \gamma_\VK, \gamma_\RK  \}$, so erhält man durch Augmentierung von $f$ entlang $Q$ mit $\gamma$ einen $b$-Fluss $\tilde{f}$.
	Bezüglich $\tilde{f}$ haben dann sowohl alle Vorwärtskanten als auch alle Rückwärtskanten von $C$ eine geringere Auslastung als $q^*$.
	Demnach ist auch $\tilde{f}$ ein auslastungsminimaler $b$-Fluss, dessen Flaschenhalskanten, also Kanten mit Auslastung $q^*$, eine echte Teilmenge von $E'$ sind, da mindestens eine Rückwärtskante in $E'$ enthalten war.
	Dies ist jedoch ein Widerspruch zur Inklusionsminimalität von $E'$.
\end{proof}

Weiter lässt sich nun das folgende fundamentale Theorem zeigen, das den Zusammenhang zwischen der Auslastung von $b$-Flüssen und der von Schnitten darlegt:

\begin{theorem}\label{thm-strong-duality-sparsest-cut-min-flow}
	In einem gerichteten Netzwerk $(V, E, u)$ mit Balancevektor $b\in\R^V$ mit $\sum_{v\in V}b_v = 0$ ist die Auslastung eines $b$-Flusses $f$ mindestens so groß wie die Auslastung eines Schnittes $X$ mit ausgehenden Kanten; das heißt \[\max_{e\in E} \frac{f_e}{u_e} \geq \frac{b(X)}{u(\delta^+(X))}.\]
	Existiert ein $b$-Fluss und ist $E$ nichtleer, so ist die minimale Auslastung eines $b$-Flusses gerade die Auslastung eines dünnsten Schnittes; das bedeutet
	\[
	\min_{\text{$f$ $b$-Fluss}}~\max_{e\in E}\frac{f_e}{u_e} = \max_{\substack{X\subseteq V\\ \delta^+(X)\neq\emptyset}} ~ \frac{b(X)}{u(\delta^+(X))}.
	\]
\end{theorem}
\begin{proof}
	Sei zunächst $f$ ein $b$-Fluss und $X$ ein Schnitt mit $\delta^+(X)\neq\emptyset$.
	Sei $e^*$ eine Kante mit maximaler Auslastung, also eine Kante mit $f_{e^*}/u_{e^*}=\max_{e\in E} f_e / u_e$.
	Für $e^*$ gilt somit $0\geq f_e u_{e^*} - f_{e^*}u_e$ für alle $e\in E$ und man folgere
	\[
	0\geq \frac{\sum_{e\in \delta^+(X)}(f_e u_{e^*} - f_{e^*}u_e)}{u(\delta^+(X)) u_{e^*}} = \frac{f(\delta^+(X))}{u(\delta^+(X))} - \frac{f_{e^*}}{u_{e^*}} \geq \frac{b(X)}{u(\delta^+(X))} - \frac{f_{e^*}}{u_{e^*}},
	\]
	wobei in der letzten Ungleichung $b(X) = f(\delta^+(X)) - f(\delta^-(X))$ eingeht.
	
	Sei nun ein $b$-Fluss $f$ mit minimaler Auslastung $q^*:=\max_{e\in E} f_e/u_e$ gegeben.
	Gesucht ist nun ein Schnitt mit Auslastung $q^*$.
	Ist $b$ der Nullvektor, so ist $q^*=0$ und jeder Schnitt mit ausgehenden Kanten ist ein dünnster Schnitt.
	
	Sonst sei $f$ ein $b$-Fluss minimaler Auslastung $q^* > 0$ mit möglichst wenig maximal ausgelasteten Kanten.
	Sei $e^*=vw$ eine solche maximal ausgelastete Kante und sei $X\subseteq V$ die Menge aller Knoten, die im Residualgraph $G_f$ vom Knoten $v$ erreicht werden können.
	Eine ausgehende Kante $xy\in\delta^+(X)$ muss dann Auslastung $q^*$ haben, da $x$ von $v$ in $G_f$ erreichbar ist und daher $xy$ nicht als Vorwärtskante in $G_f$ erscheinen kann; sonst wäre $y$ auch in $X$.
	Die Auslastung einer eingehenden Kante $xy\in\delta^-(X)$ muss jedoch verschwinden, da $xy$ nicht als Rückwärtskante in $G_f$ auftaucht.
	Außerdem ist $e^*\in \delta^+(X)$, da sonst $w$ von $v$ aus in $G_f$ erreichbar wäre und mit $e^{*\leftarrow}$ einen Kreis in $G_f$ bilden würde, der eine Rückwärtskante maximaler Auslastung benutzt, welchen es nach Lemma~\ref{lemma-no-circle-in-res-graph-inclus-min} nicht geben kann.
	Also ist $\delta^+(X)\neq \emptyset$ und es gilt:
	\[
	\frac{b(X)}{u(\delta^+(X))} = \frac{f(\delta^+(X))}{u(\delta^+(X))} = \sum_{e\in\delta^+(X)} \frac{u_e}{u(\delta^+(X))} q^* = q^*.
	\]
\end{proof}

\begin{corollary}\label{cor-easy-characterization-sparsest-cut}
	Seien $f$ ein $b$-Fluss mit Auslastung $q$ und $X$ ein Schnitt mit $\delta^+(X)\neq \emptyset$ in einem Netzwerk $(V, E, u)$.
	Dann sind die beiden Aussagen äquivalent:
	\begin{enumerate}[label=(\roman*)]
		\item Es sind $X$ ein dünnster Schnitt und $f$ ein $b$-Fluss minimaler Auslastung.
		\item Die Auslastung bezüglich $f$ aller ausgehenden Kanten von $X$ beträgt $q$ und $f$ verschwindet auf allen eingehenden Kanten von $X$.
	\end{enumerate}
\end{corollary}
\begin{proof}
	Aufgrund der Flusserhaltung gilt $b(X) = f(\delta^+(X)) - f(\delta^-(X))$.
	In \[
	\frac{b(X)}{u(\delta^+(X))} \leq \frac{f(\delta^+(X))}{u(\delta^+(X))} \leq \frac{u(\delta^+(X)) q}{u(\delta^+(X))} = q
	\]
	gilt Gleichheit genau dann, wenn $f(\delta^-(X))=0$ und $f_{e}/u_{e} = q$ für alle $e\in\delta^+(X)$ gelten.
	Theorem~\ref{thm-strong-duality-sparsest-cut-min-flow} liefert nun die Behauptung.
\end{proof}

\todo{Insbesondere sind alle inklusionsminimalen dünnsten Schnitte -- gegeben eines minimalen $b$-Flusses -- durch die Methode Residualgraph berechenbar (in sehr geringer polynomieller Zeit, nämlich vermutlich sogar $O(n+m)$ -- Allerdings muss man hier bisschen aufpassen, findet man ausgehend von einer maximal ausgelasteten Kante eine weitere, so muss man (für inklusionsminimalität das Vorgehen von hier zurücksetzen)}

\todo{Removing circles as naive approach of finding minimal b flow does not terminate.}
\chapter{Schmale Flüsse mit Zurücksetzen}\label{sec-thin-flows}
\newcommand*{\PlH}{\makebox[1ex]{\textbf{$\cdot$}}}

\section{Definition und Eigenschaften}

Der Abschnitt beginnt mit der Einführung einer neuen Klasse statischer Flüsse:

\begin{definition}[$s$-Fluss]
	Ein $b$-Fluss $f$ in einem Graphen $(V, E)$ heißt \emph{$s$-Fluss} für ein $s\in V$, falls $b_v\leq 0$ für alle $v\in V\setminus \{ s \}$ gilt.
	Dabei bezeichnet $b_s$ den \emph{Wert von $f$}. 
\end{definition}

\begin{definition}[Netzwerk mit Zurücksetzen]
	Ein \emph{Netzwerk mit Zurücksetzen auf $E_1$ und Ursprung $s$} ist ein Tupel $(V, E, u, s, E_1)$, wobei $(V, E, u)$ ein azyklisches Netzwerk ist, in dem der Knoten $s\in V$ jeden anderen Knoten erreicht, und $E_1\subseteq E$ die Menge der \emph{zurücksetzenden Kanten} ist.
\end{definition}

\begin{definition}[Auslastung]
	Sei ein $s$-Fluss $f$ in einem Netzwerk mit Zurücksetzen auf $E_1$ und Ursprung $s$ gegeben.
	Dann ist die \emph{Auslastung einer Kante} gegeben durch $f_e/u_e$.
	Die \emph{Auslastungsübertragung einer Kante $vw$} ist gegeben durch \[ \rho_{vw}(l_v, f_{vw}) := \begin{cases}
		\max\{ l_v, f_{vw} / u_{vw} \}, & \text{falls $vw\notin E_1$,}\\
		f_{vw} / u_{vw}, & \text{falls $vw\in E_1$.}
	\end{cases}
	\]
	Die \emph{Auslastung eines $s$-$v$-Pfades $P=(e_1,\dots,e_k)$} ist dann gegeben durch die Verkettung $(\rho_{e_k}(\PlH, f_{e_k})\circ \dots \circ \rho_{e_1}(\PlH, f_{e_1}))(0)$.
	Die zu $f$ \emph{zugehörige Knotenauslastung $l\in\R_{\geq 0}^V$} sei dann für einen Knoten $v\in V$ festgelegt durch die minimale Auslastung eines $s$-$v$-Pfades.
\end{definition}

Betrachtet man einen $s$-$v$-Pfad $P=(e_1, \dots, e_k)$, der Kanten aus $E_1$ enthält, so ist die Auslastung von $P$ gerade $\max_{j=i}^k f_{e_j}/u_{e_j}$, wobei $e_i$ die letzte Kante des Pfades in $E_1$ ist.
Daher nennt man die Kanten in $E_1$ die zurücksetzenden Kanten, da sie die Auslastung eines vorangegangen Pfades auf ihre eigene zurücksetzen.

\begin{proposition}\label{prop-congestion-labels-dijkstra}
	Die Knotenauslastungen eines $s$-Flusses $f$ sind gegeben durch die eindeutige Lösung $(\tilde{l}_v)_{v\in V}$ des Gleichungssystems
	\[
	\tilde{l}_w = \begin{cases}
		0, & \text{falls $w=s$,}\\
		\min_{vw\in \delta^-(w)} \rho_{vw}(\tilde{l}_v, f_{vw}), & \text{sonst.}
	\end{cases}
	\]
\end{proposition}
\begin{proof}
	Die Existenz einer Lösung folgt aus der Azyklizität des Netzwerks und wegen der Erreichbarkeit jedes Knotens von $s$ aus.
	
	Sei also $\tilde{l}\in\R_{\geq 0}^V$ die Lösung des Gleichungssystems.
	Der Teilgraph $(V, E')$ mit \[
	E':= \left\{ vw\in E ~\middle\vert~ \rho_{vw}(\tilde{l}_v, f_{vw}) = \min_{uw\in\delta^-(w)}  \rho_{uw}(\tilde{l}_u, f_{uw}) \right\}
	\]
	ebenfalls azyklisch und jeder Knoten ist von $s$ aus erreichbar.
	Man zeige $l_w = \tilde{l}_w$ durch eine Induktion über die Distanz von $w$ zu $s$ bezüglich der Anzahl an Kanten.
	Für $w=s$ gilt offenbar $l_s = 0$.
	Sei nun ein $s$-$w$-Pfad $P$ gegeben, seien $vw$ die letzte Kante und $Q$ das restliche Anfangsstück dieses Pfades.
	Nach Induktionsvoraussetzung ist die Auslastung von $Q$ mindestens $\tilde{l}_v$.
	Daher ist die Auslastung von $P$ mindestens $\rho_{vw}(\tilde{l}_v, f_{vw})$ aufgrund der Monotonie von $\rho_{vw}(\PlH, f_{vw})$, wodurch $l_v \geq \tilde{l}_v$ folgt.
	Außerdem hat ein $s$-$v$-Pfad $P$, der in $(V, E')$ verläuft, hat Auslastung $\tilde{l}_v$.
	Da solch ein Pfad existiert, gilt also $l_v \leq \tilde{l}_v$.
\end{proof}

\newcommand{\problemThinFlow}{\textsc{ThinFlow}}
\begin{definition}[Schmaler Fluss]
	Seien ein $s$-Fluss $f$ in einem Netzwerk mit Zurück\-setzen auf $E_1$ und Ursprung $s$ sowie die durch $f$ induzierten Knotenauslastungen $l$ gegeben.
	Der Fluss $f$ heißt \emph{schmaler Fluss mit Zurücksetzen auf $E_1$}, falls die Auslastung jedes $s$-$v$-Pfades mit positivem Fluss $l_v$ beträgt.
	
	Dabei nennt man das Tupel $(V, E, u, s, b, E_1)$ eine \emph{Instanz des $\problemThinFlow$ Problems}.
\end{definition}

Insbesondere ist bei einem schmalen Fluss die Knotenauslastung $l_v$ eines Knotens $v$ gerade die Auslastung eines jeden $s$-$v$-Pfades mit positivem Fluss und $0$, falls kein solcher existiert.

\begin{remark}
	Ronald Koch betrachtet in~\cite{Koch2012} allgemeinere schmale Flüsse: Dort ist das zugrundeliegende Netzwerk nicht als azyklisch vorausgesetzt.
	Diese Einschränkung wird hier im Sinne von~\cite{Cominetti2015} beibehalten.
\end{remark}

\begin{figure}
	\centering
	\newcommand{\newnode}[4]{\node[lul] (#1) at #2 {#3 \\[0.5em] #4};}
	\begin{subfigure}{\textwidth}
		\begin{tikzpicture}[lul/.style={draw,
			ellipse,
			align=center,
			inner sep=0pt,
			outer sep=4pt,
			text width=7mm,
			minimum height=1.5cm
		}]
		
		\newnode{0}{(0,2)}{$a$}{$32$}
		\newnode{1}{(3,2)}{$b$}{$0$}
		\newnode{2}{(5,0)}{$c$}{$-2$}
		\newnode{3}{(7,2)}{$d$}{$-12$}
		\newnode{4}{(10,3)}{$e$}{$-2$}
		\newnode{5}{(10,1)}{$f$}{$0$}
		\newnode{6}{(13,2)}{$g$}{$-16$}
		
		\begin{scope}[-Latex]
		\path [-Latex] (0) edge node[above] {$4$} (1);
		\path [-Latex] (0) edge[bend right] node[above] {$4$} (2);
		\path [-Latex] (0) edge[bend left] node[above] {$4$} (4);
		\path [-Latex] (1) edge node[right] {$2$} (2);
		\path [-Latex] (1) edge node[above] {$2$} (3);
		\path [-Latex] (2) edge node[left] {$2$} (3);
		\path [-Latex] (3) edge node[above] {$1$} (4);
		\path [-Latex] (3) edge node[above] {$2$} (5);
		\path [-Latex] (4) edge node[above] {$1$} (6);
		\path [-Latex] (5) edge node[above] {$1$} (6);
		\end{scope}
		\end{tikzpicture}
		\subcaption{Der Ursprungsgraph. Die Kanten $e$ sind mit Kapazitäten $u_e$ und die Knoten $v$ mit einem Bezeichner $v$ und einer Balance $b_v$ beschriftet.}
	\end{subfigure}
	\par\bigskip
	\begin{subfigure}{\textwidth}
		\begin{tikzpicture}[lul/.style={draw,
			ellipse,
			align=center,
			inner sep=0pt,
			outer sep=4pt,
			text width=7mm,
			minimum height=1.5cm
		}]
		
		\newnode{0}{(0,2)}{$a$}{$0$}
		\newnode{1}{(3,2)}{$b$}{$2{,}75$}
		\newnode{2}{(5,0)}{$c$}{$2{,}75$}
		\newnode{3}{(7,2)}{$d$}{$5$}
		\newnode{4}{(10,3)}{$e$}{$2{,}5$}
		\newnode{5}{(10,1)}{$f$}{$5$}
		\newnode{6}{(13,2)}{$g$}{$8$}
		
		\begin{scope}[-Latex]
		\path [-Latex] (0) edge node[above] {$11/4$} (1);
		\path [-Latex] (0) edge[bend right] node[above] {$11/4$} (2);
		\path [-Latex] (0) edge[bend left] node[above] {$10/4$} (4);
		\path [-Latex] (1) edge node[right] {$1/2$} (2);
		\path [-Latex] (1) edge node[above] {$10/2$} (3);
		\path [-Latex] (2) edge node[left] {$10/2$} (3);
		\path [-Latex] (3) edge node[above] {$0/1$} (4);
		\path [-Latex] (3) edge node[above] {$8/2$} (5);
		\path [-Latex] (4) edge node[above] {$8/1$} (6);
		\path [-Latex] (5) edge node[above] {$8/1$} (6);
		\end{scope}
		\end{tikzpicture}
		\subcaption{Ein schmaler Fluss $f$ ohne Zurücksetzen.}
		\label{subfig-thin-flow-without-resetting}
	\end{subfigure}
	\par\bigskip
	\begin{subfigure}{\textwidth}
		\begin{tikzpicture}[lul/.style={draw,
			ellipse,
			align=center,
			inner sep=0pt,
			outer sep=4pt,
			text width=7mm,
			minimum height=1.5cm
		},
		scale=1]
		
		\newnode{0}{(0,2)}{$a$}{$0$}
		\newnode{1}{(3,2)}{$b$}{$3$}
		\newnode{2}{(5,0)}{$c$}{$3$}
		\newnode{3}{(7,2)}{$d$}{$5{,}5$}
		\newnode{4}{(10,3)}{$e$}{$2$}
		\newnode{5}{(10,1)}{$f$}{$5{,}5$}
		\newnode{6}{(13,2)}{$g$}{$8$}
		
		\begin{scope}[-Latex]
		\path [-Latex] (0) edge node[above] {$12/4$} (1);
		\path [-Latex] (0) edge[bend right] node[above] {$12/4$} (2);
		\path [-Latex] (0) edge[bend left] node[above] {$8/4$} (4);
		\path [-Latex] (1) edge node[right] {$1/2$} (2);
		\path [-Latex] (1) edge node[above] {$11/2$} (3);
		\path [-Latex] (2) edge node[left] {$11/2$} (3);
		\path [Square-Latex] (3) edge node[above] {$2/1$} (4);
		\path [-Latex] (3) edge node[above] {$8/2$} (5);
		\path [-Latex] (4) edge node[above] {$8/1$} (6);
		\path [-Latex] (5) edge node[above] {$8/1$} (6);
		\end{scope}
		\end{tikzpicture}
		\subcaption{Ein schmaler Fluss mit Zurücksetzen auf $(d,e)$.}
		\label{subfig-thin-flow-with-resetting}
	\end{subfigure}
\caption{Ein schmaler Fluss einmal ohne Zurücksetzen und einmal mit Zurücksetzen auf $(d,e)$.
	In~\ref{subfig-thin-flow-without-resetting} und~\ref{subfig-thin-flow-with-resetting} hat die Kantenbeschriftung die Form $f_e/u_e$ und in den Knoten steht die Knotenauslastung.}
\end{figure}

\begin{lemma}\label{lemma-thin-flow-t-def}
	Ein $s$-Fluss $f$ in einem Netzwerk mit Zurücksetzen auf $E_1$ und Ursprung $s$ ist genau dann ein schmaler Fluss mit Zurücksetzen auf $E_1$, wenn eine Knotenbewertung $l\in\R_{\geq 0}^V$ existiert, die die folgenden Bedingungen erfüllt:
	\begin{enumerate}[label=(T\arabic*)]
		\item\label{def-thin-flow-source} $l_s = 0$,
		\item\label{def-thin-flow-x-zero} $l_w \leq l_v$, \tabto{4.5cm} für $vw\in E \setminus E_1$ mit $f_{vw}=0$,
		\item\label{def-thin-flow-x-positive} $l_w = \max\{ l_v, f_{vw} / u_{vw} \}$,  \tabto{4.5cm} für $vw\in E\setminus E_1$ mit $f_{vw} > 0$,
		\item\label{def-thin-flow-resetting-edge} $l_w = f_{vw} / u_{vw}$,  \tabto{4.5cm} für $vw\in E_1$,
		\item\label{def-thin-flow-no-resetting-edge} $l_w \geq \min_{vw\in \delta^-(w)} l_v$, \tabto{4.5cm} falls $\delta^-(w)\cap E_1 = \emptyset$.
	\end{enumerate}
	Diese Knotenbewertung stimmt dann mit der Knotenauslastung von $f$ überein.
	Außerdem gelten die Bedingungen~\ref{def-thin-flow-source}, \ref{def-thin-flow-x-zero} und~\ref{def-thin-flow-no-resetting-edge} bereits für die Knotenauslastungen von $f$, falls $f$ nur ein $s$-Fluss ist.
\end{lemma}
\begin{proof}
	Sei $f$ ein schmaler Fluss mit Zurücksetzen auf $E_1$ und sei $l\in\R_{\geq 0}^V$ die Knotenauslastung von $f$.
	Man zeige, dass $l$ die Bedingungen~\ref{def-thin-flow-source}-\ref{def-thin-flow-no-resetting-edge} erfüllt, und benutze dabei die Darstellung aus Proposition~\ref{prop-congestion-labels-dijkstra}.
	Wegen $l_s = 0$ gilt bereits~\ref{def-thin-flow-source}.
	Für eine Kante $vw\in E\setminus E_1$ mit $f_{vw}=0$ gilt $l_w\leq \rho_{vw}(l_v, f_{vw}) = l_v$, wodurch auch~\ref{def-thin-flow-x-zero} folgt.
	Ist $vw\in E$ mit $f_{vw} > 0$, so existiert ein $s$-$w$-Pfad mit positivem Fluss, der die Kante $vw$ benutzt, sodass die Auslastung dieses Pfades $l_w=\rho_{vw}(l_v, f_{vw})$ ist, da der $s$-$v$-Teilpfad die Auslastung $l_v$ besitzt.
	Für $vw\notin E_1$ bedeutet das gerade $l_w = \max\{ l_v, f_{vw}/u_{vw} \}$ also~\ref{def-thin-flow-x-positive}, und für $vw\in E_1$ zeigt das $l_w = f_{vw} / u_{vw}$.
	Für~\ref{def-thin-flow-resetting-edge} bleibt der Fall $f_{vw} = 0$ zu prüfen:
	Dann ist aber $l_w = \min_{uw\in\delta^-(w)} \rho_{uw}(l_u, f_{uw}) = f_{vw} / u_{vw} = 0$.
	Zuletzt betrachte man den Fall, dass $w$ keine eingehende zurücksetzende Kante hat.
	Dann folgt aber bereits $l_w = \min_{vw\in \delta^-(w)} \max\{ l_v, f_{vw} / u_{vw} \} \geq \min_{vw\in\delta^-(w)} l_v$, was auch~\ref{def-thin-flow-no-resetting-edge} impliziert.
	
	Sei nun umgekehrt $l\in\R_{\geq 0}^V$ eine Knotenbewertung, die~\ref{def-thin-flow-source}-\ref{def-thin-flow-no-resetting-edge} erfüllt.
	Man verwende eine Induktion über die Distanz eines Knotens $w$ zu $s$ bezüglich der Kantenzahl, um zu zeigen, dass $l_w$ die Knotenauslastung von  $f$ ist.
	Für $w=s$ gilt $l_s=0$ bereits nach~\ref{def-thin-flow-source}.
	Für $w\neq s$ ist $l_w$ nach~\ref{def-thin-flow-x-zero},~\ref{def-thin-flow-x-positive} und~\ref{def-thin-flow-resetting-edge} eine untere Schranke an $\rho_{vw}(l_v, f_{vw})$ für $vw\in\delta^-(w)$.
	Es bleibt zu zeigen, dass für eine eingehende Kante $vw$ der Wert $\rho_{vw}(l_v, f_{vw})$ auch $l_w$ annimmt:
	Falls eine eingehende Kante $vw\in E_1$ oder eine eingehende Kante $vw$ mit $f_{vw} > 0$ existiert, ist $\rho_{vw}(l_v, f_{vw}) = l_w$ nach~\ref{def-thin-flow-resetting-edge} und~\ref{def-thin-flow-x-positive}.
	Sonst ist $l_w\geq \min_{vw\in \delta^-(w)} l_v = \min_{vw\in \delta^-(w)} \rho_{vw}(l_v', x_{vw}')$ nach~\ref{def-thin-flow-no-resetting-edge}, was die Behauptung zeigt.
	Um zu sehen, dass $f$ nun ein schmaler Fluss mit Zurücksetzen auf $E_1$ ist, genügen die Bedingungen~\ref{def-thin-flow-source}, \ref{def-thin-flow-x-positive} und~\ref{def-thin-flow-resetting-edge}, da dadurch jeder $s$-$v$-Teilpfad eines Pfades mit positivem Fluss gerade Auslastung $l_v$ hat.
\end{proof}

\begin{remark}\label{remark-thin-flow}
	Hier ist, wie in~\cite[Definition~4]{Cominetti2011}, im Vergleich zu~\cite[Definition~6]{Koch2011} die Bedingung~\ref{def-thin-flow-no-resetting-edge} zusätzlich eingeführt worden.
	Diese ist nötig, um im Beweis von Theorem~\ref{thm-alpha-extension-is-nash-flow} zu zeigen, dass die erweiterten Ankunftszeiten tatsächlich mit den angegebenen übereinstimmen.
	Ohne diese Bedingung würde dies nämlich nicht gelten, wie man in Abbildung~\ref{figure-labels} erkennen kann.
\end{remark}
\todo{
	Introduce network inflow with normalized thin-flows
	\begin{definition}[Schmaler Fluss]\label{def-thin-flow}
		Der Fluss $x'$ heißt \emph{normiert}, falls $F=d$ gilt.
		Ist $E_1$ nichtleer, so heißt $x'$ \emph{schmaler Fluss mit Zurücksetzen auf $E_1$}, sonst heißt $x'$ \emph{schmaler Fluss ohne Zurücksetzen}.
	\end{definition}
}

\begin{lemma}\label{lemma-equivalent-thin-flow}
	Ein $s$-Fluss $f$ ist genau dann ein schmaler Fluss mit Zurück\-setzen auf $E_1$, wenn $f_{vw}= 0$ für alle $vw\in E$ mit $l_w < \rho_{vw}(l_v, f_{vw})$ gilt.
	Dabei sei $l\in\R_{\geq 0}^V$ die Knotenauslastung von $f$.
\end{lemma}
\begin{proof}
	Angenommen, $f$ sei ein schmaler Fluss mit Zurücksetzen auf $E_1$.
	Für eine Kante $vw\in E$ mit $l_w < \rho_{vw}(l_v, f_{vw})$ ist $vw\notin E_1$ und $f_{vw}=0$ nach~\ref{def-thin-flow-resetting-edge} und~\ref{def-thin-flow-x-positive}.

	Nun nehme man an, es gelte $f_{vw}=0$ für alle $vw\in E$ mit $l_w < \rho_{vw}(l_v, f_{vw})$, und man zeige die Eigenschaften~\ref{def-thin-flow-x-positive} und \ref{def-thin-flow-resetting-edge}.
	Der Rest folgt dann mit Lemma~\ref{lemma-thin-flow-t-def}.
	Ist $vw\notin E_1$ mit $f_{vw}>0$, so gilt $l_w = \rho_{vw}(l_v, f_{vw})$, wodurch Bedingung~\ref{def-thin-flow-x-positive} folgt.
	Für~\ref{def-thin-flow-resetting-edge} betrachte man zunächst $vw\in E_1$ mit $f_{vw}>0$: Hier folgt wieder $l_w = \rho_{vw}(l_v, f_{vw})$.
	Falls $f_{vw}=0$ gilt, ist $l_w=\min_{\tilde{v}w\in \delta^-(w)} \rho_{\tilde{v}w}(l_{\tilde{v}}, f_{\tilde{v}w}) = 0$.
\end{proof}

\todo{Vlt im folgenden Korollar auch/oder Gleichheit zu $\max f_e/u_e$ zeigen}
\begin{corollary}\label{cor-thin-flows-max-node-label-equals-max-edge-label}
	Für einen schmalen $s$-Fluss $f$ in einem Netzwerk mit nichtleerer Kantenmenge und zugehöriger Knotenauslastung $l$ gilt
	\[
		\max_{vw\in E} \rho_{vw}(l_v, f_{vw}) = \max_{v\in V} l_v
	\]
\end{corollary}
\begin{proof}
	Für $w\neq s$ ist $l_w = \min_{vw\in\delta^-(w)} \rho_{vw}(l_v, x_{vw})$.
	Daher gilt bereits \glqq$\geq$\grqq.
	Angenommen es gebe eine Kante $vw\in E$ mit $\rho_{vw}(l_v, f_{vw}) > \max_{u\in V} l_u$
	Dann ist aber bereits $f_{vw}/u_{vw} > \max_{u\in V} l_u \geq 0$, womit $l_w \geq \rho_{vw}(l_v, f_{vw})$ nach Lemma~\ref{lemma-equivalent-thin-flow} gelten müsste.
\end{proof}

\begin{lemma}[Eindeutigkeit der Knotenauslastung]\label{lemma-node-congestion-unique}
	Seien ein azyklisches Netzwerk mit Ursprung $s$ und Versorgungsrate $d\in\R_{\geq 0}$ sowie $E_1\subseteq E$ gegeben.
	Dann sind die Knotenauslastungen aller normierter schmaler Flüsse mit Zurücksetzen auf $E_1$ identisch.
\end{lemma}
\begin{proof}
	Angenommen, es existieren zwei normierte schmale Flüsse $x'$ und $y'$ mit Zurück\-setzen auf $E_1$ mit unterschiedlichen Knotenauslastungen $l'\neq h'$.
	Dann ist oBdA. die Menge $S:=\{ v\in V \mid l_v' > h_v' \}$ nichtleer.
	Da $x'$ und $y'$ beide $b$-Flüsse sind, gilt 
	\begin{equation}\label{proof-uniqueness-labels-eq}
	\sum_{e\in \delta^+(S)} x'_e - \sum_{e\in\delta^-(S)} x'_e = \sum_{v\in S} b_v = \sum_{e\in\delta^+(S)} y_e' - \sum_{e\in\delta^-(S)} y_e'.
	\end{equation}
	Für Kanten $vw\in \delta^+(S)$ gilt $x_e' \leq y_e'$, da für $x_e' > y_e'$ Lemma~\ref{lemma-equivalent-thin-flow} und $v\in S$ die Ungleichung $l_w' = \rho_{vw}(l_v', x_{vw}') > \rho_{vw}(h_v', y_{vw}')\geq h_w'$ implizieren würden, welche jedoch im Widerspruch zu $w\notin S$ steht.
	Ebenso gilt für Kanten $vw\in\delta^-(S)$ die Ungleichung $x_e' \geq y_e'$, weil sonst $l_w' \leq \rho_{vw}(l_v', x_{vw}') \leq \rho_{vw}(h_v', y_{vw}') = h_w'$ wegen $v\notin S$ gelten würde, was $w\in S$ widerspricht.
	Gleichung~\ref{proof-uniqueness-labels-eq} impliziert dann sogar $x_e = y_e$ für alle $e\in \delta(S):=\delta^-(S) \cup \delta^+(S)$.
	Für $vw\in \delta^-(S)$ ist dann $y_e'=x_e'=0$, da für $y_e'=x_e'>0$ wieder mit $v\notin S$ die Ungleichung $l_w' = \rho_{vw}(l_v, x_{vw}')\leq \rho_{vw}(h_v, y_{vw}')=h_w'$ der Bedingung $w\in S$ widerspricht.
	
	Aufgrund der Azyklizität existiert ein Knoten $w\in S$ mit $\delta^-(w)\subseteq \delta^-(S)$.
	Eingehende Kanten von $w$ sind daher nicht in $E_1$ enthalten, da für solch eine Kante $\rho_{vw}(l_v', x_e') = 0 = \rho_{vw}(l_v', y_e')$ gelten würde, sodass $l_w' = 0 = h_w'$ der Voraussetzung $w\in S$ widerspricht.
	Daher ist $l_w' = \min_{vw\in \delta^-(w)} l_v'$ und $h_w' = \min_{vw\in\delta^-(w)} h_v'$, was den Widerspruch $l_w' \leq h_w'$ impliziert.
\end{proof}

Zwar sind schmale Flüsse im Allgemeinen nicht eindeutig -- nichtmal ohne zurück\-setzende Kanten --, wie man leicht in Abbildung~\ref{figure-thin-flows-not-unique} erkennen kann, jedoch ist der Fluss auf einigen Kanten eindeutig:

\begin{figure}
	\centering
	\begin{subfigure}{.45\textwidth}
		\begin{tikzpicture}[scale=0.8]
		\node[draw, circle] (S) at (0,2) {$s$};
		\node[draw, circle] (U) at (2,2) {$u$};
		\node[draw, circle] (W) at (4,1) {$w$};
		\node[draw, circle] (V) at (4,3) {$v$};
		\node[draw, circle] (T) at (6,2) {$t$};
		
		\path [->] (S) edge node[above] {$2$} (U);
		\path [->] (U) edge node[above] {$1$} (V);
		\path [->] (U) edge node[below] {$1$} (W);
		\path [->] (V) edge node[above] {$1$} (T);
		\path [->] (W) edge node[below] {$1$} (T);
		\end{tikzpicture}
	\end{subfigure}
	\begin{subfigure}{.45\textwidth}
		\begin{tikzpicture}[scale=0.8]
		\node[draw, circle] (S) at (0,2) {$s$};
		\node[draw, circle] (U) at (2,2) {$u$};
		\node[draw, circle] (W) at (4,1) {$w$};
		\node[draw, circle] (V) at (4,3) {$v$};
		\node[draw, circle] (T) at (6,2) {$t$};
		
		\path [->] (S) edge node[above] {$2$} (U);
		\path [->] (U) edge node[above] {$2$} (V);
		\path [->] (U) edge node[below] {$0$} (W);
		\path [->] (V) edge node[above] {$2$} (T);
		\path [->] (W) edge node[below] {$0$} (T);
		\end{tikzpicture}
	\end{subfigure}
	\caption{Zwei verschiedene schmale Flüsse derselben Instanz ohne Zurück\-setzen. Auf den Kanten steht der Flusswert; die Kapazitäten sind allesamt $1$.}
	\label{figure-thin-flows-not-unique}
\end{figure}

\begin{corollary}[Eindeutigkeit schmaler Flüsse]
	 Für eine Instanz von~$\problemThinFlow$ sind alle Lösungen auf den zurück\-setzen\-den Kanten sowie auf Kanten $vw$ mit $l_v \neq l_w$ identisch.
	 Dabei ist $(l_v)_{v\in V}$ die eindeutige Knotenauslastung.
\end{corollary}
\begin{proof}
	Da nach Lemma~\ref{lemma-node-congestion-unique} die Knotenauslastung $(l_v)_{v\in V}$ für jeden schmalen Fluss $f$ eindeutig ist, gilt $f_{vw} = u_{vw} l_w$ für $vw\in E_1$ nach~\ref{def-thin-flow-resetting-edge}.
	Für $vw\notin E_1$ betrachte man zunächst den Fall $l_v < l_w$.
	Nach Definition gilt $l_w \leq \max \{ l_v, f_{vw}/u_{vw} \}$, sodass man $f_{vw}/u_{vw} > l_v \geq 0$ folgern kann.
	Nach Lemma~\ref{lemma-equivalent-thin-flow} gilt also $f_{vw} = l_w u_{vw}$.
	Im Fall $l_v > l_w$ muss $f_{vw} = 0$ gelten, da sonst $l_w=\max\{ l_v, f_e/u_e \}$ gelten würde.
\end{proof}

\begin{theorem}[Existenz schmaler Flüsse]\label{thm-existence-thin-flow}
	Für jede Instanz von $\problemThinFlow$ existiert ein schmaler Fluss, falls im zugehörigen Netzwerk ein $s$-Fluss existiert.
\end{theorem}

Um die Existenz zu beweisen, benötigen wir zunächst den Fixpunktsatz von Kakutani.
Dazu führen wir die folgenden Begriffe ein:

\begin{definition}[Korrespondenz, Fixpunkt]
	Eine \emph{Korrespondenz} von einer Menge $A$ in eine Menge $B$ ist eine Abbildung $f: A \to \mathcal{P}(B)\setminus \{ \emptyset \}$ von $A$ in die Potenzmenge von $B$ ohne die leere Menge.
	Sind $A$ und $B$ topologische Räume, so nennt man $f$ \emph{abgeschlossen}, wenn die zugehörige Relation $R_f := \{ (a,b) \in A\times B \mid b\in f(a)  \}$ in der Produkttopologie abgeschlossen ist.

	Ein \emph{Fixpunkt} einer Korrespondenz $f: X \to \mathcal{P}$ ist ein Punkt $x\in X$ mit $x\in f(x)$.
\end{definition}

Nun lautet der Fixpunktsatz von Kakutani (siehe~\cite{Heuser1991Fix}):

\begin{satz}[Fixpunktsatz von Kakutani]\label{satz-kakutani}
	Seien $C\subseteq E$ eine nichtleere, konvexe und kompakte Teilmenge eines normierten Raumes $E$ und $f: C \to \mathcal{P}(C)$ eine abgeschlossene, konvexwertige Korrespondenz.
	Dann besitzt $f$ einen Fixpunkt.
\end{satz}

\begin{proof}[Beweis von Theorem~\ref{thm-existence-thin-flow}]
	Man betrachte die Menge $C$ aller $b$-Flüsse als Teilmenge des metrischen Raums $\R^E$.
	Nach Voraussetzung ist diese nichtleer, konvex, da für $f, g\in C, \lambda \in [0,1]$ der Fluss $\lambda f + (1-\lambda)g$ wieder die Flusserhaltung erfüllt, und kompakt, da jeder Fluss in $C$ aufgrund der Azyklizität in $s$-$v_i$-Wege zerlegbar ist, wodurch jede Kante maximal Fluss $b_s$ besitzen kann, sodass $C$ eine beschränkte Teilmenge von $\subseteq [0, b_s]^E$ ist.
	Außerdem ist $C$ abgeschlossen, da es die Lösungsmenge eines linearen Gleichungssystem ist.
	Auf $C$ definiere man die Korrespondenz
	\[
	\Gamma: C\to \mathcal{P}(C)\setminus\{ \emptyset \}, ~ f\mapsto \{ g\in C \mid \forall vw\in E: l_w < \rho_{vw}(l_v, f_{vw}) \implies g_{vw} = 0 \}.
	\]
	Dabei sei $l\in\R_{\geq 0}^V$ die zu $f$ gehörige Knotenauslastung.
	Diese Korrespondenz ist wohldefiniert, da $\Gamma(f)$ für jedes $f\in C$ nichtleer ist:
	Der Knoten $s$ kann im Graphen $G':= (V, E')$ mit $E':= \{ vw\in E \mid \rho_{vw}(l_v, f_{vw})=l_w \}$ jeden Knoten erreichen, da $G'$ azyklisch ist und jeder Knoten $w\neq s$ mindestens eine eingehende Kante mit $l_w = \rho_{vw}(l_v, f_{vw})$ besitzt. 
	Daher existiert ein $s$-$v$-Weg $P_v$ in $G'$ für alle $v\in V$ und der Fluss $\sum_{v\in V\setminus\{ s\}} b_v \cdot P_v$ ist in $\Gamma(f)$ enthalten.
	
	Für jedes $f\in C$ ist $\Gamma(f)$ konvex: Ist für $g, h\in \Gamma(f), \lambda\in [0,1]$ und eine Kante $vw\in E$ die Bedingung $l_w < \rho_{vw}(l_w, f_{vw})$ erfüllt, so gilt auch $\lambda g_{vw} + (1-\lambda) h_{vw} = 0$, womit $\lambda g + (1-\lambda)h$ im Bild $\Gamma(f)$ liegt.
	Des Weiteren ist $\Gamma$ abgeschlossen: 
	Sei eine konvergente Folge $((f^{n}, g^{n}))_{n\in\N}$ in $R_\Gamma$ mit Grenzwert $(f, g)$ gegeben.
	Die zu $f$ bzw. $f^n$ gehörigen Knotenauslastungen seien gegeben durch $l$ bzw. $l^n$.
	Angenommen, es gelte $l_w < \rho_{vw}(l_v, f_{vw})$.
	Da die Zuordnung $x'\mapsto l'$ eines $s$-Flusses auf seine Knotenauslastung nach Lemma~\ref{prop-congestion-labels-dijkstra} stetig ist, existiert ein $N\in\N$ mit $l_w^n<\rho_{vw}(l_v^n, f_{vw}^n)$ für alle $n\geq N$.
	Damit ist auch $g_{vw} = \lim_{n\to\infty} g_{vw}^n = 0$.
	
	Daher existiert nach dem Fixpunktsatz von Kakutani~\ref{satz-kakutani} ein Fixpunkt von $\Gamma$, welcher nach Lemma~\ref{lemma-equivalent-thin-flow} ein schmaler Fluss mit Zurücksetzen auf $E_1$ ist.
\end{proof}

\section{Effiziente Berechnung schmaler Flüsse}

Die Existenz schmaler Flüsse mit Zurücksetzen aus Theorem~\ref{thm-existence-thin-flow} suggeriert bereits einen exponentiellen Algorithmus, um einen solchen Fluss mit Wert $d$ zu berechnen:
Man rate die Menge $E'\subseteq E$ aller Kanten $vw$, die $l_w' = \rho_{vw}(l_v', x_{vw}')$ erfüllen, sowie für alle $vw\in E'\setminus E_1$ eine Binärzahl $\sigma_{vw}\in\{0,1\}$, die $\max\{ l_v', x_{vw}/u_{vw}\} = \sigma_{vw} l_v' + (1-\sigma_{vw}) x_{vw}/u_{vw}$ erfüllt.
Dann suche man einen zulässigen Punkt in folgendem Polyeder:
\begin{align*}
&	(x', l') \in \R_{\geq 0}^{E'\times V} \\
	\text{u.d.N.}\quad &
	x'(\delta^+(v)) - x'(\delta^-(v)) =
		\begin{cases}
			d, & \text{falls $v=s$},\\
			0, & \text{sonst,} 
		\end{cases} & \text{für $v\in V\setminus\{t \}$,} \\
	& l'_s = 0, \\
	& \begin{array}{@{}l}
		l_w' = \sigma_{vw} l_v' + (1-\sigma_{vw}) x'_{vw}/u_{vw}\\
		\sigma_{vw} l_v' + (1-\sigma_{vw}) x'_{vw}/u_{vw} \geq x'_{vw}/u_{vw}\\
		\sigma_{vw} l_v' + (1-\sigma_{vw}) x'_{vw}/u_{vw} \geq l_v'
	\end{array}
	& \text{für $vw\in E'\setminus E_1$},\\
	& l_w' = x'_{vw}/u_{vw} & \text{für $vw\in E_1$.}
\end{align*}
\todo{Genauerer Beweis}

\newcommand{\TFNP}{\mathbf{TFNP}}

Existiert solch ein Punkt $(x', l')$, so ist $x'$ nach Lemma~\ref{lemma-equivalent-thin-flow} ein schmaler Fluss mit Zurücksetzen auf $E_1$ mit zugehöriger Knotenbewertung $l'$.
Außerdem muss aufgrund der Existenz eines schmalen Flusses mindestens eine Wahl von $E'$ und $\sigma$ existieren, sodass der zugehörige Polyeder nichtleer ist.
\todo{Habe ich hier nicht das Problem, dass ich das nur für $\Q$ in polynomieller Zeit berechnen kann?}
Die Zulässigkeit eines solchen Polyeders kann in polynomieller Zeit beispielsweise durch die Ellipsoidmethode ermittelt werden. 
Daher liegt das Problem, zu einem Netzwerk $(V, E, u, s, t)$ einen schmalen $d$-Fluss mit Zurücksetzen auf $E_1\subseteq E$ zu finden, in der Komplexitätsklasse $\TFNP$:

\begin{definition}[Komplexitätsklasse $\TFNP$]
	Eine Relation $P(x,y)$ liegt in der Komplexitätsklasse $\TFNP$ genau dann, wenn es einen deterministischen, polynomiellen Algorithmus gibt, der entscheidet, ob $P(x,y)$ für zwei Kandidaten $x$ und $y$ gilt, und wenn für alle $x$ ein Kandidat $y$ existiert, dessen Kodierung polynomielle Größe in der Kodierungslänge von $x$ besitzt, sodass $P(x, y)$ gilt.
\end{definition}

In diesem Abschnitt wird ein polynomieller Algorithmus zur Berechnung von schmalen Flüssen ohne Zurücksetzen vorgestellt.

\begin{lemma}\label{lemma-thin-flows-without-resetting-have-minimal-congestion}
	Sei ein schmaler $b$-Fluss $f$ ohne Zurücksetzen mit zugehöriger Knotenauslastung $l$ gegeben.
	Dann ist $f$ ein $b$-Fluss minimaler Auslastung $q$ und für $E\neq\emptyset$ ist $X:=\{ v\in V \mid l_v < q \}$ ein dünnster Schnitt.
\end{lemma}
\begin{proof}
	Ist $b$ der Nullvektor -- wie im Fall $E=\emptyset$ --, ist $f$ der Nullfluss, da das Netzwerk azyklisch ist.
	
	Sei also $q>0$ die Auslastung von $f$ im Falle von $b\neq 0$.
	Nach Korollar~\ref{cor-thin-flows-max-node-label-equals-max-edge-label} genügt es zu
	zeigen, dass die Auslastung jeder ausgehenden Kante von $X$ bezüglich $f$ gerade $q$ beträgt und auf eingehenden Kanten von $X$ kein Fluss fließt.
	Wegen $l_s=0$ ist $s$ im Schnitt $X$ enthalten.
	Sei $vw\in\delta^+(X)$ eine ausgehende Kante.
	Dann gilt $l_v < l_w = q $ und nach Proposition~\ref{prop-congestion-labels-dijkstra} auch $q = l_w \leq \rho_{vw}(l_v, f_{vw}) = f_{vw} / u_{vw}$.
	Da $q$ eine obere Schranke an $f_{vw}/u_{vw}$ ist, gilt also $f_{vw}/u_{vw} = q$.
	Für eine eingehende Kante $vw\in\delta^-(X)$ gilt $\rho_{vw}(l_v, f_{vw})\geq l_v > l_w$ und somit muss nach Lemma~\ref{lemma-equivalent-thin-flow} der Fluss $f$ auf der Kante $vw$ verschwinden.
\end{proof}

\begin{definition}
	Seien $\mathcal{I}:=(V, E, u, b)$ eine $b$-Fluss-Instanz und $X\subseteq V$ ein dünnster Schnitt mit Auslastung $q$.
	Dann heißt $\mathcal{I}[X] := (X, E[X], u[X], b[X])$ die \emph{durch $X$ induzierte Teilinstanz von $\mathcal{I}$},
	wobei $E[X] := \{ vw \in E \mid v, w \in X \}$ als die Menge der Kanten zwischen Knoten in $X$, $u[X]$ als die Einschränkung von $u$ auf $E[X]$ und $b[X]$ durch \[
	b[X]_v := b_v - u(\delta^+_E(v)\cap\delta^+_E(X)) q \quad \text{für $v\in X$}
\] definiert ist.
\end{definition}

Man bemerke, dass $b[X]_v \leq b_v$ für $v\in X$ erfüllt und $b[X]$ ein zulässiger Balancevektor ist: \[
	\sum_{v\in X} b[X]_v = b(X) - u(\delta^+_E(X)) q = 0
\]

\begin{proposition}\label{prop-restricted-minimal-flow-is-b-flow-on-induced-instance}
	Ist $f$ ein $b$-Fluss minimaler Auslastung einer $b$-Fluss-Instanz $\mathcal{I}$, so ist die Einschränkung $f':=f_{\mid E[X]}$ ein $b[X]$-Fluss der durch $X$ induzierten Instanz $(X, E[X], u[X], b[X])$ für jeden dünnsten Schnitt $X$.
	Ist $f$ sogar ein $s$-Fluss mit Knotenauslastung $l$ und $b_s > 0$, so stimmt die Einschränkung von $l$ auf $X$ mit der Knotenauslastung $l'$ vom $s$-Fluss $f'$ überein.
\end{proposition}
\begin{proof}
	Nach Korollar~\ref{cor-easy-characterization-sparsest-cut} gelten $f_{e}/u_{e}=q$ für $e\in\delta^+(X)$ und $f_{e} = 0$ für $e\in\delta^-(X)$.
	Daher kann man zeigen, dass $f'$ ein $b[X]$-Fluss ist, denn für alle $v\in X$ gilt:
	\begin{align*}
	f'(\delta^+_{E[X]}(v)) - f'(\delta^-_{E[X]}) &= f(\delta^+_E(v)) - f(\delta^-_E(v)) - f(\delta^+_{E[X]}(v)\cap\delta^+_{E[X]}(X)) \\
	&= b_v - u(\delta^+_{E[X]}(v)\cap\delta^+_{E[X]}(X))q = b[X]_v.
	\end{align*}
	Sei nun $f$ sogar ein $s$-Fluss mit $b_s > 0$, wodurch $s$ in $X$ enthalten ist.
	Sei $v\in X$ gegeben.
	Da jeder $s$-$v$-Pfad in $(X, E')$ auch in $(V, E)$ enthalten ist, gilt bereits $l'_v\leq l_v$.
	Außerdem liegt jeder $s$-$v$-Pfad in $(V, E)$ mit positivem Fluss bezüglich $f$ auch komplett in $(X, E')$, da $f$ auf $\delta^-_E(X)$ verschwindet.
	Daraus folgt $l'_v \geq l_v$.
\end{proof}

Daher kann man bereits erkennen, dass ein schmaler Fluss ohne Zurücksetzen eingeschränkt auf eine durch einen dünnsten Schnitt induzierte Teilinstanz wieder ein schmaler Fluss ist.
Das Ziel ist es nun, aus einem schmalen Fluss der Teilinstanz einen schmalen Fluss auf der ursprünglichen Instanz zu konstruieren.

\begin{lemma}\label{lemma-flow-minimal-congestion-sparsest-cut-then-outside-X-congestion-q}
	Seien ein $s$-Fluss $f$ minimaler Auslastung $q^*$ mit zugehöriger Knotenauslastung $l$ und ein dünnster Schnitt $X$ in einem Netzwerk ohne zurücksetzende Kanten gegeben.
	Für alle $v\in V\setminus X$ hat jeder $s$-$v$-Pfad $P$ Auslastung $q^*=l_v$.
\end{lemma}
\begin{proof}
	Ist $b$ der Nullvektor, so ist $f$ der Nullfluss und $l_v = 0$ für alle $v\in V$.
	Sonst ist $s$ in $X$ enthalten und für alle $v\in V\setminus X$ enthält jeder $s$-$v$-Pfad mindestens eine Kante, die $X$ verlässt und nach Korollar~\ref{cor-easy-characterization-sparsest-cut} Auslastung $q^*$ besitzt.
	Da das Netz keine zurücksetzenden Kanten hat, beträgt die Auslastung also $\max_{e\in P} f_e / u_e = q^*$.
\end{proof}

\begin{corollary}\label{cor-thin-flow-alg-correct}
	Seien ein $s$-Fluss $f$ minimaler Auslastung sowie ein dünnster Schnitt $X$ auf einem Netzwerk ohne zurücksetzende Kanten gegeben.
	Ist $f'$ ein schmaler Fluss ohne Zurücksetzen auf der durch $X$ induzierten Teilinstanz, so ist $g := f' \sqcup f$ definiert durch \[
		g_e := \begin{cases}
			f'_e, & \text{falls $e\in E[X]$,} \\
			f_e, & \text{falls $e\in E\setminus E[X]$,}
		\end{cases}
	\] ein schmaler Fluss ohne Zurücksetzen auf der ursprünglichen Instanz.
\end{corollary}
\begin{proof}
	Es ist leicht zu sehen, dass $g$ ein $s$-Fluss minimaler Auslastung ist:
	Für einen Knoten $v \in V\setminus X$ ist $g(\delta^+_E(v)) - g(\delta^-_E(v)) = f(\delta^+_E(v)) - g(\delta^-_E(v)) = b_v$ und für $v\in X$ gilt
	\begin{align*}
	g(\delta^+_E(v)) - g(\delta^-_E(v)) =& f'(\delta^+_{E[X]}(v))  - f'(\delta^-_{E[X]}(v))  \\
	&+ f(\delta^+_E(v)\cap\delta^+_E(X)) - f(\delta^-_E(v)\cap\delta^-_E(X)) = b_v
	\end{align*}

	Sei ein $s$-$v$-Pfad $P$ mit positivem Fluss gegeben.
	Ist $v$ nicht in $X$, so hat $P$ nach Lemma~\ref{lemma-flow-minimal-congestion-sparsest-cut-then-outside-X-congestion-q} Auslastung $l_v$.
	Ist $v$ in $X$ enthalten, muss $P$ entlang Kanten innerhalb von $X$ laufen, da $g$ auf eingehenden Kanten von $X$ verschwindet.
	Nach Proposition~\ref{prop-restricted-minimal-flow-is-b-flow-on-induced-instance} stimmt die Auslastung von $v$ bezüglich der Einschränkung von $f$ mit der Auslastung von $v$ bezüglich $f$ überein, sodass auch die Auslastung von $P$ mit der Auslastung von $v$ bezüglich $f$ übereinstimmen.
\end{proof}

\begin{algorithm}
\caption{Berechnung eines schmalen Flusses ohne Zurücksetzen}
\label{algorithm-computation-thin-flow-without-resetting}
\begin{algorithmic}[1]
	\Procedure{ThinFlow}{$V, E, u, b, s$}
	\State $f \gets 0^E$
	\State $(V', E', b') \gets (V, E, b)$
	\While{$b_s \neq 0$}
	\State Berechne auslastungsminimalen $s$-Fluss $f'$ und dünnsten Schnitt $X$
	\Statex auf $(V', E', b')$ mit Kapazitäten $u_{\mid E'}$; melde Fehler, falls $f'$ nicht existiert.
	\State $f_{\mid E'\setminus E'[X]} \gets f'_{\mid E'\setminus E'[X]}$
	\State $(V', E', b') \gets (V'[X], E'[X], b'[X])$
	\EndWhile
	\State\Return $f$
	\EndProcedure
\end{algorithmic}
\end{algorithm}

Dies bildet die Grundlage des polynomiellen Algorithmus'~\ref{algorithm-computation-thin-flow-without-resetting} zur Berechnung von schmalen Flüssen ohne Zurücksetzen.

\begin{theorem}
		Algorithmus~\ref{algorithm-computation-thin-flow-without-resetting} berechnet einen schmalen Fluss ohne Zurücksetzen oder bricht mit einer Fehlermeldung ab, falls kein $b$-Fluss existiert.
		Er benötigt eine Laufzeit von $\bigO((\size{b} + \size{u})^2 n^3 m)$.
\end{theorem}
\begin{proof}
	Für die Korrektheit des Algorithmus benutzt man Korollar~\ref{cor-thin-flow-alg-correct}:
	Man prüfe die folgende Invariante der Schleife:
	\glqq $f$ verschwindet auf $E'$ und ist $g$ ein schmaler Fluss auf $(V',E',b')$, so ist $g \sqcup f$ ein schmaler Fluss auf $(V, E, b)$.\grqq\ 
	Zu Beginn gilt dies offenbar, da in diesem Fall $(V', E', b')$ mit $(V, E, b)$ übereinstimmt und $f$ mit Nullen initialisiert wurde.
	Nun betrachte man die Invariante im Verlauf einer Iteration.
	Seien dazu $(f^1, V^1, E^1, b^1)$ die Werte von $(f, V', E', b')$ zu Beginn der Iteration, für die die Invariante bereits gilt, und $(f^2, V^2, E^2, b^2)$ die Werte nach Ende der Iteration.
	Dann sei $f'$ der berechnete auslastungsminimale Fluss auf $(V^1, E^1, b^1)$ und $(V^2, E^2, b^2)$ die durch den dünnsten Schnitt $X$ induzierte Teilinstanz von $(V^1, E^1, b^1)$.
	Man nehme nun an, $g$ sei ein schmaler Fluss auf $(V^2, E^2, b^2)$.
	Dann ist $g\sqcup f^2_{\mid E^1} = g \sqcup f'$ nach Korollar~\ref{cor-thin-flow-alg-correct} ein schmaler Fluss auf $(V^1, E^1, b^1)$ und aufgrund der Invariante ist $g \sqcup f^2 = (g \sqcup f') \sqcup f^1$ ein schmaler Fluss auf $(V, E, b)$.
	Außerdem verschwindet $f^2$ auf $E^2$, da $f$ nur auf $E^1\setminus E^2$ verändert wird.
	Terminiert die Schleife, so ist $b_s=0$, wodurch die Voraussetzung der Invariante
	für den Nullfluss $f_{\mid E'}$ erfüllt ist, sodass $f$ ein schmaler Fluss ist.
	
	Existiert kein $b$-Fluss auf $(V, E)$, so existiert auch kein auslastungsminimaler Fluss und der Algorithmus in Zeile~$5$ bricht mit einer Fehlermeldung ab.
	Sonst terminiert der Algorithmus nach mindestens $n$ Iterationen, denn in jeder Iteration wird aus $V'$ mindestens ein Knoten entfernt.
	Außerdem kann ein auslastungsminimaler $s$-Fluss nach Theorem~\ref{thm-compute-minimal-con-flow} in $\bigO((\size{b}+\size{u})^2 n^2 m)$ Zeit berechnet werden.
	Daraus kann man nach Lemma~\ref{lemma-calc-sparsest-cut} in $\bigO(n+m)$ arithmetischen Operationen einen dünnsten Schnitt gewinnen, wodurch eine Gesamtlaufzeit von $\bigO((\size{b} + \size{u})^2 n^3 m)$ entsteht.
\end{proof}


\begin{example}
	
	\begin{figure}
		\newcommand{\newnode}[4]{\node[lul] (#1) at #2 {#3 \\[0.5em] #4};}
		\centering
		\begin{subfigure}{\textwidth}
			\begin{tikzpicture}[lul/.style={draw,
				ellipse,
				align=center,
				inner sep=0pt,
				outer sep=4pt,
				text width=7mm,
				minimum height=1.5cm
			},
			scale=1]
			
			\newnode{0}{(0,2)}{$a$}{$32$}
			\newnode{1}{(3,2)}{$b$}{$0$}
			\newnode{2}{(5,0)}{$c$}{$-2$}
			\newnode{3}{(7,2)}{$d$}{$-12$}
			\newnode{4}{(10,3)}{$e$}{$-2$}
			\newnode{5}{(10,1)}{$f$}{$0$}
			\newnode{6}{(13,2)}{$g$}{$-16$}
			
			\begin{scope}[-Latex]
			\path [-Latex] (0) edge node[above] {$10/4$} (1);
			\path [-Latex] (0) edge[bend right] node[above] {$4/4$} (2);
			\path [-Latex] (0) edge[bend left] node[above] {$18/4$} (4);
			\path [-Latex] (1) edge node[right] {$0/2$} (2);
			\path [-Latex] (1) edge[snake=snake,segment amplitude=.4mm,segment length=2mm,line after snake=2mm] node[above] {$16/2$} (3);
			\path [-Latex] (2) edge node[left] {$4/2$} (3);
			\path [-Latex] (3) edge node[above] {$0/1$} (4);
			\path [-Latex] (3) edge node[above] {$8/2$} (5);
			\path [-Latex] (4) edge[snake=snake,segment amplitude=.4mm,segment length=2mm,line after snake=2mm] node[above] {$8/1$} (6);
			\path [-Latex] (5) edge[snake=snake,segment amplitude=.4mm,segment length=2mm,line after snake=2mm] node[above] {$8/1$} (6);
			\end{scope}
			
			\begin{scope}
			\draw[dashed, color=blue] (11.5,-1) -- (11.5, 4.5);
			\node[color=blue] at (11,-0.5) {$X$};
			\end{scope}
			\end{tikzpicture}
			\subcaption{Erste Iteration. Die Auslastung der Flaschenhalskanten beträgt $8$.}
		\end{subfigure}
		\par\bigskip
		\begin{subfigure}{\textwidth}
			\begin{tikzpicture}[lul/.style={draw,
				ellipse,
				align=center,
				inner sep=0pt,
				outer sep=4pt,
				text width=7mm,
				minimum height=1.5cm
			},
			scale=1]
			
			\newnode{0}{(0,2)}{$a$}{$32$}
			\newnode{1}{(3,2)}{$b$}{$0$}
			\newnode{2}{(5,0)}{$c$}{$-2$}
			\newnode{3}{(7,2)}{$d$}{$-12$}
			\newnode{4}{(10,3)}{$e$}{$-10$}
			\newnode{5}{(10,1)}{$f$}{$-8$}
			
			\begin{scope}[-Latex]
			\path [-Latex] (0) edge node[above] {$10/4$} (1);
			\path [-Latex] (0) edge[bend right] node[above] {$12/4$} (2);
			\path [-Latex] (0) edge[bend left] node[above] {$10/4$} (4);
			\path [-Latex] (1) edge node[right] {$0/2$} (2);
			\path [-Latex] (1) edge[snake=snake,segment amplitude=.4mm,segment length=2mm,line after snake=2mm] node[above] {$10/2$} (3);
			\path [-Latex] (2) edge[snake=snake,segment amplitude=.4mm,segment length=2mm,line after snake=2mm] node[left] {$10/2$} (3);
			\path [-Latex] (3) edge node[above] {$0/1$} (4);
			\path [-Latex] (3) edge node[above] {$8/2$} (5);
			\end{scope}
			
			\begin{scope}
			\draw[dashed, color=blue, rounded corners=5mm] (6.75,-1) -- (5.5, 3.75) -- (12, 1.25);
			\node[color=blue] at (6.25,-0.5) {$X$};
			\end{scope}
			\end{tikzpicture}
			\subcaption{Zweite Iteration. Die Auslastung der Flaschenhalskanten beträgt $5$.}
		\end{subfigure}
		\par\bigskip
		\begin{subfigure}{.45\columnwidth}
			\begin{tikzpicture}[lul/.style={draw,
				ellipse,
				align=center,
				inner sep=0pt,
				outer sep=4pt,
				text width=7mm,
				minimum height=1.5cm
			}]
			
			\newnode{0}{(0,2)}{$a$}{$32$}
			\newnode{1}{(3,2)}{$b$}{$-10$}
			\newnode{2}{(5,0)}{$c$}{$-12$}
			\newnode{4}{(5,3)}{$e$}{$-10$}
			
			\begin{scope}[-Latex]
			\path [-Latex] (0) edge[snake=snake,segment amplitude=.4mm,segment length=2mm,line after snake=2mm] node[above] {$11/4$} (1);
			\path [-Latex] (0) edge[bend right,decorate, decoration=snake, segment amplitude=.4mm,segment length=2mm,line after snake=2mm] node[above] {$11/4$} (2);
			\path [-Latex] (0) edge[bend left] node[above] {$10/4$} (4);
			\path [-Latex] (1) edge node[right] {$1/2$} (2);

			\end{scope}
			
			\begin{scope}
			\draw[dashed, color=blue, rounded corners=5mm] (0, 0) -- (3, 3.5) -- (6, 0.5);
			\node[color=blue] at (0,0.6) {$X$};
			\end{scope}
			\end{tikzpicture}
			\subcaption{Dritte Iteration. Die Auslastung der Flaschenhalskanten beträgt $2{,}75$.}
		\end{subfigure} \hfill
		\begin{subfigure}{.45\columnwidth}
			\centering
			\begin{tikzpicture}[lul/.style={draw,
				ellipse,
				align=center,
				inner sep=0pt,
				outer sep=4pt,
				text width=7mm,
				minimum height=1.5cm
			},
			scale=1]
			
			\newnode{0}{(0,2)}{$a$}{$10$}
			\newnode{4}{(3,2)}{$e$}{$-10$}
			
			\begin{scope}[-Latex]
			\path [-Latex] (0) edge[snake=snake,segment amplitude=.4mm,segment length=2mm,line after snake=2mm] node[above] {$10/4$} (4);
			
			\end{scope}
			
			\begin{scope}
			\draw[dashed, color=blue, rounded corners=5mm] (1.5, 0.5) -- (1.5, 3.5);
			\node[color=blue] at (1,1) {$X$};
			\end{scope}
			\end{tikzpicture}
			\subcaption{Vierte Iteration. Die Auslastung der Flaschenhalskante beträgt $2{,}5$.}
		\end{subfigure}
		
		\caption{Algorithmus zur Bestimmung von schmalen Flüssen ohne Zurück\-setzen. Geschlängelte Kanten sind Flaschenhalskanten in den jeweiligen auslastungs\-minimalen Flüssen. Kanten $e$ sind mit $f_e/u_e$ und Knoten mit der Bezeichnung und der Balance beschriftet.}
	\end{figure}	
\end{example}
\section{Berechnung dynamischer Nash-Flüsse mit schmalen Flüssen}\label{sec-nash-flow-extension}
In diesem Abschnitt soll es nun schließlich darum gehen, den Zusammenhang zwischen dynamischen Nash-Flüssen und schmalen Flüssen mit Zurücksetzen aufzuzeigen.
Danach wird kurz auf die Existenz dynamischer Nash-Flüsse und deren Berechnung bei konstanter Einflussrate in das Netzwerk eingegangen.
Dazu wird wieder die Notation dynamischer Flüsse aus den Kapiteln~\ref{chapter-dynamic-flows} und~\ref{chapter-nash-flows} verwendet.

Für diese Ergebnisse werden schmale Flüsse mit einer kleinen Modifikation verwendet:

\begin{definition}[Normierter schmaler Fluss]\label{def-normalized-thin-flow}
	Seien ein $s$-Netzwerk $\mathcal{N}$ und ein $b$-Fluss $f$ gegeben.
	Für einen $s$-$v$-Pfad $P=(e_1, \dots, e_k)$ bezeichnet die \emph{normierte Auslastung von $P$} die Verkettung $\rho_{e_k}(\cdot, f_{e_k}) \circ \cdots \circ \rho_{e_1}(\cdot, f_{e_1})(1)$.
	Die zu $f$ gehörige \emph{normierte Knotenauslastung} $l\in\R^V$ ist für jeden Knoten $v\in V$ die minimale Auslastung eines $s$-$v$-Pfades.
	Dann heißt $f$ ein \emph{normierter $b_s$-wertiger schmaler Fluss mit Zurücksetzen auf $E_1$}, falls die Auslastung jedes $s$-$v$-Pfades mit positivem Fluss $l_v$ beträgt.
\end{definition}

Schnell kann man sehen, dass sich die Ergebnisse aus den vorherigen Kapiteln ohne großen Aufwand übertragen lassen:

\begin{proposition}
	Es sei ein $b$-Fluss $f$ in einem $s$-Netzwerk $\mathcal{N}$ mit $b_s > 0$ gegeben.
	Man erhalte das Netzwerk $\mathcal{N}'$ aus $\mathcal{N}$, indem man einen zusätzlichen Knoten $s'$ und eine Kante $(s, s')$ mit Kapazität $b_s$ hinzufügt und das Angebot von $s$ mit $b'_s \coloneq 0$ und $b'_{s'}\coloneq b_s$ auf $s'$ überträgt.
	Ergänzt man $f$ durch den Flusswert $b_s$ auf der Kante $(s', s)$, so erhält man den $b'$-Fluss $f'$.
	
	Dann ist die normierte Auslastung jedes $s$-$v$-Pfads $P$ in $\mathcal{N}$ gerade die Auslastung des $s'$-$v$-Pfads $P'$ in $\mathcal{N}'$, der vor dem Pfad $P$ noch die Kante $(s', s)$ benutzt.
	Insbesondere ist $f$ genau dann ein normierte schmaler Fluss in $\mathcal{N}$, wenn $f'$ ein schmaler Fluss in $\mathcal{N'}$ ist.
\end{proposition}
\begin{proof}
	Die Kante $(s', s)$ hat für jeden $b'$-Fluss Auslastung $1$.
	Daher ist die normierte Auslastung eines $s$-$v$-Pfades gerade die Auslastung vom durch die Kante $(s', s)$ verlängerten Pfad.
\end{proof}

Dadurch können normierte schmale Flüsse durch die Bestimmung nicht-normier\-ter schmalen Flusses ermittelt werden.
Außerdem können Lemma~\ref{lemma-equivalent-thin-flow} ohne Verän\-derung und Lemma~\ref{lemma-thin-flow-t-def} mit der kleinen Modifikation, dass statt~\ref{def-thin-flow-source}, also $l_s = 0$, hier $l_s = 1$ gelten muss, für normierte schmale Flüsse übernommen werden.

Das folgende Theorem liefert das Resultat, dass ein dynamischer Nash-Fluss zu jedem Zeitpunkt $\theta\in\R$ einen normierten schmalen Fluss mit Zurücksetzen induziert.
Die Voraussetzung, dass hierbei die entsprechenden Ableitungen existieren, kann man dadurch rechtfertigen, dass nach \cite[Kap. VII, Folgerung~4.12~b)]{Elstrodt2011} absolut stetige Funktionen fast überall differenzierbar sind.
Zudem gilt nach \cite[Kap. VII, Aufgabe~4.10]{Elstrodt2011} die Substitutionsregel für absolut stetige Funktionen, wodurch
\[
\int_{l_v(\theta_1)}^{l_v(\theta_2)} f_{vw}^+(t) \diff t = \int_{\theta_1}^{\theta_2} f_{vw}^+(l_v(t)) l_v'(t) \diff t
\]
gefolgert werden kann.
Daher gilt auch für fast alle $\theta\in\R$:
\[ 
x_{vw}'(\theta) = f_{vw}^+(l_v(\theta))l_v'(\theta) \text{~~~ bzw. analog ~~~} x_{vw}'(\theta) = f_{vw}^-(l_w(\theta)) l_w'(\theta).
\]

\begin{theorem}
	Seien ein dynamischer Nash-Fluss $f$ im Graphen $(V,E)$ sowie ein Zeitpunkt $\theta\in\R$ gegeben.
	Existieren die Ableitungen $x_{vw}'(\theta)$ und $l_v'(\theta)$ mit 
	\[
	x_{vw}'(\theta) = f_{vw}^+(l_v(\theta)) l_v'(\theta)= f_{vw}^-(l_w(\theta))l_w'(\theta)
	\]
	für alle Kanten $vw$ und Knoten $v$, so ist der statische Fluss $x'(\theta) \in \R^{E_\theta}$, eingeschränkt auf die zu $\theta$ aktiven Kanten, ein normierter schmaler Fluss mit Zurück\-setzen auf den Kanten mit Warteschlange $E_1\coloneq \{vw\in E \mid q_{vw}(l_v(\theta))>0 \}$ von Wert $d(\theta)$.
	Dabei sind die Ableitungen $(l_v'(\theta))_{v\in V}$ die normierten Knotenauslastungen.
\end{theorem}
\begin{proof}
	Man bemerke zunächst $l_s'(\theta) = 1$.
	Da außerdem der Einfluss in das Netzwerk 
	\[
	d(\theta)= \sum_{e\in \delta^+(s)} f_e^+(\theta) - \sum_{e\in\delta^-(s)} f_e^-(\theta) = \sum_{e\in \delta^+(s)} x_e'(\theta) - \sum_{e\in\delta^-(s)} x_e'(\theta)
	\]
	erfüllt, ist $x'(\theta)$ ein $d(\theta)$-wertiger Fluss, weil $x_e'(\theta)$ nach Lemma~\ref{lemma-x-locally-constant} für inaktive Kanten $e$ verschwindet. Es bleibt also die Bedingungen~\ref{def-thin-flow-x-zero}-\ref{def-thin-flow-no-resetting-edge} zu zeigen.
	
	Um Bedingung~\ref{def-thin-flow-no-resetting-edge} zu zeigen,
	nehme man an, dass die Warteschlange jeder eingehenden Kante $vw$ eines Knotens $w$ zur Zeit $l_v(\theta)$ leer ist.
	Da für alle $n\in\N$ eine eingehende und zur Zeit $\theta_n \coloneq \theta + 1/n$ aktive Kante existiert, gibt es eine Kante $vw$ die zu unendlich vielen $\theta_n$ aktiv ist.
	Betrachtet man diese Teilfolge $(\theta_{n_k})_{k\in\N}$ der aktiven Zeitpunkte von $vw$, ist $vw$ wegen der Stetigkeit auch zur Zeit $\theta$ aktiv und es gilt 
	\[
	l_w'(\theta) = \lim_{k\to\infty} \frac{l_w(\theta_{n_k})- l_w(\theta)}{1/n_k} = \lim_{k\to\infty} \frac{ l_v(\theta_{n_k}) + q_{vw}(l_v(\theta_{n_k})) - l_v(\theta) }{1/n_k} \geq l_v'(\theta).
	\]
	Also ist insbesondere $l_w'(\theta) \geq \min_{vw\in \delta^-(w)\cap E_\theta} l_v'(\theta)$.
	
	Sei nun eine Kante $vw\in E_\theta$, also eine aktive Kante zum Zeitpunkt $\theta$, gegeben. Man prüfe die Bedingungen~\ref{def-thin-flow-x-zero}, \ref{def-thin-flow-x-positive} und \ref{def-thin-flow-resetting-edge} jeweils in den folgenden drei Fällen:
	
	\begin{description}[leftmargin=0cm, topsep=0cm, itemindent=0.5cm]
		\item[1. Fall:] $\exists \varepsilon > 0:\forall \theta'\in (\theta, \theta + \varepsilon ] : q_{vw}(l_v(\theta')) > 0$.
		
		Nach Lemma~\ref{lemma-nash-flow-waiting-queue-implies-active-edge} ist $[\theta,\theta+\varepsilon]\subseteq \Theta_{vw}$.
		Außerdem ist $q_{vw}$ nach Proposition~\ref{prop-feasible-flow}~\ref{prop-feasible-flow-positive-queue} auf $[ l_v(\theta') , l_w(\theta') - \tau_{vw} )$ positiv, also auch auf $( l_w(\theta)-\tau_{vw} , l_w(\theta + \varepsilon) - \tau_{vw} )$.
		Mit~\ref{def-feasible-flow-queue-with-capacity} folgere man
		\[ x_{vw}(\theta + \varepsilon) - x_{vw}(\theta) = \int_{l_w(\theta)-\tau_{vw}}^{l_w(\theta + \varepsilon)-\tau_{vw}} f_{vw}^-(t+\tau_{vw}) \diff t
		= u_{vw} (l_w(\theta + \varepsilon) - l_w(\theta)).\]
		Teilt man diese Gleichung durch $\varepsilon$, so erhält man $x_{vw}'(\theta) = u_{vw} l_w'(\theta)$ für $\varepsilon\rightarrow 0$.
		Ist $vw\in E_1$, so ist also Bedingung~\ref{def-thin-flow-resetting-edge} erfüllt.
		Für Bedingung~\ref{def-thin-flow-x-zero} setze man $x_{vw}'(\theta)=0$ voraus, wodurch $l_w'(\theta)=0$ folgt und mit der Monotonie von $l_v$ gilt $0 \leq l_v'(\theta)$.
		
		Ist $vw\notin E_1$, ist also die Warteschlange zum Zeitpunkt $l_v(\theta)$ leer, so gilt: 
		\[ 
		l_w(\theta+\varepsilon) - l_w(\theta) = l_v(\theta + \varepsilon) + q_{vw}(l_v(\theta + \varepsilon)) - l_v(\theta) \geq l_v(\theta + \varepsilon) - l_v(\theta).
		\]
		Teilt man wieder durch $\varepsilon$, so erhält man für $\varepsilon \rightarrow 0$ Bedingung~\ref{def-thin-flow-x-positive} mit $l_w'(\theta) \leq l_v'(\theta)$ und dem Resultat des letzten Absatzes.
		
		\item[2. Fall:] $\exists \varepsilon > 0: (\theta, \theta + \varepsilon] \subseteq \Theta_{vw}^c$.
		
		Nach Lemma~\ref{lemma-nash-flow-waiting-queue-implies-active-edge} gilt bereits $vw\notin E_1$ und nach Lemma~\ref{lemma-x-locally-constant} ist $x_{vw}'(\theta) = 0$. 
		Es muss nur Bedingung~\ref{def-thin-flow-x-zero} geprüft werden:
		Es gilt $l_w(\theta + \varepsilon) - l_w(\theta) < l_v(\theta + \varepsilon) - l_v(\theta)$ wegen $\theta\in\Theta_{vw}$ .
		Teilt man diese Ungleichung durch $\varepsilon$, so erhält man für $\varepsilon\rightarrow 0$ die Bedingung $l_w'(\theta)\leq l_v'(\theta)$.
		
		\item[3. Fall:] $\forall \varepsilon>0: \exists \theta_{\varepsilon}\in (\theta, \theta+\varepsilon]: T_{vw}(l_v(\theta_\varepsilon)) = l_w(\theta_\varepsilon)$.
		
		Dies ist die exakte Umkehrung der Bedingung von Fall 2.
		Zusätzlich betrachte man diesen Fall nur, falls Fall 1 nicht eintritt.
		Das heißt, für alle $\theta_\varepsilon$ existiert ein $\theta'\in(\theta, \theta_\varepsilon]$ mit $q_{vw}(l_v(\theta')) = 0$; insbesondere ist $vw$ nicht in $E_1$ enthalten.
		Dann wähle man $\theta'_\varepsilon\coloneq \max\{ \theta'\in (\theta, \theta_\varepsilon] \mid q_{vw}(l_v(\theta')) = 0 \}$ als das Maximum solcher Zeitpunkte, welches aufgrund der Stetigkeit von $q_{vw}\circ l_u$ existiert.
		Nach Konstruktion ist $q_{vw}\circ l_v$ im Intervall $(\theta_\varepsilon', \theta_\varepsilon)$ positiv und nach Lemma~\ref{lemma-nash-flow-waiting-queue-implies-active-edge} ist die Kante $vw$ in diesem Intervall aktiv.
		Nun impliziert $\theta_\varepsilon'\in \Theta_{vw}$ gerade $\theta_\varepsilon\in\Theta_{vw}$, da $\Theta_{vw}$ ab\-ge\-schlossen ist.
		Daher folgt aus $l_w(\theta_\varepsilon') - l_w(\theta) = l_v(\theta_\varepsilon') - l_v(\theta)$ 		Bedingung~\ref{def-thin-flow-x-zero}, indem man durch $\theta_\varepsilon'-\theta$ teilt und $l_w'(\theta) = l_v'(\theta)$ für $\varepsilon\rightarrow0$ erhält.
		
		Für Bedingung~\ref{def-thin-flow-x-positive} bleibt zu zeigen, dass $x_{vw}'(\theta) /u_{vw}\leq l_w'(\theta)$ gilt.
		Wegen Bedingung~\ref{def-feasible-flow-capacity} ist $x_{vw}(\theta + \varepsilon)-x_{vw}(\theta) = \int_{l_w(\theta)}^{l_w(\theta+\varepsilon)} f_{vw}^-(t) dt\leq (l_w(\theta + \varepsilon) - l_w(\theta)) u_{vw}$ für beliebiges $\varepsilon>0$.
		Durch Teilen mit $\varepsilon u_{vw}$ erhält man das gewünschte Resultat für $\varepsilon\rightarrow 0$.
	\end{description}\vspace{-1.4em}
\end{proof}

\begin{definition}
	Ein \emph{dynamischer Fluss $f$ mit Zeithorizont $T\geq0$} ist ein Fluss, für dessen Zufluss $d(\theta)= 0$ für $\theta\geq T$ gilt.
\end{definition}

Durch folgende Proposition erkennt man, dass in einem Nash-Fluss mit Zeithorizont $T$ der Zu- bzw. Abfluss ab der frühestmöglichen Ankunftszeit am Start- bzw. Endknoten jeder Kante bei Startzeit $T$ fast überall verschwinden.

\begin{proposition}
	Für einen dynamischen Nash-Fluss $f$ mit Zeithorizont $T$ und eine Kante $e\in E$ gilt $x_{e}(\theta) = x_{e}(T)$ für alle $\theta \geq T$.
	Insbesondere verschwinden $f_{vw}^+$ ab dem Zeitpunkt $l_v(T)$ und $f_{vw}^-$ ab dem Zeitpunkt $l_w(T)$ fast überall für alle Kanten $vw\in E$.
\end{proposition}
\begin{proof}
	Der statische $s$-$t$-Fluss $x(\theta) -x(T)$ ist nach Proposition~\ref{prop-nash-flow-s-t-path-decomposable} in $s$-$t$-Wege zerlegbar.
	Zudem hat der Fluss $x(T)$ den gleichen Flusswert wie $x(\theta)$, da
	\[ \sum_{e\in \delta^+(s)} x_e(T) - \sum_{e\in\delta^-(s)} x_e(T) = \sum_{e\in \delta^+(s)} x_e(\theta) - \sum_{e\in\delta^-(s)} x_e(\theta)\]
	wegen $d(\xi) = 0$ für $\xi \geq T $ gilt.
	Daher ist $x(\theta)- x(T)$ bereits der Nullfluss, sodass $x(T)$ und $x(\theta)$ bereits auf allen Kanten übereinstimmen, wodurch die Behauptung folgt.
\end{proof}

\begin{definition}[$\alpha$-Erweiterung]
	Seien ein dynamischer Nash-Fluss $f$ mit Zeithorizont $T$ und ein normierter schmaler Fluss $x'$ mit Zurücksetzen auf den Kanten mit Warteschlange $E_1 \coloneq \{ vw\in E \mid q_{vw}(l_v(T)) > 0 \} $ und Knotenbewertung $l'$ im Graphen $G_T$ der aktiven Kanten und ein $\alpha \geq 0$ gegeben.
	
	Ergänzt man die Werte aus $f$, sodass für alle zur Zeit $T$ aktiven Kanten $vw\in E_T$
	\begin{align*}
	&\tilde{f}_{vw}^+(\theta)\coloneq \frac{x_{vw}'}{l_v'} \text{~~~ für $\theta\in [l_v(T), l_v(T)+\alpha l_v')$ und } \\ &\tilde{f}_{vw}^-(\theta)\coloneq \frac{x_{vw}'}{l_w'} \text{~~~ für $\theta\in [l_w(T), l_w(T)+\alpha l_w')$}
	\end{align*}
	gelten, so erhält man eine \emph{$\alpha$-Erweiterung $\tilde{f}$ von $f$ mit $x'$}.
	Dabei entspricht der Netzwerkzufluss von $\tilde{f}$ im Intervall $[T, T+\alpha)$ gerade dem Wert von $x'$.
\end{definition}

Im nächsten Theorem wird schließlich gezeigt, dass eine solche $\alpha$-Erweiterung wieder einen Nash-Fluss erzeugt.

\begin{notation}
	Im folgenden Theorem und Beweis werden alle zur $\alpha$-Erweiterung $\tilde{f}$ gehörigen Größen wie der kumulative Zufluss $\tilde{F}_e^+$, die Wartezeit $\tilde{q}_e$ etc. mit einer Tilde notiert.
\end{notation}

\begin{theorem}\label{thm-alpha-extension-is-nash-flow}
	Für jede $\alpha$-Erweiterung $\tilde{f}$ eines dynamischen Nash-Flusses $f$ mit Zeithorizont $T$ und normiertem schmalen Fluss $x'$ mit Knotenauslastungen $l'$, die für alle Kanten mit positiver Warteschlange zum Zeitpunkt $T$ die Bedingung
	\begin{equation}\tag{$\alpha 1$}\label{equation-alpha-queuing-edge}
		l_w(T) - l_v(T) + \alpha(l_w' - l_v') \geq \tau_{vw}
	\end{equation}
	erfüllt, gelten die folgenden Aussagen:
	\begin{enumerate}[label=(\roman*)]
		\item Für positive $x_{vw}'$ und $\theta\in[T, T+\alpha]$ gilt $l_w(T) + (\theta - T)l_w' \geq l_v(T) + (\theta - T)l_v' + \tau_{vw}$.
		Für $\theta \leq l_v(T)$ gilt $\tilde{f}^-_{vw}(\theta + \tau_{vw})=f^-_{vw}(\theta + \tau_{vw})$ für alle $vw\in E$.
		Insbesondere ist $\tilde{q}_e(\theta) = q_e(\theta)$ und $\tilde{T}_e(\theta) = T_e(\theta)$ für $\theta \leq l_v(T)$.
		\item Die $\alpha$-Erweiterung $\tilde{f}$ ist ein zulässiger dynamischer Fluss mit Zeithorizont $T+\alpha$ und für alle $\gamma \in [0, \alpha]$ gilt
		\[ \tilde{F}_{vw}^+(l_v(T) + \gamma l_v') = \tilde{F}_{vw}^-(l_w(T) + \gamma l_w'). \]
		\item Gilt zusätzlich für alle zum Zeitpunkt $T$ inaktiven Kanten die Bedingung \begin{equation}\tag{$\alpha 2$}\label{equation-alpha-inactive-edge}
		l_w(T) - l_v(T) + \alpha(l_w' -l_v') \leq \tau_{vw},
		\end{equation}
		so sind die $\tilde{f}$ zugeordneten frühesten Ankunftszeiten $\tilde{l}_v(\theta)$ für $\theta \leq T+\alpha$ durch
		\[ \tilde{l}_v(\theta) = \begin{cases}
		l_v(\theta), & \text{ falls $\theta < T$,} \\
		l_v(T) + (\theta - T) l_v', & \text{ falls $\theta \in [T, T+\alpha]$.}
		\end{cases}\]
		gegeben und der Fluss $\tilde{f}$ ist ein dynamischer Nash-Fluss.
	\end{enumerate}
\end{theorem}
\begin{proof}
	\begin{description}[leftmargin=0cm, topsep=0cm, itemindent=0.5cm]
		\item[Zu Aussage]\emph{(i):}
		
		Ist $vw\in E_1$ mit $l_w'<l_v'$, so gilt mit Voraussetzung~(\ref{equation-alpha-queuing-edge})
		\[
		l_w(T)-l_v(T) + (\theta - T)(l_w' - l_v') \geq l_w(T)-l_v(T)+\alpha(l_w'- l_v')\geq \tau_{vw}.
		\]
		Sonst gilt $l_w' \geq l_v'$ nach Bedingung~\ref{def-thin-flow-x-positive} und mit $T\in \Theta_{vw}$ folgt \begin{align*}
		l_w(T)+(\theta-T)l_w' &= l_v(T) + q_{vw}(l_v(T))+\tau_{vw}+(\theta - T)l_w'\\
		&\geq l_v(T) + (\theta-T)l_v'+\tau_{vw}.
		\end{align*}
		Daraus folgt wiederum mit $\theta + \tau_{vw} \leq l_v(T) + \tau_{vw} \leq l_w(T)$ auch \[ 
		\tilde{f}_{vw}^-(\theta + \tau_{vw}) = f_{vw}^-(\theta + \tau_{vw}) \text{~~~ für alle $\theta\leq l_v(T)$}.
		\]
		Insbesondere gilt $\tilde{q}_{vw}(\theta) = q_{vw}(\theta)$ und sogar $ \tilde{T}_{vw}(\theta)= T_{vw}(\theta)$ für $\theta \leq l_v(T)$.
		
		\item[Zu Aussage]\emph{(ii):}
		
		Um zu zeigen, dass $\tilde{f}$ zulässig ist, zeige man die Eigenschaften \ref{def-feasible-flow-capacity}-\ref{def-feasible-flow-queue-with-capacity}.
		Die Bedingung~\ref{def-feasible-flow-no-flow-at-node} gilt, da $x'$ ein statischer $s$-$t$-Fluss ist und Flusserhaltung in $V\setminus \{ s, t \}$ erfüllt.
		Für die Bedingungen~\ref{def-feasible-flow-capacity}, \ref{def-feasible-flow-no-negative-flow} und \ref{def-feasible-flow-queue-with-capacity} genügt es, Kanten $e\in E_T$ mit $x_e' > 0$ zu prüfen, da sonst $\tilde{f}_e$ mit $f_e$ übereinstimmt und $f$ bereits zulässig ist.
		Sei $vw$ also eine Kante mit $x_{vw}' > 0$, die zur Zeit $T$ bzgl. $f$ aktiv ist.
		Es gilt die Kapazitätsbeschränkung \ref{def-feasible-flow-capacity}, da $l_w'\geq x_{vw}' / u_{vw}$ wegen~\ref{def-thin-flow-x-positive} und~\ref{def-thin-flow-resetting-edge} gilt, wodurch $\tilde{f}^-_{vw}(\theta)=x_{vw}'/l_w'\leq u_{vw}$ für $\theta\in[l_w(T), l_w(T)+\alpha l_w')$ folgt.
		
		Für \ref{def-feasible-flow-no-negative-flow} zeige man $\tilde{F}^+_{vw}(\theta)\geq \tilde{F}_{vw}^-(\theta+\tau_{vw})$ für alle $\theta\in\R$.
		Für $\theta\leq l_v(T)$ gilt dies bereits nach (i).
		Existiert ein $\gamma\in[0, \alpha]$ mit $\theta=l_v(T) + \gamma l_v'$, so gilt
		\begin{align*}
		\tilde{F}_{vw}^+(l_v(T) + \gamma l_v')&=F_{vw}^+(l_v(T))+\gamma x_{vw}'\\
		&= F_{vw}^-(l_w(T))+ \gamma x_{vw}'= \tilde{F}_{vw}^-(l_w(T)+\gamma l_w').
		\end{align*}
		Daraus folgt die Aussage, da $\theta + \tau_{vw}\leq l_w(T) + \gamma l_w'$ nach (i) erfüllt ist.
		Für alle $\theta > l_v(T)+\alpha l_v'$ gilt 
		\[\tilde{F}_{vw}^+(\theta) = \tilde{F}_{vw}^+(l_v(T) + \alpha l_v') = \tilde{F}_{vw}^-(l_w(T) + \alpha l_w') \geq \tilde{F}_{vw}^-(\theta + \tau_{vw}).\]
		Es bleibt Bedingung \ref{def-feasible-flow-queue-with-capacity} zu prüfen, d.h. Warteschlangen sollen mit der Kantenkapazität abgebaut werden.
		Sei also $\tilde{q}_{vw}(\theta)$ positiv.
		Nach (i) gilt 
		\[
		\tilde{f}_{vw}^-(\theta + \tau_{vw}) = f_{vw}^-(\theta + \tau_{vw}) = u_{vw}
		\]
		für $\theta\leq l_v(T)$.
		Ist $\theta > l_v(T)$, so unterscheide man, ob die Warteschlange von $vw$ zur Zeit $l_v(T)$ positiv ist:
		Ist dies der Fall, so gilt $l_w' = x_{vw}' / u_{vw}$ nach~\ref{def-thin-flow-resetting-edge}, und ohne Einschränkung gilt $\theta + \tau_{vw} \geq l_w(T)$, denn $f_{vw}^-$ ist nach Proposition~\ref{prop-feasible-flow}~\ref{prop-feasible-flow-positive-queue} und Eigenschaft~\ref{def-feasible-flow-queue-with-capacity} konstant $u_{vw}$ auf $[l_v(T)+\tau_{vw},l_w(T))$.
		Existiert ein $\gamma\in [0, \alpha]$ mit $\theta = l_v(T) + \gamma l_v'$, so gilt nach (i):
		\[
		\theta + \tau_{vw} = l_v(T) + \gamma l_v' + \tau_{vw} \leq l_w(T) + \gamma l_w'.
		\]
		Damit ist $\tilde{f}^-_{vw}(\theta + \tau_{vw}) = x_{vw}'/l_w' = u_{vw}$.
		Ist hingegen $\theta > l_v(T) + \alpha l_v'$, so gilt
		\[
		0<\tilde{F}^+_{vw}(\theta) - \tilde{F}_{vw}^-(\theta +\tau_{vw}) = \alpha x_{vw}' - \min \{ \alpha l_w', \theta + \tau_{vw} - l_w(T) \} x_{vw}' / l_w'.
		\]
		Insbesondere ist also $\theta + \tau_{vw} < l_w(T) + \alpha l_w'$ und auch hier gilt \[
		\tilde{f}^-_{vw}(\theta + \tau_{vw}) = x_{vw}'/l_w' = u_{vw}.
		\]
		Nun betrachte man Kanten $vw$, die zum Zeitpunkt $l_v(T)$ keine Warteschlange haben.
		Nach~\ref{def-thin-flow-x-positive} gilt hier $l_w' = \max \{ l_v', x_{vw}' / u_{vw} \}$.
		Existiert ein $\gamma\in [0, \alpha]$ mit $\theta = l_v(T) + \gamma l_v'$, so ist $\theta+\tau_{vw}$ in $[l_w(T), l_w(T) + \alpha l_w']$ enthalten und es gilt
		\[
		0<\tilde{F}_{vw}^+(\theta) - \tilde{F}_{vw}^-(\theta + \tau_{vw})= \gamma x_{vw}' - \gamma l_v' x_{vw}'/l_w'.
		\]
		Daher ist $l_w' > l_v'$ und es müssen $l_w' = x_{vw}'/u_{vw}$ und $\tilde{f}_{vw}^-(\theta + \tau_{vw}) = u_{vw}$ gelten.
		Für $\theta > l_v(T) + \alpha l_v'$ gilt $\theta + \tau_{vw} > l_w(T) + \alpha l_v'$ und dadurch gilt
		\[
		0 < \tilde{F}_{vw}^+(\theta) - \tilde{F}_{vw}^-(\theta + \tau_{vw}) = \alpha x_{vw}' - \min \{ \alpha l_w', \theta - l_v(T) \}x_{vw}'/l_w'.
		\]
		Insbesondere ist $\theta - l_v(T) < \alpha l_w'$, was äquivalent zu $\theta + \tau_{vw} < l_w(T) + \alpha l_w'$ ist.
		Nun kann man $l_v' < l_w'$ folgern, was wiederum $l_w' = x_{vw}'/u_{vw}$ impliziert.
		Damit gilt $\tilde{f}^-(\theta + \tau_{vw}) = u_{vw}$.
		
		\item[Zu Aussage]\emph{(iii):}
		
		Man bemerke, dass $(l_v(\theta))_{v\in V}$ das Gleichungssystem in Proposition~\ref{prop-arrival-times-vector} für $\theta \leq T$ erfüllt, da hier $\tilde{T}_{vw}(\theta)$ und $T_{vw}(\theta)$ nach (i) übereinstimmen.
		Für $\theta \in (T, T+\alpha)$ löst $(l_v(T) + (\theta - T)l_v')_{v\in V}$ das System:
		Wegen $l_s' = 1$ gilt $l_s(T) + (\theta - T)l_s' = \theta$.
		Für $w\neq s$ ist $l_w(T) + (\theta - T) l_w'$ eine untere Schranke an $\tilde{T}_{vw}(l_v(T) + (\theta - T) l_v')$ für zur Zeit $T$ inaktive Kanten $vw$:
		Zunächst zeige man $l_w(T) + (\theta - T)l_w' \leq l_v(T) + (\theta - T)l_v' + \tau_{vw}$:
		Ist $l_w' \leq l_v'$, so gilt dies bereits, da $vw$ zur Zeit $l_v(T)$ keine Warteschlange hat.
		Für $l_w' \geq l_v'$ gilt mit der zusätzlichen Voraussetzung~(\ref{equation-alpha-inactive-edge}):
		\[
		l_w(T) - l_v(T) + (\theta - T) (l_w' - l_v') \leq l_w(T) - l_v(T) + \alpha (l_w' - l_v') \leq \tau_{vw}.
		\]
		Demnach ist also $l_w(T) + (\theta - T) l_w' \leq l_v(T) + (\theta - T) l_v' + \tau_{vw} \leq \tilde{T}_{vw}(l_v(T) + (\theta - T) l_v')$.
		
		Im nächsten Schritt zeige man $\tilde{T}_{vw}(l_v(T) + (\theta - T)l_v') = l_w(T) + (\theta - T) l_w'$ für Kanten $vw\in E_T$, für die $x_{vw}'$ positiv ist oder deren Warteschlange zur Zeit $l_v(T)$ positiv ist:
		Für $x_{vw}' > 0$ impliziert~(ii) zusammen mit Proposition~\ref{prop-feasible-flow}~\ref{prop-feasible-flow-det-outflow} bereits $\tilde{T}_{vw}(l_v(T) + (\theta-T)l_v') = l_w(T) + (\theta - T) l_w'$.
		Für $vw\in E_1$ mit $x_{vw}'=0$ gilt jedoch $l_w' = x_{vw}' / u_{vw} = 0$ nach~\ref{def-thin-flow-resetting-edge}.
		Entsprechend erfüllt die Warteschlange
		\begin{align*}
		\tilde{z}_{vw}(l_v(T) + (\theta - T)l_v') - z_{vw}(l_v(T)) &= (\theta - T) x_{vw}' - \int_{l_v(T) + \tau_{vw}}^{l_v(T) + (\theta - T) l_v' + \tau_{vw}} f_e^-(t) \diff t \\
		&= (\theta - T) (x_{vw}' -l_v' u_{vw}),
		\end{align*}
		da zusätzlich $(\theta-T) l_v' \leq \alpha (l_v' - l_w') \leq q_{vw}(l_v(T))$ gilt.
		Damit folgt
		\begin{align*}
		\tilde{T}_{vw}(l_v(T) + (\theta - T)l_v') &= l_w(T) + (\theta - T) l_v' + \tilde{q}_{vw}(l_v(T) + (\theta - T)l_v')- q_{vw}(l_v(T)) \\
		&= l_w(T) + (\theta - T) l_w'.
		\end{align*}
		Zuletzt betrachte man aktive Kanten $vw\notin E_1$ mit $x_{vw}' = 0$.
		Für diese Kanten gelten $l_w(T) = l_v(T) + \tau_{vw}$ und $l_w' \leq l_v'$ nach~\ref{def-thin-flow-x-zero}.
		Außerdem ist 
		\[
		z_{vw}(l_v(T) + (\theta - T) l_v') = F_{vw}^+(l_v(T)) - F_{vw}^-(l_w(T)) = 0,
		\]
		wodurch man $\tilde{T}_{vw}(l_v(T) + (\theta - T) l_v') = l_w(T) + (\theta - T)l_v' \geq l_w(T) + (\theta - T) l_w'$ folgern kann.
		Dabei gilt sogar Gleichheit nach~\ref{def-thin-flow-no-resetting-edge}, falls $w$ keine eingehende Kante mit positiver Warteschlange hat.
		Daher ist $l_w(T)+(\theta - T) l_w'$ nicht nur eine untere Schranke, sondern auch das Minimum an $\tilde{T}_{vw}(l_v(T) + (\theta - T) l_v')$ für alle $vw\in \delta^-(w)$.
		
		Um nun zu erkennen, dass $\tilde{f}$ ein Nash-Fluss ist, zeige man Bedingung~(ii) aus Theorem~\ref{thm-equivalencies-nash-flow}, d.h. man zeige $\tilde{x}_e^+(\theta) = \tilde{x}_e^-(\theta)$ für alle Kanten $e$ und Zeitpunkte $\theta$.
		Für $\theta \leq T$ gilt $\tilde{x}_e^+(\theta) = x_e^+(\theta) = x_e^-(\theta) = \tilde{x}_e^-(\theta)$ für alle Kanten $e\in E$.
		Des Weiteren gilt $\tilde{x}_e^+(\theta) = \tilde{x}_e^+(T) = \tilde{x}_e^-(T) = \tilde{x}_e^-(\theta)$ für Kanten mit $x_{e}' = 0$ und $\theta > T$.
		Für $\theta$ zwischen $T$ und $T+\alpha$ liefert~(ii) die Behauptung.
		Für $\theta > T+\alpha$ gilt schließlich $\tilde{x}_e^+(\theta) = \tilde{x}_e^+(T + \alpha) = \tilde{x}_e^-(T+\alpha) = \tilde{x}_e^-(\theta)$.
\end{description}\vspace{-1.4em}
\end{proof}
\begin{remark}[Fortsetzung von Bemerkung~\ref{remark-thin-flow}]\label{remark-thin-flow-part-2}
	Es soll erörtert werden, warum die Bedingung~\ref{def-thin-flow-no-resetting-edge}, die hier im Vergleich zu~\cite[Definition~6]{Koch2011} hinzugezogen wurde, für die Korrektheit von Theorem~\ref{thm-alpha-extension-is-nash-flow} benötigt wird:
	Dies wird genutzt, damit die erweiterten Ankunftszeiten tatsächlich mit den angegebenen übereinstimmen.
	
	Man betrachte den dynamischen Nullfluss im Netzwerk aus Abbildung~\ref{figure-labels} mit Zeithorizont $0$.
	Die Kanten $st$ und $sv$ sind aktiv, wohingegen die Kante $vt$ inaktiv ist.
	Dann ist der Fluss $x'$ mit $x_{sv}'=x_{vt}' = 0$ und $x_{st}'=2$ nach \cite[Definition 6]{Koch2011} ein schmaler Fluss mit Zurücksetzen auf $\emptyset$ und Knotenauslastungen $l_s' = 1$, $l_v' = 0$ und $l_t' = 2$, aber für eine $\alpha$-Erweiterung mit $x'$ gilt $l_v(\theta) = \theta + 1 > 1 = l_v(0) + l_v' \theta$ für alle $\theta > 0$.

	\begin{figure}
		\centering
		\begin{tikzpicture}
		\node[draw, circle] (S) at (0,2) {$s$};
		\node[draw, circle] (V) at (4,2) {$v$};
		\node[draw, circle] (T) at (2,0) {$t$};
		
		\path [->] (S) edge node[above] {$1/1$} (V);
		\path [->] (S) edge node[below left] {$1/1$} (T);
		\path [->] (V) edge node[below right] {$10/1$} (T);
		\end{tikzpicture}
		\caption{Ein dynamisches Netzwerk mit Kantenbeschriftung $\tau_e / u_e$.}
		\label{figure-labels}
	\end{figure}
\end{remark}

Weiter kann man zeigen, dass man das $\alpha$ in Theorem~\ref{thm-alpha-extension-is-nash-flow} stets strikt positiv wählen kann:
\begin{proposition}
	Für einen dynamischen Nash-Fluss $f$ mit Zeithorizont $T$ und normierten schmalen Fluss $x'$ mit Knotenauslastungen $l'$ existiert ein $\alpha>0$, das sowohl (\ref{equation-alpha-queuing-edge}) und (\ref{equation-alpha-inactive-edge}) erfüllt.
	Dabei kann $\alpha$ gewählt werden zwischen $0$ und \[
	\inf\left(\left\{ \frac{q_{vw}(l_v(T))}{l_v' - l_w'} ~\middle|~ vw\in E_1, l_w' < l_v' \right\} \cup \left\{ \frac{l_v(T) + \tau_{vw} -l_w(T)}{l_w' - l_v'} ~\middle|~ vw\notin E_\theta, l_w' > l_v' \right\}\right).
	\]
\end{proposition}
\begin{proof}
	Für Kanten $vw$ mit positiver Warteschlange zur Zeit $T$ muss (\ref{equation-alpha-queuing-edge}), also $l_w(T) - l_v(T) + \alpha(l_w' - l_v') \geq \tau_{vw}$, gelten.
	Wegen Lemma~\ref{lemma-nash-flow-waiting-queue-implies-active-edge} ist dies äquivalent zu $\alpha(l_v' - l_w') \leq q_{vw}(l_v(T))$.
	Ist also $l_w'$ größer oder gleich $l_v'$, so ist die linke Seite nicht-positiv, sodass die Gleichung erfüllt ist.
	Für $l_w' < l_v'$ muss jedoch zusätzlich $\alpha\leq q_{vw}(l_v(T)) / (l_v' - l_w')$ gefordert werden.
	
	Für zur Zeit $\theta$ inaktive Kanten $vw$ muss~(\ref{equation-alpha-inactive-edge}), d.h. $\alpha(l_w' -l_v') \leq l_v(T) + \tau_{vw} - l_w(T)$, gelten.
	Per Definition ist $l_v(T) + \tau_{vw} - l_w(T)$ für inaktive Kanten positiv.
	Analog muss für diese Kanten mit $l_w' > l_v'$ also $\alpha \leq (l_v(T) + \tau_{vw} -l_w(T))/(l_w' - l_v')$ gefordert werden.
	
	Da alle diese oberen Schranken positiv sind, ist auch das Infimum $\alpha^*$ dieser Werte positiv.
\end{proof}

Dies hat Koch in~\cite{Koch2011} genutzt, um einen Algorithmus zur Berechnung von dynamischen Nash-Flüssen mit konstanter Netzwerkeinflussrate vorzuschlagen:
Ausgehend vom dynamischen Nullfluss mit Zeithorizont $0$ wird der Fluss nach und nach mit normierten schmalen Flüssen um einen möglichst großen Zeitraum $\alpha$ erweitert.

Ob dieser Algorithmus jedoch terminiert in dem Sinne, dass irgendwann ein $\alpha$ beliebig groß gewählt werden kann und somit der dynamische Nash-Fluss ab einem Zeitpunkt $T$ konstant gewählt werden kann, ist eine bisher unbeantwortete Frage.
Cominetti, Correa und Olver beantworten Teile der Frage in~\cite{CominettiExample}:
Übersteigt die minimale Warteschlangenkapazität $u(\delta^+(S))$ eines $s$-$t$-Schnitts $S$ den konstanten Netzwerkzufluss $d$ nicht, so erreicht ein dynamischer Nash-Fluss schließlich einen stabilen Zustand, ab dem der Fluss konstant ist.
Ob dieser Zustand jedoch durch endlich viele $\alpha$-Erweiterungen berechnet werden kann, bleibt ungeklärt.
Jedoch geben die Autoren ein Beispiel, in dem erst exponentiell viele $\alpha$-Erweiterungen zu einem stabilen Zustand führen.

Unter Benutzung des Lemmas von Zorn kann man jedoch die Existenz von dynamischen Nash-Flüssen mit konstantem Netzwerkzufluss zeigen:

\begin{theorem}
	Für einen konstanten Netzwerkzufluss $d\in\R_{> 0}$ existiert ein dynamischer Nash-Fluss.
\end{theorem}
\begin{proof}
	Man betrachte auf der Menge aller dynamischer Nash-Flüsse mit Zeithorizont $T$, wobei $T=\infty$ zugelassen wird, und Netzwerkzufluss $d$ die folgende Ordnung~$\preceq$:
	Für zwei dynamische Nash-Flüsse $f$ bzw. $g$ mit Zeithorizont $T^1$ bzw. $T^2$ gelte genau dann $(f, T^1)\preceq (g, T^2)$, wenn $T^1 \leq T^2$ gilt und $g^+_{vw}$ mit $f^+_{vw}$ bzw. $g^-_{vw}$ mit $f^-_{vw}$ auf $(-\infty, l_v(T^1)$ bzw. $(-\infty, l_w(T^1))$ für alle Kanten $vw$ übereinstimmen, wobei $l_v$ die früheste Ankunftszeit bzgl. $f$ an einem Knoten $v$ sein soll.
	Sei nun eine aufsteigende Kette $(f^i, T^i)_{i\in\N}$ bezüglich $\preceq$ gegeben.
	Man bemerke, dass für $i\leq j$ die frühesten Ankunftszeiten der Flüsse $f^i$ und $f^j$ auf $(-\infty, T^i)$ übereinstimmen.
	
	Man definiere nun $T^* \coloneq \lim_{i\in\N} T^i$ sowie die Ankunftszeit $l_v(\theta)$ an einem Knoten $v$ zur Zeit $\theta\in(-\infty, T^*)$, die von einem Fluss $f^i$ der Kette mit $T^i \geq \theta$ angenommen wird.
	Sei außerdem $l_v(T^*) \coloneq \lim_{i\in\N} l_v(T^i)$.
	Definiert man nun $f$ als den Fluss, dessen Zuflussrate $f_{vw}^+$ für $\theta<l_v(T^*)$ den Wert von $f^{i+}_{vw}(\theta)$ eines $i\in\N$ mit $l_v(T^i) > l_v(\theta)$ hat und für $\theta \geq l_v(T^*)$ den Wert $0$ hat, sowie dessen Abflussrate $f_{vw}^-$ für $\theta<l_w(T^*)$ den Wert von $f^{i-}_{vw}(\theta)$ eines $i\in\N$ mit $l_w(T^i) > l_w(\theta)$ hat und für $\theta \geq l_w(T^*)$ den Wert $0$ hat.
	Dann ist $f$ nicht nur ein zulässiger Fluss, sondern sogar ein dynamischer Nash-Fluss mit Zeithorizont $T^*$.
	Somit folgt, dass $(f, T^*)$ eine obere Schranke der Kette $(f^i, T^i)_{i\in\N}$ bezüglich $\preceq$ ist.
	Nach dem Lemma von Zorn besitzt die Ordnung $\preceq$ ein maximales Element $(f, T^*)$.
	Dann muss $T^*$ aber bereits unendlich sein, da man durch eine $\alpha$-Erweiterung von $f$ mit $\alpha > 0$ sonst einen dynamischen Nash-Fluss $(\tilde{f}, T^* + \alpha)$ erhält, für den $(\tilde{f}, T^* + \alpha) \npreceq (f, T^*)$ gelten würde.
\end{proof}

Mit der gleichen Argumentation kann man auch zeigen, dass auch für jeden rechts-konstanten Netzwerkzufluss ein dynamischer Nash-Fluss existiert:
Dabei muss lediglich $T^* + \alpha$ noch vor der nächsten Sprungstelle liegen.
Cominetti u. a. zeigen in \cite{Cominetti2015} sogar die Existenz für Netzwerke mit $\tau_e > 0$ für alle $e\in E$, bei denen der Netzwerkzufluss eine beliebige Funktion in $L^p(0,T)$ ist.


\clearpage          % neue Seite für Literaturverzeichnis
\nocite*
\thispagestyle{empty}
\bibliography{literature}
\end{document}		
