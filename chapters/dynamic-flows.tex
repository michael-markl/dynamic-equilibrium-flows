\section{Dynamische Flüsse}\label{sec-dynamic-flows}

In 

\begin{definition}[Netzwerk]
	Ein \emph{Netzwerk} $(G, u, s, t, \tau)$ ist ein gerichteter Graph $G=(V,E)$ mit endlicher Knotenmenge $V$ und Kantenmenge $E$, einer \emph{Quelle} $s\in V$ und einer Senke $t\in V$, sodass alle Knoten von $s$ aus erreichbar sind.
	Jeder Kante $e\in E$ werden eine Kapazität $u_e > 0$ und eine Verzögerungszeit $\tau_e\geq 0$ zugeordnet, sodass alle Zykel $C$ eine positive Gesamtverzögerung $\sum_{e\in C}\tau_e$ haben.
\end{definition}

\begin{definition}
	Der Funktionenraum $\mathfrak{F}_0$ sei die Menge aller Funktionen $g: \R \to \R_{\geq 0}$, die lokal integrierbar bzgl. des Lebesgue-Maßes sind, also $\int_a^b |g(t)| \diff t< \infty$ für beliebige beschränkte Intervalle $(a,b)$ erfüllen, und auf der negativen Achse verschwinden, das heißt, es gilt $g(t)=0$ für alle $t<0$.
\end{definition}

\begin{definition}[Dynamischer Fluss]
	Ein \emph{dynamischer Fluss $f=(f^+, f^-)$} ist ein Paar zweier über die Kanten $E$ eines Netzwerks indizierten Familien mit $f^+_e,f^-_e\in\mathfrak F_0$ für alle $e\in E$.
	Dabei bezeichnen $f_e^+(\theta)$ bzw. $f_e^-(\theta)$ die \emph{Zu- bzw. Abflussrate an Kante $e\in E$ zum Zeitpunkt $\theta\in\R$}.
	
	Der (kumulative) \emph{Zu- bzw. Abfluss an einer Kante $e$ bis zum Zeitpunkt $\theta$} sei definiert durch $F^+_e(\theta):=\int_0^\theta f^+_e(t) \diff t<\infty$ bzw. $F^-_e(\theta):=\int_0^\theta f^-_e(t) \diff t<\infty$.
	
	Die \emph{(Länge der) Warteschlange $z_e(\theta)$ und die Wartezeit $q_e(\theta)$ an Kante $e\in E$ zum Zeitpunkt $\theta\in\R$} seien gegeben durch $z_e(\theta):= F_e^+(\theta) - F_e^-(\theta + \tau_e)$ und $q_e(\theta) := z_e(\theta) / u_e$.
	
	Man beschreibe die \emph{Austrittszeit $T_e(\theta)$ aus einer Kante $e\in E$ bei Eintrittszeit $\theta$}, zu der ein Partikel eine Kante verlässt, die es zum Zeitpunkt $\theta$ betreten hat, als $T_e(\theta):=\theta + q_e(\theta) + \tau_e$.
\end{definition}

\begin{definition}[Zulässiger dynamischer Fluss]
	Ein dynamischer Fluss $f=(f^+, f^-)$ heißt \emph{zulässig}, falls er die folgenden Eigenschaften erfüllt:
	\begin{enumerate}[label=(F\arabic*)]
		\item\label{def-feasible-flow-capacity} Keine Abflussrate übersteigt die Kapazität, d.h. $\forall e\in E, \theta\in\R: f_e^-(\theta)\leq u_e$.
		\item\label{def-feasible-flow-no-negative-flow} Fluss verlässt eine Kante nur, falls er sie zuvor betreten hat,\\ d.h. $\forall e\in E, \theta\in\R: F_e^+(\theta) \geq F_e^-(\theta + \tau_e).$
		\item\label{def-feasible-flow-no-flow-at-node} Bis auf Quelle und Senke erfüllt jeder Knoten $v$ Flusserhaltung,\\
		d.h. $\forall\theta\in\R: \sum_{e\in\delta^-(v)}f^-_e(\theta) - \sum_{e\in\delta^+(v)} f_e^+(\theta) = 0$.\\
		Für die Senke $t$ muss dieser Wert nicht-positiv und für die Quelle $s$ nicht-negativ sein und heißt für $s$ \emph{der Zufluss $d(\theta)$ in das Netzwerk}.
		\item\label{def-feasible-flow-queue-with-capacity} Warteschlangen werden mit der Kapazität der Kante abgebaut,\\ d.h. $\forall e\in E, \theta\in\R: q_e(\theta) > 0 \implies f_e^-(\theta + \tau_e) = u_e$.
	\end{enumerate}
\end{definition}

\todo{FIFO Interpretation}

\begin{proposition}\label{prop-feasible-flow}
	Für eine Kante $e\in E$ und einen zulässigen dynamischen Fluss $f$ gilt:
	\begin{enumerate}[label=(\roman*)]
		\item\label{prop-feasible-flow-T-mon-inc-cont} Die Funktion $\theta \mapsto \theta + q_e(\theta)$ ist monoton wachsend und stetig.
		\item\label{prop-feasible-flow-positive-queue} Für alle $e\in E$ und $\theta\in\R$ ist die Länge der Warteschlange $z_e$ auf dem Intervall $(\theta, \theta + q_e(\theta))$ positiv.
		\item\label{prop-feasible-flow-det-outflow} Zu jeder Zeit $\theta\in\R$ ist $F_e^+(\theta) = F_e^-(T_e(\theta))$.
		\item\label{prop-feasible-flow-queue-delay} Für zwei Zeitpunkte $\theta_1 \leq \theta_2$ mit $\int_{\theta_1}^{\theta_2} f^+_e d\lambda = 0$ und $q_e(\theta_2)>0$ ist $\theta_1 + q_e(\theta_1) = \theta_2 + q_e(\theta_2)$.
	\end{enumerate}
\end{proposition}
\begin{proof}
	In~\ref{prop-feasible-flow-T-mon-inc-cont} folgt die Stetigkeit bereits aus der Stetigkeit von $F_e^+$ und $F_e^-$.
	Für die Monotonie seien $\theta_1 \leq \theta_2$ gegeben.
	Mit der Monotonie von $F_e^+$ und mit $F_e^-(\theta_2 + \tau_e) = F_e^-(\theta_1+\tau_e) + \int_{\theta_1+\tau}^{\theta_2+\tau} f_e^-(t)\diff t\leq F_e^-(\theta_1 + \tau_e) + (\theta_2 - \theta_1)u_e$ gilt: 
	$$
		\theta_1 + q_e(\theta_1)
		= \theta_1 + \frac{F_e^+(\theta_1) - F_e^-(\theta_1 + \tau_e)}{u_e}
		\leq \theta_1 + \frac{F_e^+(\theta_2) - F_e^-(\theta_1+\tau_e)}{u_e} \leq \theta_2 + q_e(\theta_2).
	$$
	
	Für $\theta'\in (\theta, \theta+q_e(\theta))$ gilt also $\theta' + q_e(\theta') \geq \theta + q_e(\theta)$, womit $q_e(\theta') \geq (\theta - \theta') + q_e(\theta) > q_e(\theta)$ gerade Aussage (ii) beweist.
	
	Aussage (iii) folgt dann mit~\ref{def-feasible-flow-queue-with-capacity} und Aussage~(ii), denn es gilt 
	$\int_{\theta}^{\theta + q_e(\theta)}f_e^-(t + \tau_e) \diff t = q_e(\theta)  u_e = z_e(\theta)$ und damit ist $F_e^-(T_e(\theta)) = F_e^-(\theta+\tau_e) + \int_{\theta+\tau_e}^{\theta+\tau_e+q_e(\theta)}f_e^-(t)\diff t = F_e^+(\theta)$.
	
	Es bleibt Aussage (iv) zu zeigen:
	Für alle $\theta'\in [\theta_1, \theta_2]$ gilt $F_e^+(\theta') = F_e^+(\theta_2)$.
	Also ist die Warteschlange $z_e(\theta') = F_e^+(\theta_2) - F_e^-(\theta' + \tau_e) \geq z_e(\theta_2) > 0$ positiv und nach~\ref{def-feasible-flow-queue-with-capacity} gilt $f_e^-(\theta' + \tau_e)=u_e$.
	Die Differenz der Warteschlangen ist somit nach Bedingung~\ref{def-feasible-flow-queue-with-capacity} 
	$z_e(\theta_1)-z_e(\theta_2)=-F^-_e(\theta_1 + \tau_e) + F^-_e(\theta_2 + \tau_e) = (\theta_2 - \theta_1)u_e$, was $q_e(\theta_1) - q_e(\theta_2) = \theta_2 - \theta_1$ impliziert.
\end{proof}