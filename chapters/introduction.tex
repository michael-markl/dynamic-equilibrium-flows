\section{Einführung}\label{introduction}

\newcommand{\R}{\mathbb{R}}

\begin{definition}[strategisches Spiel]
	Ein \emph{strategisches Spiel} $\mathcal{G}$ ist ein Tupel $(P, X, \theta)$, wobei $P$ eine Menge von Spielern, $X$ die Menge der Strategien  und $\theta=(\theta_p)_{p_\in P}$ die Familie der Auszahlungsfunktionen der Spieler ist, wobei $\theta_p$ eine Funktion $X^P\to\R$ für jeden Spieler $p\in P$ ist.
\end{definition}

Man beachte, dass in dieser Definition eines strategischen Spiels die Spieler eine gemeinsame Strategiemenge haben.

\begin{definition}[statisches Flussnetzwerk]
	Ein \emph{statisches Flussnetzwerk} $\mathcal{N}:=(G,s,t,\mu)$ ist ein Graph $G:=(V,E)$ mit Knoten $V$ und Kanten $E$, einer \emph{Quelle} $s\in V$, einer \emph{Senke} $t\in V$ zusammen mit einem \emph{$d$-wertigen statischen Fluss} $\mu: \mathcal{P} \to \R$, wobei $\mathcal{P}$ die Menge aller einfachen $s$-$t$-Pfade in G ist und $d=\sum_{p\in\mathcal{P}}\mu(p)$ gilt.
\end{definition}

\begin{definition}[statisches Routenplanungsspiel]
	Sei ein Graph $G:=(V,E)$ mit Knoten $V$ und Kanten $E$, einer Quelle $s\in V$, einer Senke $t\in V$ und einem Zufluss $d\in\R_+$ gegeben.
	Ein \emph{statisches Routenplanungsspiel} ist ein strategisches Spiel mit  Spielermenge $[0,d]\subseteq\R$, wobei ein Spieler als \emph{Flusspartikel} bezeichnet wird, und mit Strategiemenge $\mathcal{P}$ aller $s$-$t$-Pfade in $G$.
	Die Auszahlungsfunktion $\theta_x$ gibt die Kosten des Partikels $x$ in Abhängigkeit des Flusses an.
	
	Der Fluss $\mu$ heißt Nashfluss, falls $l_P(\mu)=min_{P'\in\mathcal{P}}l_{P'}(\mu)$ für alle $P\in\mathcal{P}$ mit $\mu_P > 0$ gilt.
\end{definition}

\section{Dynamische Flüsse mit Zeithorizont}


Betrachte Zeitraum $[0,T]$ mit $T\in\R_+$.
$d$ Spieler erscheinen an Quelle $s$ über Zeitraum von $0$ bis $T$

\begin{definition}[Netzwerk]
	Ein \emph{Netzwerk} ist ein gerichteter Graph $G=(V,E)$ mit endlicher Knotenmenge $V$ und Kantenmenge $E\subseteq V\times V$, einer \emph{Quelle} $s\in V$ und einer Senke $t\in V$.
	Jeder Kante $e\in E$ werden eine Kapazität $u_e\geq 0$ und eine Verzögerungszeit $\tau_e\geq 0$ zugeordnet.
\end{definition}

Für ein Netzwerk $\mathcal{N}$ bezeichne $\mathcal{P}$ die Menge aller $s$-$t$-Pfade.

\begin{definition}[Fluss über Zeit]
	Ein Fluss über Zeit $f=(f^+, f^-)$ ist ein Paar zweier über die Kanten $E$ indizierten Familien von Lebesgue-integrierbaren Funktionen $f^+_e,f^-_e: \R \to \R_{\geq 0}$ an $e$ für alle $e\in E$.
	
	Dabei bezeichnen $f_e^+(\theta)$ die \emph{Einflussrate an $e$ zum Zeitpunkt $\theta\in\R$} und $f_e^-(\theta)$ die \emph{Ausflussrate aus $e$ zum Zeitpunkt $\theta\in\R$} für $e\in E$.
	
	Der (kumulative) \emph{Einfluss bzw. Ausfluss an einer Kante $e$ bis zum Zeitpunkt $\theta$} sei definiert durch $F^+_e(\theta):=\int_{[0,\theta)} f^+_e d\lambda$ bzw. $F^-_e(\theta):=\int_{[0,\theta)} f^-_e d\lambda$.
\end{definition}

\begin{definition}[Zulässiger Fluss über Zeit]
	Sei ein Fluss über Zeit $f=(f^+, f^-)$ gegeben. $f$ heißt zulässig, falls
	\begin{enumerate}
		\item keine Ausflussrate die Kapazität übersteigt, d.h. $\forall e\in E, \theta\in\R: f_e^-\leq u_e$, und
		\item Fluss eine Kante nur verlässt, falls er die Kante zuvor betreten hat, d.h. $\forall e\in E, \theta\geq 0: F_e^+(\theta) \geq F_e^-(\theta + \tau_e)$, und
		\item TODO
	\end{enumerate}
\end{definition}
